\documentclass[a4paper,11pt]{article}
\pdfoutput=1 % if your are submitting a pdflatex (i.e. if you have
             % images in pdf, png or jpg format)

\usepackage{jheppub} % for details on the use of the package, please
                     % see the JHEP-author-manua

\usepackage{graphicx,epsfig,wrapfig,amssymb,color,amsmath,bm,breqn}

\usepackage[T1]{fontenc} % if needed

\allowdisplaybreaks

%\usepackage[utf8]{inputenc} 


%============= FORMATING (A4) ======================================
%\setlength{\textwidth}{17cm}
%\setlength{\textheight}{26.2cm}
%\setlength{\topmargin}{-1cm}
%\renewcommand{\baselinestretch}{1.5}

\bibliographystyle{JHEP}

\usepackage{lineno} % page numbers

%=================== USE COLOURS =====================================

 
\usepackage{hyperref} %links in the table of contents
\hypersetup{
    colorlinks,
    citecolor=black,
    filecolor=black,
    linkcolor=black,
    urlcolor=black
}



\newcommand{\green}[1]{{\color{green} #1}}
\newcommand{\blue}[1]{{\color{blue} #1}}
\newcommand{\red}[1]{{\color{red} #1}}

%===================  NEW COMMANDS  ==================================
\newcommand{\be}{\begin{equation}}
\newcommand{\ee}{\end{equation}}
\newcommand{\ba}{\begin{eqnarray}}
\newcommand{\ea}{\end{eqnarray}}
\newcommand{\la}{\langle}
\newcommand{\ra}{\rangle}
\newcommand{\di}{ {\rm d} }
\newcommand{\with}[3]{{\Biggl|{\renewcommand*{\arraystretch}{0.8}
	\begin{array}{l} 
	\phantom{X}\\
	\mbox{\scriptsize ${#1}$}\\
	\mbox{\scriptsize ${#2}$}\\
	\mbox{\scriptsize #3}\end{array}}}}
%
%\newcommand{\xbj}{x}                   % Bjorken variable
%\newcommand{z}{z}
%\newcommand{\de}{d}                    % integration measure
%\newcommand{\ii}{i}                    % imaginary unit
%\newcommand{\eps}{\epsilon}
%\newcommand{\Tr}{\rm Tr} % Trace operator
%\newcommand{\half}{ {\textstyle\frac{1}{2}} }
\newcommand{\slim}{\mskip 1.5mu}       % small space in math
%\newcommand{\bm}{\boldsymbol}
%\newcommand{\bm}{\boldmath}
%\newcommand{\lf}{\left}
%\newcommand{\rg}{\right}
%\newcommand{\h}{\hat{\vec{h}}}
\def\T{_{_T}}
\def\C{_{_C}}

%Editing commands
\newcommand{\TD}[1]{{{\bf TODO:} \color[rgb]{0.65,0,0.15} #1}}
\newcommand{\AP}[1]{{\bf AP:} {\color{red} #1}}

\definecolor{darkgreen}{rgb}{0,0.65,0}
\definecolor{orange}{rgb}{1,0.4,0.1}
\newcommand{\darkgreen}[1]{{\color{darkgreen} #1}}
\newcommand{\orange}[1]{{\color{orange}#1}}

\newcommand{\ak}[1]{\orange{#1}}
\newcommand{\ps}[1]{\blue{#1}}
\newcommand{\gs}[1]{{\color[rgb]{0.65,0,0.65}#1}}
\newcommand{\ms}[1]{\darkgreen{#1}}
\newcommand{\kt}[1]{{\color[rgb]{0.1,0.5,0}#1}}


\newcommand{\SB}[1]{{\bf SB:} {\color{magenta} #1}}

\newcommand{\asym}[2]{{A_{#1}^{#2}}}
\newcommand{\asympre}[2]{{A_{#1,\langle y\rangle}^{#2}}}





 
 % PETER'S DEFINITIONS
%\def\bfkperp{{\bf p}_T} 
%\def\bfpperp{{\bf k}_T} 
%\def\bfhp{{\bf h}_\perp} 
%\def\bfPhperp{{\bf P}_{h\perp}}
%\def\Phperp{P_{h\perp}}

%\def\kperp{p_T}
%\def\pperp{k_T}
%\def\avkperp{\la \kperp^2 \ra}
%\def\avpperp{\la \pperp^2 \ra}

\def\bflperp{{\bm \ell}_\perp} 

 % STANDARD DEFINITIONS
 %INT conventions:
\newcommand{\vect}[1]{\ensuremath{{\bm{#1}}}}
\def\bfkperp{{\bm k}_\perp} 
\def\bfpperp{{\bm P}_\perp} 
\def\bfhp{\hat{\bm h}} 
\def\bfPhperp{{\bm P}_{hT}}
\def\Phperp{P_{hT}}

\def\kperp{k_\perp}
\def\pperp{P_\perp}
\def\avkperp{\la \kperp^2 \ra}
\def\avpperp{\la \pperp^2 \ra}


\newcommand*{\FigPath}{./figs}%  
\newcommand*{\BibPath}{.}%  



%===================  TITLE, AUTHORS, AFFILIATIONS ===================

 
\preprint{\red{version ``00y''}; JLAB-THY-18-XXXX}

\title{	Semi-Inclusive Deep Inelastic Scattering 
	in Wandzura-Wilczek-type approximation}

\author[a]{S.~Bastami}
\author[b]{H.~Avakian}
\author[c]{A.~V.~Efremov} 
\author[d,e]{A.~Kotzinian} 
\author[f]{B.~U.~Musch} 
\author[k]{B.~Parsamyan} 
\author[g,b]{A.~Prokudin} 
\author[h]{M.~Schlegel} 
\author[i]{G.~Schnell} 
\author[a,j]{P.~Schweitzer} 
\author[a]{K.~Tezgin}


\affiliation[a]{Department of Physics, University of Connecticut, 
	Storrs, CT 06269, U.S.A.}
\affiliation[b]{Thomas Jefferson National Accelerator Facility, 
	Newport News, VA 23606, U.S.A.}
\affiliation[c]{Joint Institute for Nuclear Research, Dubna, 
	141980 Russia}
\affiliation[d]{Yerevan Physics Institute,  Alikhanyan Brothers St.,
	375036 Yerevan, Armenia}
\affiliation[e]{INFN, Sezione di Torino, 
	10125 Torino, Italy}
\affiliation[f]{Institut f\"ur Theoretische Physik, Universit\"at 
  	Regensburg, 93040 Regensburg, Germany}
\affiliation[g]{Division of Science, Penn State Berks, Reading, 
	PA 19610, USA}
\affiliation[k]{CERN, 1211 Geneva 23, Switzerland}
\affiliation[h]{Department of Physics, New Mexico State University, 
	Las Cruces, NM 88003-001, USA}
\affiliation[i]{Department of Theoretical Physics, University of the Basque 
	Country UPV/EHU, 48080 Bilbao, Spain, and
	IKERBASQUE, Basque Foundation for Science, 48013 Bilbao, Spain}
\affiliation[j]{Institute for Theoretical Physics, Universit\"at T\"ubingen,
	D-72076 T\"ubingen, Germany} % Auf der Morgenstelle 14, 

% e-mail addresses: one for each author, in the same order as the authors
\emailAdd{saman.bastami@uconn.edu}
\emailAdd{avakian@jlab.org}
\emailAdd{efremov@theor.jinr.ru}
\emailAdd{aram.kotzinian@cern.ch}
\emailAdd{bmusch@b-mu.de}
\emailAdd{bakur@cern.ch}
\emailAdd{prokudin@jlab.org}
\emailAdd{schlegel@nmsu.edu}
\emailAdd{gunar.schnell@desy.de}
\emailAdd{peter.schweitzer@phys.uconn.edu}
\emailAdd{kemal.tezgin@uconn.edu}




%\date{\today}
%===================  PREPRINT NUMBER, JOURNAL =======================
%\
%===================  ABSTRACT =======================================
%\begin{abstract}

\abstract{
We present the complete cross-section 
for the production of unpolarized hadrons in semi-inclusive
deep-inelastic scattering up to power-suppressed ${\cal O}(1/Q^2)$ terms in
the Wandzura--Wilczek-type approximation which consists in systematically
assuming that $\bar{q}gq$--terms are much smaller than $\bar{q}q$--correlators.
We compute all twist-2 and twist-3 structure functions and the corresponding 
asymmetries, and discuss the applicability of the Wandzura--Wilczek-type 
approximations on the basis of available data. We make predictions which 
can be tested by data from Jefferson Lab, COMPASS, HERMES, and the future 
Electron-Ion Collider. The results of this paper can be readily used for 
phenomenology and for event generators, and will help to improve our 
understanding of the TMD theory beyond leading twist.
} 
%\end{abstract}
%\pacs{13.88.+e, % Polarization in interactions and scattering
%      13.85.Ni, % Inclusive production with identified hadrons
%      13.60.-r, % Photon and charged-lepton interactions with hadrons
 %     13.85.Qk} % Hadron-induced inclusive production with identified leptons, 
                % photons, or other nonhadronic particles (energy > 10 GeV)
\keywords{
	Wandzura--Wilczek approximation, 
	semi-inclusive deep inelastic scattering,
	transverse momentum dependent distribution and fragmentation 
	functions, spin and azimuthal asymmetries, leading and subleading twist}

 
%%%%%%%%%%%%%%%%%%%%%%%%%%%%%%%%%%%%%%%%%%
\begin{document}
%%%%%%%%%%%%%%%%%%%%%%%%%%%%%%%%%%%%%%%%%%

\linenumbers % line numbers

\maketitle

\flushbottom

\subsection*{Color Legend}

General comments: 
	\TD{dark red color, command: {$\backslash{\rm TD}\{\dots\}$}}\\
Alexei's corrections: 
	\AP{redish color, command: {$\backslash{\rm AP}\{\dots\}$}}\\
Aram's corrections: 
	\ak{orange color, command: {$\backslash{\rm ak}\{\dots\}$}}\\
Gunar's corrections: 
	\gs{violett color, command: {$\backslash{\rm gs}\{\dots\}$}}\\
Kemal's corrections: 
	\kt{greenish color, command: {$\backslash{\rm kt}\{\dots\}$}}\\
Marc's corrections: 
	\ms{green color, command: {$\backslash{\rm ms}\{\dots\}$}}\\
Peter's corrections: 
	\ps{blue color, command: {$\backslash{\rm ps}\{\dots\}$}}\\
Anatoli Vasilievich, Bakur, Bernie, Saman: \\
	corrections already introduced in black color \& confirmed by e-mail.


\newpage
%======= SECTION 1: INTRODUCTION =====================================
\section{Introduction}
\label{Sec-1:introduction}

A great deal of what is known about the quark-gluon structure of 
nucleon is due to studies of parton distribution functions (PDFs) 
in deep-inelastic reactions. Leading--twist PDFs  tell us  how likely 
it is to find an unpolarized parton 
(described by PDF $f_1^a(x)$, $a=q,\,\bar q,\,g$) 
or a longitudinally polarized parton 
(described by PDF $g_1^a(x)$, $a=q,\,\bar q,\,g$)
in a fast-moving unpolarized or longitudinally polarized nucleon, 
which carries the fraction $x$ of the nucleon momentum.
This information depends on the ``resolution (renormalization) scale'' 
associated with the hard scale $Q$ of the process.
Although the PDFs  $f_1^a(x)$ and $g_1^a(x)$ continue being the 
subject of intense research (small-$x$, large-$x$, helicity sea 
and gluon distributions) they can be considered as rather 
well-known, and the frontier has been extended in the last years 
to go beyond the one-dimensional picture offered by those PDFs.

One way to do this consists in a systematic inclusion of transverse 
parton momenta $\kperp$ whose effects manifest themselves in terms of
transverse momenta of the reaction products in the final state.
If these transverse momenta are much smaller than the hard scale $Q$
of the process, the formal description is given in terms of 
transverse momentum dependent distribution functions (TMDs) 
and fragmentation functions (FFs)
which are defined in terms of quark-quark correlators 
\cite{Kotzinian:1994dv,Mulders:1995dh,Boer:1997nt,Goeke:2005hb,Bacchetta:2006tn},
and depend on two independent variables: the fraction $x$ of 
nucleon momentum carried by parton, and intrinsic transverse 
momentum $\kperp$ of the parton.
Being a vector in the plane transverse with respect to the
light-cone direction singled out by the hard momentum flow in the process,
$\kperp$ allows us to access novel information on the nucleon spin structure 
through correlations of $\kperp$ with nucleon and/or parton spin. The 
latter is a well-defined concept for twist-2 TMDs interpreted in 
the infinite momentum frame or in lightcone quantization formalism.

One powerful tool to study TMDs are measurements of the 
semi-inclusive deep-inelastic scattering (SIDIS) process.
By exploring various possibilities for the lepton beam and target 
polarizations unambiguous information can be accessed on the 8 leading--twist 
TMDs \cite{Boer:1997nt} and, if one assumes factorization, on certain
linear combinations of the 16 subleading--twist TMDs 
\cite{Goeke:2005hb,Bacchetta:2006tn}.
	It is important to stress that this information could not have 
	been obtained without advances in target polarization techniques 
	employed in the HERMES, COMPASS and JLab experiments 
	\cite{Stock:1994vv,Crabb:1997cy,Goertz:2002vv}.
Complementary information can be obtained 
from the Drell-Yan process \cite{Arnold:2008kf}, 
and $e^+e^-$ annihilation \cite{Metz:2016swz}.

In QCD the TMDs are independent functions. Each TMD contains unique
information on a different aspect of the nucleon structure. 
Twist-2 TMDs have partonic interpretations. Twist-3 TMDs 
give insights on quark-gluon correlations in the nucleon
\cite{Miller:2007ae,Burkardt:2007rv,Burkardt:2009rf}. 
Besides positivity constraints \cite{Bacchetta:1999kz} 
there is little model-independent information on TMDs. 
An important question with practical applications is:
do useful {\sl approximations} for TMDs exist? 
Experience from collinear PDFs encourages to explore this possibility: 
the twist-3 $g_T^a(x)$ and $h_L^a(x)$ can be expressed in terms of 
contributions from twist-2 $g_1^a(x)$ and $h_1^a(x)$, and additional 
quark-gluon-quark ($\bar{q}gq$) correlations or current quark mass 
terms \cite{Wandzura:1977qf,Jaffe:1991ra}
(the index $a=q\,,\bar q$ does not include gluons for
$h_1^a$, $h_L^a$ and other chiral odd TMDs below).
We shall refer to the latter generically as $\bar{q}gq$--terms, keeping in 
mind one deals in each case with matrix elements of different operators.
The $\bar{q}gq$--correlations contain new insights on hadron structure 
which are worthwhile exploring for their own sake, 
see \cite{Jaffe:1989xx} on $g_T^a(x)$.

The striking observation is that the $\bar{q}gq$--terms in $g_T^a(x)$ 
and $h_L^a(x)$ are small: theoretical mechanisms predict this 
\cite{Balla:1997hf,Dressler:1999hc,Gockeler:2000ja,Gockeler:2005vw}, 
and in the case of $g_T^a(x)$ data confirm or are compatible with these 
predictions \cite{Abe:1998wq,Anthony:2002hy,Airapetian:2011wu}.
This approximation (``neglect of $\bar{q}gq$--terms'') is commonly 
known as Wandzura--Wilczek (WW) approximation \cite{Wandzura:1977qf}.
The possibility to apply this type of approximation also to TMDs has 
been explored in specific cases in \cite{Kotzinian:1995cz,Kotzinian:1997wt,
Kotzinian:2006dw,Avakian:2007mv,Metz:2008ib,Teckentrup:2009tk,Tangerman:1994bb}.
In both cases, PDFs and TMDs, one basically assumes that the 
contributions from $\bar{q}gq$--terms can be neglected with respect to 
$\bar{q}q$--terms. But the nature of the omitted matrix elements is 
different, and in the context of TMDs one often prefers to speak 
about WW-type approximations.

The present work is the first study of all SIDIS structure functions up to 
twist-3 in a unique approach. Our results are of importance for experiments 
prepared in the near-term at Jefferson Lab (JLab) with 12$\,$GeV beam energy
upgrade and COMPASS or proposed in the long-term (Electron Ion Collider), 
and provide helpful input for the development of Monte Carlo event generators 
\cite{Avakian:2015vha}.

Our predictions will either be confirmed within the expected accuracy
or will not be supported by data. In either case our work will provide
a useful benchmark, and call for dedicated theoretical 
studies to explain (i) why the pertinent $\bar{q}gq$--terms are 
small or (ii) why they are sizable. In either case our results will 
deepen the understanding of  $\bar{q}gq$--correlations, pave the 
way towards testing the validity of the TMD factorization approach 
at subleading twist, and help us to guide further developments.

In this work, after introducing the SIDIS process and defining TMDs and 
FFs (Sec.~\ref{Sec-2:SIDIS+TMDs+FF}), we shall introduce the WW(-type) 
approximations, and review what is presently known about them 
from experiment and theory (Sec.~\ref{Sec-3:WW}).
We will show that under the assumption of the validity of these approximations 
all leading and subleading SIDIS structure functions are described in terms of 
a basis of 6 TMDs and 2 FFs (Sec.~\ref{Sec-4:SIDIS-in-WW-approximation}),
and review how these basis functions describe available data 
(Sec.~\ref{Sec-5:twist-2+basis}).
We will systematically apply the WW and/or WW-type approximations
to SIDIS structure at leading (Sec.~\ref{Sec-6:twist-2-and-WW}) 
and subleading (Sec.~\ref{Sec-7:twist-3-and-WW}) twist, and
conclude with a critical discussion (Sec.~\ref{Sec-8:conclusions}).
The Appendices \ref{App:basis} and \ref{App:factor} contain 
technical details. 
	In App.~C we describe an open source 
	package publicly available on \texttt{github.com}\footnote{\href{%
	https://github.com/prokudin/WW-SIDIS}{
	https://github.com/prokudin/WW-SIDIS}} 
	implemented in \texttt{mathematica}~\cite{Mathematica} 
	(already available) and \texttt{Python} 
	(to be released in the near future).



\newpage
%======= SECTION 2: SIDIS & TMDs & FFs ===============================
\section{The SIDIS process in terms of TMDs and FFs}
\label{Sec-2:SIDIS+TMDs+FF}

In this section we review the description of the SIDIS process, 
define structure functions, PDFs, TMDs, FFs and recall how they
describe the SIDIS structure functions.

\subsection{The SIDIS process}
\label{Sec-2.1:SIDIS+structure-functions}

%------ BEGIN FIGURE 1: Kinematics of SIDIS ---------------------------
\begin{wrapfigure}[9]{RD}{8cm}
\vspace{-7mm}
\centering
	\includegraphics[width=7.2cm]{\FigPath/Fig02a-kin-SIDIS-UT-NEW.pdf}
        \caption{\label{fig-kin-SIDIS}
    	Kinematics of SIDIS process $lN\to l^\prime h X$
	in 1-photon exchange approximation.}
\vspace{-5mm}
\end{wrapfigure}
%------ END FIGURE 1 -------------------------------------------------

The SIDIS process  $lN\to l^\prime h X$ is sketched in 
Fig.~\ref{fig-kin-SIDIS}. Here $l$ and $P$ are momenta of incoming 
lepton and nucleon, $l^\prime$ and $P_h$ are the momenta of the outgoing
lepton and produced hadron. The virtual photon momentum $q=l-l^\prime$ 
selects the z-axis, and $l^\prime$ points in the direction of the x-axis 
from which azimuthal angles are counted. The relevant kinematic invariants 
are
\ba
   x  = \frac{Q^2}{2\,P\cdot  q}, \;\;
   y = \frac{P \cdot  q}{P \cdot  l}, \;\;
   z = \frac{P \cdot  P_h}{P\cdot  q}, \;\;
%   \gamma = \frac{2 M x}{Q} \; .
   Q^2=-q^2.
\label{eq:xyz}\;\;\;\;\ea
Note that we consider the production of unpolarized hadrons in DIS of 
charged leptons (electrons, positrons, muons) at $Q^2 \ll M_Z^2$ 
in the single photon exchange approximation.
In addition to $x$, $y$, $z$ the cross section is also differential 
in the azimuthal angle $\phi_h$ of the produced hadron, the square 
of its momentum component $\Phperp$ perpendicular with respect to the 
virtual photon momentum. 
In principle the cross section is also differential with respect to the
azimuthal angle $\psi_l$ characterizing the overall orientation of the 
lepton scattering plane in a fixed lab frame. The origin of this 
angle can be chosen arbitrarily. For a transversely polarized target 
it is convenient to choose the origin of the angle $\psi_l$ in the 
direction of $\vec{S}_T$.
It is convenient to define the unpolarized 
lepton--quark scattering subprocess cross section
\ba\label{Eq:sigma0-FUU}
	\frac{d \hat{\sigma}}{dy} 
	= 
	\frac{4 \pi \alpha_{em}^2}{x\,y\,Q^2}
	\biggl(1-y+\frac12y^2\biggr)\;
\ea

To leading order in $1/Q$ the SIDIS cross-section is given by  
\begin{subequations}\ba\hspace{-1cm}
   &&  \frac{d^6\sigma_{\rm leading}}{dx\,dy\,dz\,d\psi_l\,d\phi_h\,d \Phperp^2}
   =	 \frac{1}{4\pi}\;\frac{d \hat{\sigma}}{dy}\;F_{UU}(x,z,\Phperp^2)
        \Biggl\{\;1 
        + \cos(2\phi_h)\,   p_1\,A_{UU}^{\cos(2\phi_h)} \nonumber\\
   && \hspace{2cm}
  	+ S_L\sin(2\phi_h)\,p_1\,A_{UL}^{\sin(2\phi_h)}    
	+ \lambda\,S_L\,    p_2\,A_{LL}  \phantom{\frac11}\nonumber\\
   && \hspace{2cm}
	+ \lambda\,S_T\cos(\phi_h-\phi_S)\,p_2\,A_{LT}^{\cos( \phi_h-\phi_S)}
       	+ S_T\sin( \phi_h-\phi_S)\, A_{UT}^{\sin( \phi_h-\phi_S)} \phantom{\frac11}
	\nonumber\\ 
   && \hspace{2cm}
	+ S_T\sin( \phi_h+\phi_S)\,p_1\,A_{UT}^{\sin( \phi_h+\phi_S)} 
        + S_T\sin(3\phi_h-\phi_S)\,p_1\,A_{UT}^{\sin(3\phi_h-\phi_S)}\Biggr\}
    \hspace{1cm} \label{Eq:SIDIS-leading}
\ea
where $F_{UU}$ is the structure function due to transverse
polarization of the virtual photon (sometimes denoted as $F_{UU,T}$),
and we systematically neglect $1/Q^2$ corrections in kinematic factors 
and a structure function (sometimes denoted as $F_{UU,L}$) arising from
longitudinal polarization of the virtual photon (and analogous for the 
structure function $\propto S_T\,\sin( \phi_h-\phi_S)$, see below).
The structure functions 
(and asymmetries) also depend on $Q^2$ via the scale dependence of 
TMDs and FFs, which we do not show in formulas throughout this work.

At subleading order in the $1/Q$ expansion one has
\ba\hspace{-1cm}
   &&   \frac{d^6\sigma_{\rm subleading}}{dx\,dy\,dz\,d\psi_l\,d\phi_h\,d \Phperp^2}
   =	\frac{1}{4\pi} \; \frac{d \hat{\sigma}}{dy} \; F_{UU}(x,z,\Phperp^2)
        \Biggl\{ 
          \cos(\phi_h)\,p_3\,A_{UU}^{\cos(\phi_h)}
	\nonumber\\ 
   && \hspace{2cm}
	+ \lambda\sin(\phi_h)\,p_4\,A_{LU}^{\sin(\phi_h)}+ S_L\sin(\phi_h)\,p_3\,A_{UL}^{\sin(\phi_h)}    
	+ S_T\sin(\phi_S)\,p_3\,A_{UT}^{\sin(\phi_S)} \phantom{\frac11}\nonumber\\ 
   && \hspace{2cm}
	+ S_T\sin(2\phi_h-\phi_S)\,p_3\,A_{UT}^{\sin(2\phi_h-\phi_S)}
        + \lambda\,S_L\cos(\phi_h)\,p_4\,A_{LL}^{\cos(\phi_h)}\phantom{\frac11}
	\nonumber\\
   && \hspace{2cm}
  	+ \lambda\,S_T\cos(\phi_S)\,p_4\,A_{LT}^{\cos(\phi_S)}
        + \lambda\,S_T\cos(2\phi_h-\phi_S)\,p_4\,A_{LT}^{\cos(2\phi_h-\phi_S)}
	\Biggr\}
   \hspace{1cm} \label{Eq:SIDIS-subleading}
\ea\end{subequations}
Neglecting $1/Q^2$ corrections, the kinematic prefactors $p_i$ are given by 
% \ba\label{Eq:y-prefactors}
% &&	p_1 = \varepsilon \equiv  
%        \frac{1-y -\frac{1}{4}\slim \gamma^2 y^2}{1-y
%        + \frac{1}{2}\slim y^2 +\frac{1}{4}\slim \gamma^2 y^2} 
%        \approx \frac{1-y}{1-y+\frac12\,y^2} 		\;,\nonumber\\
% &&	p_2 = \sqrt{1-\epsilon^2} \approx
%        \frac{y(1-\frac12\,y)}{1-y+\frac12\,y^2} 	\;,\nonumber\\
% &&	p_3 =  \sqrt{2\varepsilon(1+\varepsilon)}
%        \approx \frac{(2-y)\sqrt{1-y}}{1-y+\frac12\,y^2}\;,\nonumber\\
% &&	p_4 =  \sqrt{2\varepsilon(1-\varepsilon)}
%        \approx \frac{y\sqrt{1-y}}{1-y+\frac12\,y^2} 	\;,\;\;\;
% \ea
\be\label{Eq:y-prefactors}
	p_1 = \frac{1-y}{1-y+\frac12\,y^2} 		\, , \;\;\;
	p_2 = \frac{y(1-\frac12\,y)}{1-y+\frac12\,y^2}	\, , \;\;\;
	p_3 = \frac{(2-y)\sqrt{1-y}}{1-y+\frac12\,y^2} 	\, , \;\;\;
	p_4 = \frac{y\sqrt{1-y}}{1-y+\frac12\,y^2}     	\, .
\ee
and the asymmetries are defined as 
\be
	A_{XY}^{\rm weight}\equiv A_{XY}^{\rm weight}(x,z,\Phperp)=
	\frac{F_{XY}^{\rm weight}(x,z,\Phperp)}{F_{UU}(x,z,\Phperp)}.
\ee
Hereby the first index $X=U(L)$ denotes the unpolarized beam
(longitudinally polarized beam with helicity $\lambda$).
The second index $Y=U(L,T)$ refers to the target which can be unpolarized
(or longitudinally, transversely polarized with respect to virtual photon).
The superscript ``weight'' indicates the azimuthal dependence with no index 
indicating an isotropic angular distribution of the produced hadrons.

In the partonic description the structure functions in 
(\ref{Eq:SIDIS-leading})    are ``twist-2.'' Those in
(\ref{Eq:SIDIS-subleading}) are ``twist-3'' and contain a
factor $M/Q$ in their definitions, see below. In our treatment to $1/Q^2$ 
accuracy we neglect two structure functions due to longitudinal virtual 
photon polarization which contribute at order ${\cal O}(M^2/Q^2)$ in the 
partonic description of the process, one being $F_{UU,L}$ and the other 
contributing to the $\sin(\phi_h-\phi_S)$ angular distribution 
\cite{Bacchetta:2006tn}.

Experimental collaborations often define asymmetries in terms of counts 
$N(\phi_h)$. This means the kinematic prefactors $p_i$ and $1/(x\,y\,Q^2)$ 
are included in the numerators or denominators of the asymmetries which
are averaged over $y$ within experimental kinematics. We will call the 
corresponding asymmetries $\asympre{XY}{\rm weight}$.
For instance, in unpolarized case one has 
\be
	N(x,\dots,\phi) = \frac{N_0(x,\dots)}{2\pi} \biggl(1
		+ \cos\phi\;\asympre{UU}{\cos\phi_h}(x,\dots)
		+ \cos2\phi\;\asympre{UU}{\cos2\phi_h}(x,\dots)\Biggr)
\ee
where $N_0$ denotes the total ($\phi_h$--averaged) number of counts 
and the dots indicate further kinematic variables in the kinematic 
bin of interest (which may also be averaged over).
It would be preferable if asymmetries were analyzed with known kinematic 
prefactors divided out on event-by-event basis. One could then directly 
compare asymmetries $\asym{XY}{\rm weight}$ measured in different 
experiments and kinematics, and focus on effects of evolution 
or power suppression for twist-3. In practice, often the kinematic 
factors were included. We will define and comment on the explicit 
expressions as needed.

For completeness we remark that after integrating the cross section
over transverse hadron momenta one obtains 
\begin{subequations}\ba
     	\frac{d^4\sigma_{\rm leading}}{dx\,dy\,dz\,d\psi_l}
   &=&	 \frac{1}{2\pi}\;\frac{d \hat{\sigma}}{dy} \; F_{UU}(x,z) 
        \Biggl\{\;1 + \lambda\,S_L\,    p_2\,A_{LL} \Biggr\}
    	\label{Eq:SIDIS-leading-integrated} \\
	\frac{d^4\sigma_{\rm subleading}}{dx\,dy\,dz\,d\psi_l}
   &=&	 \frac{1}{2\pi}\;\frac{d \hat{\sigma}}{dy} \; F_{UU}(x,z) 
        \Biggl\{ S_T\sin(\phi_S)\,p_3\,A_{UT}^{\sin(\phi_S)} 
  	+ \lambda\,S_T\cos(\phi_S)\,p_4\,A_{LT}^{\cos(\phi_S)}
          \Biggr\}
     \hspace{1cm} \label{Eq:SIDIS-subleading-integrated}
\ea\end{subequations}
where (and analogous for the other structure functions)
\be\label{Eq:FUU-integrated}
	F_{UU}(x,z) = \int d^2\Phperp\;F_{UU}(x,z,\Phperp)
\ee
and the asymmetries are defined as
\be
	A_{XY}^{\rm weight}(x,z) = \frac{F_{XY}^{\rm weight}(x,z)}{F_{UU}(x,z)}\,.
\ee

The connection of ``collinear'' SIDIS structure functions
in (\ref{Eq:SIDIS-leading-integrated},~\ref{Eq:SIDIS-subleading-integrated})
to those known from inclusive DIS is established by integrating over $z$
and summing over hadrons as 
\begin{subequations}\begin{alignat}{4}
	&\sum\limits_h\int d z\;z\;F_{UU}(x,z) 
	&\equiv	&&	& 2\,x\,F_1(x) \;, 
	\label{Eq:DIS-F1}\\ % x\sum_q e_q^2\,f_1^q(x) 
	&\sum\limits_h\int d z\;z\;F_{LL}(x,z) 
	&\equiv && 	& 2\,x\,g_1(x) \;, 
	\label{Eq:DIS-g1}\\ % x\sum_q e_q^2\,g_1^q(x) 
	&\sum\limits_h\int d z\;z\;F_{LT}^{\cos\phi_S}(x,z) \;\;
	&\equiv && \;\; -\,\gamma\; & 2\,x\biggl(g_1(x)+g_2(x)\biggr) \;, 
	\label{Eq:DIS-gT}\\ % -\,\frac{2M_Nx^2}{Q} \sum_q e_q^2\,g_T^q(x) 
	&\sum\limits_h\int d z\;z\;F_{UT}^{\sin\phi_S}(x,z) 
	&=      && 	    & \;\; 0 \, ,
	\label{Eq:DIS-zero}
\end{alignat}\end{subequations}
where $\gamma=2M_Nx/Q$ signals the twist-3 character of $F_{LT}^{\cos\phi_S}(x,z)$.
Notice that $F_{UT}^{\sin\phi_S}(x,z)$ has no DIS counterpart due to time reversal
symmetry of strong interactions, and terms suppressed by $1/Q^2$ are 
consequently neglected troughout this work including the twist-4 DIS 
structure function $F_L(x)$.


\subsection{TMDs, FFs and structure functions}
\label{Sec-2.2:def-TMD-FF}

TMDs are defined in terms of light-front correlators
\be\label{Eq:correlator}
    	\Phi(x,\bfkperp)_{ij} = \int\frac{ d \xi^- d^2{\bm \xi}_\perp}{(2\pi)^3}
	\;e^{ik\xi}\;\la N(P,S)|\bar\psi_j(0)\,{\cal W}_{(0,\,\infty)}
	{\cal W}_{(\infty,\xi)}\,\psi_i(\xi)|N(P,S)\ra
    	\with{ }{\xi^+\!=\!0}{$k^+ = xP^+$}
	\ee
where the Wilson-lines refer to the SIDIS process 
\cite{Collins:2002kn}. For a generic four-vector $a^\mu$ we define
the light-cone coordinates $a^\mu=(a^+,a^-,a_\perp)$ with 
$a^\pm=(a^0\pm a^3)/\sqrt{2}$. 
The light-cone direction is singled out by the virtual photon momentum 
and transverse vectors like $\bfkperp$ are perpendicular to it. In the
virtual-photon--nucleon center-of-mass frame, nucleon and the partons 
inside it move in the $(+)$--lightcone direction, while the struck 
quark and the produced hadron move in $(-)$--light-cone direction.
In the nucleon rest frame the polarization vector is given by 
$S=(0,{\bm S}_T,S_L)$ with ${\bm S}_T^2+S_L^2=1$.

The 8 leading--twist TMDs \cite{Boer:1997nt} are projected out from 
the correlator (\ref{Eq:correlator}) as follows (\blue{blue: T-even} TMDs, 
\red{red: T-odd} TMDs; all TMDs depend on $x$, $k_\perp$, renormalization 
scale and carry a flavor index which we do not indicate for brevity)
\begin{subequations}\ba
    \frac12\;{\rm Tr}\biggl[\gamma^+ \;\Phi(x,\bfkperp)\biggr]
    &=& \hspace{5mm}
    \blue{f_1}-\frac{\varepsilon^{jk}\kperp^j S_T^k}{M_N}\,\red{f_{1T}^\perp}\;, 
    \label{Eq:TMD-pdfs-I}\\
    \frac12\;{\rm Tr}\biggl[\gamma^+\gamma_5 \;\Phi(x,\bfkperp)\biggr] &=&
    S_L\,\blue{g_1} + \frac{\bfkperp \cdot{\bm S}_T}{M_N}\,\blue{g_{1T}^\perp}\;, 
    \label{Eq:TMD-pdfs-II}\\
    \frac12\;{\rm Tr}\biggl[i\sigma^{j+}\gamma_5 \;\Phi(x,\bfkperp)\biggr] &=&
    S_T^j\,\blue{h_1}  + S_L\,\frac{\kperp^j}{M_N}\,\blue{h_{1L}^\perp} +
    \frac{\kappa^{jk}S_T^k}{M_N^2}\,
    \blue{h_{1T}^\perp} + \frac{\varepsilon^{jk}\kperp^k}{M_N}\,\red{h_1^\perp}\;, 
    \label{Eq:TMD-pdfs-III} \hspace{15mm}
\ea
and the 16 subleading twist TMDs \cite{Mulders:1995dh,Bacchetta:2006tn}
are given by
\ba
\hspace{-5mm}    
	\frac12{\rm Tr}\biggl[\,1\;\Phi(x,\bfkperp)\biggr]         &=&
    	\frac{M_N}{P^+}\biggl[
	\hspace{5mm}\blue{e}
	-\frac{\varepsilon^{jk}\kperp^j S_T^k}{M_N}\,\red{e_T^\perp}
    	\biggr], \label{Eq:sub-TMD-pdfs-I}\\
\hspace{-5mm}    
	\frac12{\rm Tr}\biggl[i\gamma_5\Phi(x,\bfkperp)\biggr]        &=&
        \frac{M_N}{P^+}\biggl[
    	S_L\red{e_L} +\frac{\bfkperp \cdot {\bm S}_T}{M_N}\,\red{e_T}
    	\biggr], \label{Eq:sub-TMD-pdfs-II}\\
\hspace{-5mm}    
	\frac12{\rm Tr}\biggl[\,\gamma^j\,\Phi(x,\bfkperp)\biggr]        &=&
        \frac{M_N}{P^+}\biggl[
    	\frac{\kperp^j}{M_N}\blue{f^\perp}\!+\varepsilon^{jk}S_T^k\red{f_T}
	\!+\!S_L\frac{\varepsilon^{jk}\kperp^k}{M_N}\red{f_L^\perp}
	\!-\!\frac{\kappa^{jk}\varepsilon^{kl}S_T^l}{M_N^2}\red{f_T^\perp}\!
	\biggr], \label{Eq:sub-TMD-pdfs-III}\\
\hspace{-5mm}    
	\frac12{\rm Tr}\biggl[\,\gamma^j\gamma_5\Phi(x,\bfkperp)\biggr] &=&
    	\frac{M_N}{P^+}\biggl[
    	S_T^j\,\blue{g_T} 
	+ S_L\,\frac{\kperp^j}{M_N}\blue{g_L^\perp} +
	\frac{\kappa^{jk}S_T^k}{M_N^2}
    	\,\blue{g_T^\perp} 
	+\frac{\varepsilon^{jk}\kperp^k}{M_N}\,\red{g^\perp} 
	\biggr], \label{Eq:sub-TMD-pdfs-IV}\\
\hspace{-5mm}    
	\frac12{\rm Tr}\biggl[i\,\sigma^{jk}\gamma_5\Phi(x,\bfkperp)\biggr] &=&
    	\frac{M_N}{P^+}\biggl[
    	\frac{S_T^j \kperp^k-S_T^k \kperp^j}{M_N}\,\blue{h_T^\perp}
    	-\varepsilon^{jk}\,\red{h} 
	\biggr], \label{Eq:TMD-pdfs-V} \\
\hspace{-5mm}    
	\frac12{\rm Tr}\biggl[i\,\sigma^{+-}\,\gamma_5\,\Phi(x,\bfkperp)\biggr] 
	&=& \frac{M_N}{P^+}\biggl[
    	S_L\,\blue{h_L} + \frac{\bfkperp\cdot{\bm S}_T}{M_N}\,\blue{h_T}
    	\biggr]. \label{Eq:TMD-pdfs-VI}
\ea\end{subequations}
where $\kappa^{jk}\equiv (\kperp^j \kperp^k-\frac12\,\bfkperp^{\:2}\delta^{jk})$.
The indices $j,k,l$ refer to the plane transverse with respect to the
light-cone, $\epsilon^{ij}\equiv\epsilon^{-+ij}$ and $\epsilon^{0123}=+1$.
Dirac-structures not listed in (\ref{Eq:TMD-pdfs-I}--\ref{Eq:TMD-pdfs-VI}) 
are twist-4 \cite{Goeke:2005hb}.
Integrating out transverse momenta in the correlator (\ref{Eq:correlator})
leads to the ``usual'' PDFs known from collinear kinematics
\cite{Ralston:1979ys,Jaffe:1991ra}, namely at twist-2 level
\begin{subequations}\ba
    \frac12\;{\rm Tr}\biggl[\gamma^+ \;\Phi(x)\biggr]
    &=& \hspace{5mm}
    \blue{f_1}\;, 	\label{Eq:pdf-I}\\
    \frac12\;{\rm Tr}\biggl[\gamma^+\gamma_5 \;\Phi(x)\biggr] &=&
    S_L\,\blue{g_1}\;, 	\label{Eq:pdf-II}\\
    \frac12\;{\rm Tr}\biggl[i\sigma^{j+}\gamma_5 \;\Phi(x)\biggr] &=&
    S_T^j\,\blue{h_1}\;, \label{Eq:pdf-III} \hspace{75mm}
\ea
and at twist-3 level
\ba
    \frac12\;{\rm Tr}\biggl[\,1\;\Phi(x)\biggr] &=&
    \frac{M_N}{P^+}\;\blue{e}\;,  \label{Eq:sub-pdf-I}\\
    \frac12\;{\rm Tr}\biggl[\gamma^j\gamma_5 \;\Phi(x)\biggr] &=&
    \frac{M_N}{P^+}\;S_T^j\,\blue{g_T} \;, \label{Eq:sub-pdf-II}\\ \hspace{6mm}
    \frac12\;{\rm Tr}\biggl[\,i\,\sigma^{+-}\gamma_5 \;\Phi(x)\biggr] 
    &=& \frac{M_N}{P^+}\;S_L\,\blue{h_L}\,. \label{Eq:sub-pdf-III}\hspace{75mm}
\ea\end{subequations}
Other structures drop out either due to explicit $\kperp$--dependence,
or due to the sum rules \cite{Bacchetta:2006tn}
\be\label{Eq:sum-rules-T-odd}
	\int d^2\bfkperp\;f_T^a(x,\kperp^2)=
	\int d^2\bfkperp\;e_L^a(x,\kperp^2)=
	\int d^2\bfkperp\;h^a(x,\kperp^2)=0
\ee
imposed by time reversal constraints.

The fragmentation functions are similarly defined in terms of the correlator
\be\label{Eq:correlator-FF}
    \Delta(z,\bfpperp)_{ij} 
    = \sum\limits_X\!\int\!
    \frac{ d \xi^+ d^2 {\bm \xi}_\perp}{2z(2\pi)^3}\,e^{ip\xi}
    \, \la 0  |{\cal W}_{(\infty,\xi)}\psi_i(\xi)\,|h,X\ra\,
    \la h,X|\bar{\psi}_j(0){\cal W}_{(0,\infty)}|0\ra
    \with{\xi^-\!=\!0}
	 {p^- \!=\! P_h^-/z}
	 {${\bm p}_\perp \!=\! -\bfpperp/z$.}
    \ee
In this work we will consider only unpolarized final state hadrons.
If the produced hadron moves fast in the $(-)$ light cone direction, 
the twist-2 FFs are projected out as 
\begin{subequations}\ba
	\frac{1}{2}{\rm Tr}\big[\gamma^-\Delta(z,\bfpperp)\big]
	&=& \blue{D_1}\, , \label{eq:DeltaTr-twist-2a}\\
	\frac{1}{2}{\rm Tr}\big[i\sigma^{j-}\gamma_5\Delta(z,\bfpperp)\big]
	&=& \epsilon^{jk}\,\frac{\pperp^k}{zm_h}\red{H_1^\perp}\;, 
	\label{eq:DeltaTr-twist-2b}
\ea
and at twist-3 level
\ba
    \frac12\;{\rm Tr}\biggl[\,1\;\Delta(z,\bfpperp)\biggr]         &=&
    \phantom{-}\frac{M_h}{P^-_h}\;\blue{E}\;,  \label{eq:DeltaTr-twist-3a}\\
    \frac12\;{\rm Tr}\biggl[\;\,\gamma^j\;\Delta(z,\bfpperp)\biggr]  &=&
    -\frac{\pperp^j}{zP_h^-}\;\blue{D^\perp}\;, \label{eq:DeltaTr-twist-3b}\\
    \frac12\;{\rm Tr}\biggl[\gamma^j\gamma_5 \,\Delta(z,\bfpperp)\biggr] &=&
    \varepsilon^{jk}\,\frac{\pperp^k}{zP_h^-}\,\red{G^\perp}\;,  
	\label{eq:DeltaTr-twist-3c}\\
    \frac12\;{\rm Tr}\biggl[i\,\sigma^{jk}\gamma_5\,\Delta(z,\bfpperp)
	\biggr] &=&
    -\varepsilon^{jk}\,\frac{M_h}{P_h^-}\;\red{H}\;.  \label{eq:DeltaTr-twist-3d}
\ea\end{subequations}
The FFs depend on $z$, $P_\perp$, renormalization scale, quark flavor and 
type of hadron which we do not indicate for brevity.
Integration over transverse hadron momenta leaves us with $D_1(z)$, $E(z)$, 
$H(z)$ while the other structures drop out due to their $\pperp$ dependence.

%\subsection{Structure functions}

The structure functions in 
Eqs.~(\ref{Eq:SIDIS-leading},~\ref{Eq:SIDIS-subleading}) are described 
in Bjorken limit at tree level in terms of convolutions of TMDs 
and FFs. We define the unit vector $\bfhp   = \bfPhperp/\Phperp$ 
and use the following convolution integrals 
(see Appendix \ref{ApendixB1} for details) 
\be
 \label{Eq:def-convolution-integral}
 {\cal C}\biggl[\omega\;f\;D\biggr]
	= x \sum_a e_a^2\int d^2\bfkperp^{ } d^2\bfpperp
 	\; \delta^{(2)}(z \bfkperp^{ }+ \bfpperp^{ }-\bfPhperp^{ })\;\omega
	%\left(\bfkperp,-\frac{\bfpperp}{z}\right)
  	\; f^a(x,\bfkperp^2)\ D^a(z,\bfpperp^2)\;,
\ee
where $\omega$ is a weight function which in general depends on 
$\bfkperp$ and $\bfpperp$.
The 8 leading--twist structure functions are 
\begin{subequations}
\label{Eqs:structure-functions-twist-2}
\ba
 F_{UU}	&=&{\cal C}\biggl[\;\omega^{\{0\}}\,f_1 D_1 \;\biggr] \label{FUU}\\
 F_{LL}	&=&{\cal C}\biggl[\;\omega^{\{0\}}\,g_1 D_1 \;\biggr] \label{FLL}\\
 F_{UT}^{\sin\left(\phi_h +\phi_S\right)} 
	&=& {\cal C}\biggl[\;\omega^{\{1\}}_{\rm A} \,h_{1} H_1^{\perp}\;\biggr]
	\label{Eq:FUTCol}\\
 F_{UT}^{\sin\left(\phi_h -\phi_S\right)} 
	&=& {\cal C}\biggl[-\,\omega^{\{1\}}_{\rm B} \,f_{1T}^{\perp } D_1\,\biggr]
	\label{Eq:FUTSiv}\\
 F_{LT}^{\cos(\phi_h -\phi_S)} 
	&=& {\cal C}\biggl[\,\omega^{\{1\}}_{\rm B} \,g_{1T}^\perp D_1\biggr] 
	\label{Eq:FLT-twist-2}\\
 F_{UU}^{\cos 2\phi_h} 	
	&=& {\cal C}\biggl[\;\omega^{\{2\}}_{\rm AB}\,h_{1}^{\perp }\,H_{1}^{\perp }\;
	\biggr] \label{F_UUcos2phi}\\
 F_{UL}^{\sin 2\phi_h} 	
	&=& {\cal C}\biggl[\;\omega^{\{2\}}_{\rm AB}\,h_{1L}^{\perp } H_{1}^{\perp }\; 
	\biggr] \label{F_UUsin2phi}\\
 F_{UT}^{\sin\left(3\phi_h -\phi_S\right)} 
	&=& {\cal C}\biggl[\;\omega^{\{3\}}_{\rm { }}\,h_{1T}^{\perp } H_1^{\perp }\; 
	\biggr] \, . \hspace{75mm} \label{Eq:FUTpretzel}
\ea\end{subequations}
At subleading--twist we have the structure functions
\begin{subequations}
\label{Eqs:structure-functions-twist-3}
\ba
	F_{UU}^{\cos\phi_h}  
	&=& 
	\frac{2M}{Q}\,{\cal C}\biggl[\phantom{-}
   	\omega^{\{1\}}_{\rm A} 
	\biggl( x h\,H_{1}^{\perp } 
   	+ r_h^{ }\,\,f_1 \frac{\tilde{D}^{\perp }}{z}\biggr)
	- \omega^{\{1\}}_{\rm B} \biggl( x  f^{\perp } D_1
   	+ r_h^{ }\,\,h_{1}^{\perp } \frac{\tilde{H}}{z}\biggr)\biggr]\;
	\label{Eq:FUUcosphi}\\
	F_{LU}^{\sin\phi_h}  
	&=& 
	\frac{2M}{Q}\,{\cal C}\biggl[ \phantom{-}
	\omega^{\{1\}}_{\rm A}
   	\biggl( x \, e \, H_1^{\perp } 
   	+ r_h^{ }\,\,f_1\frac{\tilde{G}^{\perp }}{z}\,\biggr)
   	+\omega^{\{1\}}_{\rm B}
   	\biggl( x   g^{\perp }  D_1 
   	+ r_h^{ }\,\, h_1^{\perp } \,\frac{\tilde{E}}{z} \biggr)\biggr]\;
	\label{FLUsinphi}\;\;\;\;\;\\
	F_{UL}^{\sin\phi_h} 
 	&=& 
	\frac{2M}{Q}\,{\cal C}\biggl[\phantom{-}
   	\omega^{\{1\}}_{\rm A}
    	\biggl( x   h_L  H_1^{\perp } \! 
   	+ r_h^{ }\,\,g_{1}\frac{\tilde{G}^{\perp } }{z}\biggr)
   	+\omega^{\{1\}}_{\rm B}
    	\biggl( x  f_{L}^{\perp }  D_1 \!
   	- r_h^{ }\,\, h_{1L}^{\perp }  \frac{\tilde{H}}{z}\biggr)\biggr]\;
	\label{FULsinphi}\\
	F_{LL}^{\cos \phi_h} 
 	&=& 
	\frac{2M}{Q}\,{\cal C}\biggl[ 
	-\omega^{\{1\}}_{\rm A}
   	\biggl( x  e_L  H_1^{\perp }
   	- r_h^{ }\,\,g_{1}   \frac{\tilde{D}^{\perp }}{z}\biggr)
   	-\omega^{\{1\}}_{\rm B}
   	\biggl( x   g_L^{\perp }   D_1
   	+  r_h^{ }\,\,h_{1L}^{\perp } \frac{\tilde{E}}{z}\biggr)\biggr]
	\label{FLLcosphi}\;\;\;\;\;\;\;\;\;\\
	F_{UT}^{\sin \phi_S } 
	&=&  
	\frac{2M}{Q}\,{\cal C}\biggl[ \phantom{-}
	\omega^{\{0\}}\,\biggl(x   f_T   D_1 \;
   	- r_h^{ }\, \; h_{1} \, \frac{\tilde{H}}{z} \biggr)\nonumber\\
   	&&\hspace{1.2cm}
   	-\frac{\omega^{\{2\}}_{\rm B}}{2}
	\biggl( x   h_{T}  H_{1}^{\perp } 
   	+ r_h^{ }\, g_{1T}^\perp \,\frac{\tilde{G}^{\perp }}{z}
   	- x   h_{T}^{\perp }  H_{1}^{\perp } 
	+ r_h^{ }\, f_{1T}^{\perp } \,\frac{\tilde{D}^{\perp }}{z}
   	\biggr) \biggr]\; \label{FUTsinphiS}\\ 
	F_{LT}^{\cos \phi_S} 
	&=& 
	\frac{2M}{Q}\,{\cal C}\biggl[
   	- \omega^{\{0\}}\,\biggl(x   g_T   D_1
   	+ r_h^{ }\, \, h_{1}  \frac{\tilde{E}}{z} \biggr)\nonumber\\
   	&&\hspace{1.2cm}
	+\frac{\omega^{\{2\}}_{\rm B}}{2}
   	\biggl( x   e_{T}  H_{1}^{\perp } 
   	- r_h^{ }\, g_{1T}^\perp \,\frac{\tilde{D}^{\perp }}{z}
   	+  x   e_{T}^{\perp }  H_{1}^{\perp } 
   	+ r_h^{ }\,f_{1T}^{\perp }\,\frac{\tilde{G}^{\perp }}{z}\biggr)\biggr]
	\; \label{FLTcosphiS} \\ \hspace{-5mm}
	F_{UT}^{\sin(2\phi_h -\phi_S)} 
	&=& 
	\frac{2M}{Q}\,{\cal C}\biggl[\phantom{-}
   	\frac{\omega^{\{2\}}_{\rm AB}}{2}\,
   	\biggl( x   h_{T}  H_{1}^{\perp } 
   	+ r_h^{ }\, g_{1T}^\perp \,\frac{\tilde{G}^{\perp }}{z}
        + x   h_{T}^{\perp }  H_{1}^{\perp } 
   	- r_h^{ }\, f_{1T}^{\perp } \,\frac{\tilde{D}^{\perp }}{z}
	\biggr)\nonumber\\
	&&\hspace{1.3cm}
	+
   	\omega^{\{2\}}_{\rm C}
   	\biggl( x   f_T^{\perp } D_1
   	- r_h^{ }\, \, h_{1T}^{\perp }  \frac{\tilde{H}}{z}\biggr) \biggr]\\
	F_{LT}^{\cos(2\phi_h - \phi_S)} 
	&=& \frac{2M}{Q}\,{\cal C}\biggl[
   	- \frac{\omega^{\{2\}}_{\rm AB}}{2}
   	\biggl( x   e_{T}  H_{1}^{\perp } 
   	- r_h^{ }\, g_{1T}^\perp \,\frac{\tilde{D}^{\perp }}{z}
	- x   e_{T}^{\perp }  H_{1}^{\perp } 
   	- r_h^{ }\, f_{1T}^{\perp } \,\frac{\tilde{G}^{\perp }}{z}\biggr)\nonumber\\
	&&\hspace{1.3cm}
   	- \omega^{\{2\}}_{\rm C}
   	\biggl( x   g_T^{\perp }   D_1
   	+ r_h^{ }\, \, h_{1T}^{\perp }  \frac{\tilde{E}}{z}\biggr)\biggr]\; 
	\label{FLTsin(2phi-phiS)}
\ea\end{subequations}
where $r_h = m_h/M_N$ and 
$F_{XY}^{\rm weight}\equiv F_{XY}^{\rm weight}(x,z,\Phperp)$. The 
tilde-functions $\tilde{D}^{\perp },\,\tilde{G}^{\perp },\,\tilde{H},\,\tilde{E}$
are defined in terms of $\bar{q}gq$-correlators, see 
Sec.~\ref{Sec-3.2:WW-type-TMD-FF}. The weight functions are defined as
\ba
&& \omega^{\{0\}}  	= 1 \, , \nonumber\\
&& \omega^{\{1\}}_{\rm A} 	= \frac{\bfhp\cdot\bfpperp^{ }}{z m_h}  \, , \;\;\;
   \omega^{\{1\}}_{\rm B} 	= \frac{\bfhp\cdot\bfkperp^{ }}{M_N}\,,\nonumber\\
&& \omega^{\{2\}}_{\rm A} 	=  \frac{2\, \bigl(\bfhp\cdot\bfpperp^{ }\bigr)\,
			\bigl(\bfhp\cdot\bfkperp^{ }\bigr)}{zM_Nm_h}\,,\;\;\;
   \omega^{\{2\}}_{\rm B}	= -\frac{\bfpperp^{ }\cdot\bfkperp^{ }}{zM_Nm_h}\,,\;\;\;
   \omega^{\{2\}}_{\rm C}	= \frac{2\,(\bfhp\cdot\bfkperp^{ })^2-\bfkperp^2}{2M_N^2}
			\, ,\nonumber\\
&& \omega^{\{3\}}_{\rm{ }}	= \frac{
			 4\,(\bfhp\cdot\bfpperp^{ })\,(\bfhp\cdot\bfkperp^{ })^2
			-2\,\bigl(\bfhp\cdot\bfkperp^{ }\bigr)\, 
        		\bigl(\bfkperp^{ }\cdot\bfpperp^{ }\bigr)
   			-\bigl(\bfhp\cdot\bfpperp^{ }\bigr)\,\bfkperp^2\
   			}{2 z M_N^2 m_h}\,,
			\label{Eq:wi} \ea
and $\omega^{\{2\}}_{\rm AB} = \omega^{\{2\}}_{\rm A} + \omega^{\{2\}}_{\rm B}$.
In $\omega^{\{n\}}_{i}$ the index $n=0,\,1,\,2,\,3$ indicates the (maximal)
power $(\Phperp)^n$ with which the corresponding contribution scales,
and index $i$ (if any) distinguishes different types of contributions
at the given order $N$. 
Notice that twist-3 structure functions in 
Eqs.~(\ref{Eq:FUUcosphi}--\ref{FLTsin(2phi-phiS)}) contain an explicit 
factor $M/Q$. We also recall that we neglect two structure functions
(denoted in  \cite{Bacchetta:2006tn} 
as $F_{UU,L}$ and $F_{UT,L}^{\sin( \phi_h-\phi_S)}$) 
due to longitudinal virtual photon polarization which are 
of order ${\cal O}(M^2/Q^2)$ in the TMD partonic description.

The structures surviving $\Phperp$--integration of the SIDIS cross section in 
(\ref{Eq:SIDIS-leading-integrated},~\ref{Eq:SIDIS-subleading-integrated})
are associated with the trivial weights $\omega^{\{0\}}$ and expressed in 
terms of collinear PDFs and FFs as follows (here the 
sum rules (\ref{Eq:sum-rules-T-odd}) are used)
\begin{subequations}\ba
	F_{UU}(x,z) &=& x\sum\limits_ae_a^2\,f_1^a(x)\,D_1^a(z)
	\label{Eq:FUU-collinear}\\
	F_{LL}(x,z) &=& x\sum\limits_ae_a^2\,g_1^a(x)\,D_1^a(z)
	\label{Eq:FLL-collinear}\\
 	F_{LT}^{\cos\phi_S}(x,z) &=& -\,\frac{2M_N}{Q}\; x\sum_a e_a^2\,
	\biggl(x\,g_T^q(x)\,D_1^a(z)+r_h\,h_1^a(x)\frac{\tilde{E}^a(z)}{z}\biggr)
	\label{Eq:FLT-collinear}\\
	 F_{UT}^{\sin\phi_S}(x,z) &=& -\,\frac{2\,m_h}{Q}\; x\sum_a e_a^2\,
	h_1^a(x)\frac{\tilde{H}^a(z)}{z}
	\label{Eq:FUT-collinear}
\ea\end{subequations}
Finally, integrating over $z$, summing over hadrons, and 
using the sum rules for the T-odd FFs 
$\sum_h\int d z\;\tilde{E}^a(z)=0$ and
$\sum_h\int d z\;\tilde{H}^a(z)=0$ we recover
Eqs.~(\ref{Eq:DIS-F1}--\ref{Eq:DIS-zero}) and obtain for the DIS structure 
functions
\begin{subequations}\ba
    F_1(x) & = & \frac{1}{2}\sum_a e_a^2\,f_1^a(x) \label{Eq:DIS-F1-II}\\
    g_1(x) & = & \frac{1}{2}\sum_a e_a^2\,g_1^a(x) \label{Eq:DIS-g1-II}\\
    g_2(x) & = & \frac{1}{2}\sum_a e_a^2\,g_T^a(x)\;-\;g_1(x) \label{Eq:DIS-g2}
	\hspace{25mm}
\ea\end{subequations}

Before introducing the WW-type approximations in the next section, 
we would like to add a comment on TMD factorization: the 
partonic description of the leading-twist structure functions in 
(\ref{Eqs:structure-functions-twist-2}) is based on factorization 
theorems \cite{Collins:1981uk,Ji:2004wu,Ji:2004xq,Collins:2011zzd,
Echevarria:2012js}. In contrast to this, the partonic description 
of the subleading-twist structure functions in 
(\ref{Eqs:structure-functions-twist-3}) is based on the
{\it assumption} that that SIDIS cross section factorizes.

In fact, a lot of progress has been achieved in recent years in the 
theoretical understanding of leading-twist observables within the TMD 
framework --- including definition, renormalization and evolution of 
leading-twist TMDs \cite{Aybat:2011zv,Aybat:2011ge,Echevarria:2014xaa},
NLO corrections within the TMD framework \cite{Ma:2013aca}, and  
phenomenological fits with evolution \cite{Aybat:2011ta,Kang:2015msa}.
In contrast to this, the theory for subleading-twist TMD observables is 
only pourly developed. Still to the present day, the state-of-the-art 
approach to subleading twist TMD observables is the one of 
Refs.~\cite{Kotzinian:1994dv,Mulders:1995dh,Boer:1997nt,Goeke:2005hb,
Bacchetta:2006tn}, 
based on a TMD tree-level formalism which we adopt here.
In fact, the results of Refs.~\cite{Metz:2004je,Gamberg:2006ru}
indicate doubts even in the tree-level formalism. 
Recently, an attempt was made to remedy these doubts \cite{Chen:2016hgw}.
Keeping in mind these ``words of warning,'' still the formulas 
(\ref{Eqs:structure-functions-twist-3}) 
are the best that theory has to offer currently. We may consider 
(\ref{Eqs:structure-functions-twist-3}) as a model itself for 
the twist-3 SIDIS observables. We hope that the phenomenological 
approach based on WW-type approximations pursued in this work might 
lead to more insight into these observables, and eventually might 
trigger more theory efforts in the future.

%\newpage
%======= SECTION 3: WW APPROXIMATIONS ================================
\section{WW and WW-type approximations}
\label{Sec-3:WW}

In this section we will define the approximations and review what is
known about them.
The basic idea of the approximations is simple. One uses QCD
equations of motion to separate contributions from $\bar{q}q$--terms 
and $\bar{q}gq$--terms and assumes that the latter can be neglected 
with respect to the leading $\bar{q}q$--terms with a useful accuracy
(here the $\la\dots\ra$ denote symbolically the matrix elements 
which enter the definitions of TMDs or FFs)
\be\label{Eq:WW-generic}
	\biggl|\frac{\la\bar{q}gq\ra}{\la\bar{q}q\ra}\biggr| \ll 1\,.
\ee

\subsection{WW approximation for PDF\lowercase{s}}
\label{Sec-3.1:WW-classic}

The WW approximation applies in principle to all twist-3 PDFs,
Eqs.~(\ref{Eq:sub-pdf-I},~\ref{Eq:sub-pdf-II},~\ref{Eq:sub-pdf-III}).
It was established first for $g_T^a(x)$ \cite{Wandzura:1977qf}, and 
later for $h_L^a(x)$ \cite{Jaffe:1991ra}. The situation of $e^a(x)$ 
is somewhat special, see below and the review \cite{Efremov:2002qh}.

The origin of the approximations is as follows.
The operators defining $g_T^a(x)$ and~$h_L^a(x)$ can be decomposed by means 
of QCD equations of motion in twist-2 parts, and pure twist-3 
(interaction dependent) $\bar{q}gq$--terms and current quark mass
terms. We denote $\bar{q}gq$--terms and mass-terms collectively 
and symbolically by functions with a tilde.
Such decompositions are possible because $g_T^a(x)$ and $h_L^a(x)$ are 
``twist-3'' not according to the ``strict QCD definition''
(twist $=$ mass dimension of associated local operator minus its spin).
Rather they are classified according to the ``working definition'' 
of twist \cite{Jaffe:1996zw}
(a function is ``twist $t$'' if, in addition to overall kinematic
prefactors, it contributes to cross sections in a partonic 
description suppressed by $(M/Q)^{t-2}$ where $M$ is a generic 
hadronic and $Q$ the hard scale).
The two definitions coincide for twist-2 quantities, but higher twist
observables in general contain ``contaminations'' by leading twist.

In this way one obtains the decompositions and, if they apply, WW 
approximations \cite{Wandzura:1977qf,Jaffe:1991ra} (keep in mind 
here tilde terms contain pure twist-3 and current quark mass terms)
\begin{subequations}\begin{alignat}{3}
   	g_T^a(x) &=& 
        \phantom{2x}\;\int_x^1\frac{ d y}{y}\,g_1^a(y) + &\tilde{g}_T^a(x)
        \stackrel{\rm WW}{\approx} 
        \phantom{2x}\;\int_x^1\frac{ d y}{y}\,g_1^a(y) \;, 
	\label{Eq:WW-original1} \\
   	h_L^a(x) &=& 2x\int_x^1\frac{ d y}{y^2}\,h_1^a(y) + &\tilde{h}_L^a(x)
        \stackrel{\rm WW}{\approx} 2x\int_x^1\frac{ d y}{y^2}\,h_1^a(y)\;,
	\label{Eq:WW-original2}\\
   	x\,e^a(x) &=& x\,&\tilde{e}^a(x) \;\stackrel{\rm WW}{\approx} 
	\;\;\;\; 0 \label{Eq:WW-original-e}
\end{alignat}\end{subequations}
where we included $e^a(x)$ which is a special case in the sense that it 
receives no twist-2 contribution. 
A prefactor of $x$ is provided in (\ref{Eq:WW-original-e})
to cancel a $\delta(x)$--type singularity \cite{Efremov:2002qh}.

The relations (\ref{Eq:WW-original1}--\ref{Eq:WW-original-e})
have been derived basically using operator product expansion
techniques \cite{Wandzura:1977qf,Jaffe:1991ra}. Notice that the 
operators defining $g_T^a$ and and $h_L^a$ can also be decomposed 
within the TMD framework by means of a combination of relations 
derived from the QCD equations of motion and further 
constraint relations, called Lorentz-Invariance Relations (LIRs), 
(see recent review \cite{Kanazawa:2015ajw} and references therein) 
into a twist-2 part, and dynamical twist-3 (interactions dependent) 
$\bar{q}gq$-terms and current quark mass terms.

We will come back to (\ref{Eq:WW-original1},~\ref{Eq:WW-original2}) and 
review the theoretical predictions and supporting experiments, but before
we will introduce the WW-type approximations for TMDs and FFs.

\subsection{WW-type approximations for TMDs and FFs}
\label{Sec-3.2:WW-type-TMD-FF}

Analogous to WW approximations for PDFs discussed in 
Sec.~\ref{Sec-3.1:WW-classic}, also certain TMDs and FFs
can be decomposed in twist-2 contributions and tilde-terms.
The latter may be assumed, in the spirit of (\ref{Eq:WW-generic}),
to be small. Hereby it is important to keep in mind that for each TMD
or FF one deals 
with different types of (``unintegrated'') $\bar{q}gq$--correlations,
and we prefer to refer to them as WW-type approximations.

In the T-even case one obtains the following approximations,
where the terms on the left-hand-side are twist-3, those on the 
right-hand-side (if any) are twist-2,
\begin{subequations}\ba
xe^q(x,\kperp^2)	&\stackrel{\rm WW-type}{\approx}& 
			0, 
			\label{Eq:WW-type-1}\\
xf^{\perp q}(x,\kperp^2)  &\stackrel{\rm WW-type}{\approx}& 
			f_{1}^q(x,\kperp^2),
			\label{Eq:WW-type-Cahn}\phantom{\frac11}\\
xg_L^{\perp q}(x,\kperp^2)&\stackrel{\rm WW-type}{\approx}& 
			g_{1}^q(x,\kperp^2),
			\label{Eq:WW-type-gLperp}\phantom{\frac11}\\
xg_T^{\perp q}(x,\kperp^2)&\stackrel{\rm WW-type}{\approx}& 
			g_{1T}^{\perp q}(x,\kperp^2),\phantom{\frac11}
			\label{Eq:WW-type-gTperp}\\
xg_T^q(x,\kperp^2)   	&\stackrel{\rm WW-type}{\approx}& 
             		g_{1T}^{\perp(1)q}(x,\kperp^2), 
			\label{Eq:WW-type-gT}\\
xh_L^q(x,\kperp^2)	&\stackrel{\rm WW-type}{\approx}& -2 \,
                       	h_{1L}^{\perp(1)q}(x,\kperp^2),\phantom{\frac11}
                       	\label{Eq:WW-type-6}\\
xh_T^q(x,\kperp^2)      &\stackrel{\rm WW-type}{\approx}& 
                       	- h_1^q(x,\kperp^2) - h_{1T}^{\perp(1)}(x,\kperp^2),
                       	\label{Eq:WW-type-7}\\
xh_T^{\perp q}(x,\kperp^2)&\stackrel{\rm WW-type}{\approx}& 
                       	\phantom{-}h_1^q(x,\kperp^2)-h_{1T}^{\perp(1)}(x,\kperp^2).
                       	\phantom{\frac11} \label{Eq:WW-type-8}
\ea\end{subequations}
In the T-odd case one obtains the approximations
\begin{subequations}\ba
xe_L^q(x,\kperp^2)         	&\stackrel{\rm WW-type}{\approx}& 0, 
			\label{Eq:WW-type-eLperp}\\
xe_T^q(x,\kperp^2)         	&\stackrel{\rm WW-type}{\approx}& 0, \\
xe_T^{\perp q}(x,\kperp^2) 	&\stackrel{\rm WW-type}{\approx}& 0, \\
xg^{\perp q}(x,\kperp^2)   	&\stackrel{\rm WW-type}{\approx}& 0, 
                       	\label{Eq:WW-type-gperp}\\
xf_L^{\perp q}(x,\kperp^2) 	&\stackrel{\rm WW-type}{\approx}& 0, 
			\label{Eq:WW-type-fLperp}\phantom{\frac11}\\
xf_T^{\perp q}(x,\kperp^2) 	&\stackrel{\rm WW-type}{\approx}& 
                       	\phantom{-}\,f_{1T}^{\perp q}(x,\kperp^2),
			\phantom{\frac11}
                       	\label{Eq:WW-type-fTperp}\\
xf_T^{q}(x,\kperp^2)       	&\stackrel{\rm WW-type}{\approx}& 
                       	-\,f_{1T}^{\perp(1)q}(x,\kperp^2), \label{Eq:WW-type-fT}\\
xh^q(x,\kperp^2)           	&\stackrel{\rm WW-type}{\approx}& 
                       	- 2\,{h_1^{\perp(1)}(x,\kperp^2)}.\phantom{\frac11 XXXXx} 
                       	\label{Eq:WW-type-last} 
\ea\end{subequations}
The superscript ``(1)'' denotes the first transverse 
moments of TMDs defined generically as  
\ba
f^{(1)}(x,\kperp^2) = \frac{\kperp^2}{2M^2}\,f(x, \kperp^2)\; , \;\;\;
f^{(1)}(x ) = \int d^2 \bfkperp f^{(1)}(x,\kperp^2) \; . 
\ea 

Two very useful WW-type approximations follow from combining the
WW approximations (\ref{Eq:WW-original1},~\ref{Eq:WW-original2}) with the
WW-type approximations (\ref{Eq:WW-type-gT},~\ref{Eq:WW-type-6}).
This yields \cite{Tangerman:1994bb,Mulders:1995dh,Avakian:2007mv}%
\begin{subequations}\ba
   	g_{1T}^{\perp(1)a}(x)&\stackrel{\rm WW-type}{\approx}& 
        \phantom{-} x\,\int_x^1\frac{d y}{y\;}\,g_1^a(y) \;,
	\label{Eq:WW-approx-g1T}\\
    	h_{1L}^{\perp(1)a}(x)&\stackrel{\rm WW-type}{\approx}& -
	x^2\!\int_x^1\frac{d y}{y^2\;}\,h_1^a(y)\;.
	\label{Eq:WW-approx-h1L}
\ea\end{subequations}
Some of the above WW-type approximations were discussed in 
\cite{Tangerman:1994bb,Kotzinian:1995cz,Mulders:1995dh,Kotzinian:1997wt,
Kotzinian:2006dw,Avakian:2007mv,Metz:2008ib,Teckentrup:2009tk}.
WW-relations for FFs are actually not needed: in Eqs.~%
(\ref{Eqs:structure-functions-twist-2},~\ref{Eqs:structure-functions-twist-3})
either twist-2 FFs $D_1^q$, $H_1^{\perp q}$ enter or tilde FFs, as a consequence 
of how the azimuthal angles are defined \cite{Bacchetta:2006tn}. 
For completeness we quote the WW-type approximations for FFs
\cite{Bacchetta:2006tn}
\begin{subequations}\ba
	E(z,\pperp^2)      &\stackrel{\rm WW-type}{\approx}& 0 \, ,
	\label{Eq:WW-type-FF-1}\\
	G^\perp(z,\pperp^2) &\stackrel{\rm WW-type}{\approx}& 0 \, ,
	\label{Eq:WW-type-FF-2}\\
	D^\perp(z,\pperp^2) &\stackrel{\rm WW-type}{\approx}
	& \hspace{11mm}z\,D_1(z,\pperp^2)\,, \label{Eq:WW-type-FF-3}\\
	H(z,\pperp^2) &\stackrel{\rm WW-type}{\approx}& 
	-\frac{\pperp^2}{zM_h^2}\;H_1^\perp(z,\pperp^2)\,. \label{Eq:WW-type-FF-4}
\ea\end{subequations}
Having introduced the WW- and WW-type approximations, we will review
in the following what is currently known from theory and experiment 
about the WW(-type) approximations.

\subsection{Predictions from instanton vacuum model}
\label{Sec-3.3:WW-classic-instanton}

Insights on the relative size of hadronic matrix elements, such as 
Eq.~(\ref{Eq:WW-generic}), require a non-perturbative approach. It is 
by no means obvious which small parameter in the strong interaction 
regime would allow one to explain such results. An appeling 
non-perturbative approach is provided by the instanton model of 
the QCD vacuum \cite{Shuryak:1981ff,Diakonov:1983hh,Diakonov:1995qy}.
This semi-classical approach assumes that properties of the QCD vacuum 
are dominated by instantons and anti-instantons, topological non-perturbative 
gluon field configurations, which form a strongly interacting medium.
The approach provides a natural mechanism for dynamical chiral symmetry 
breaking, the dominant feature of strong interactions in the nonperturbative
regime. It was shown with variational and numerical methods that the 
instantons form a dilute medium characterized by a non-trivial small paramater
$\rho/R\sim1/3$ \cite{Shuryak:1981ff,Diakonov:1983hh,Diakonov:1995qy}
where $\rho$ and $R$ denote respectively the average instanton size $\rho$
and separation $R$.

Applying the instanton vacuum model to studies of $g_T^a(x)$ and $h_L^a(x)$
it was predicted that matrix elements of the $\bar{q}gq$ operators defining 
$\tilde{g}_T^a(x)$ \cite{Balla:1997hf} and $\tilde{h}_L^a(x)$ 
\cite{Dressler:1999hc} are strongly suppressed by powers of the small
parameter $\rho/R$ with respect to contributions from the 
respective twist-2 parts which are of order $(\rho/R)^0$. 
For Mellin moments it was found \cite{Balla:1997hf,Dressler:1999hc}
\be\label{Eq:WW-instanoton}
	\frac{\tilde{g}_T^q}{g_T^q} \sim \frac{\tilde{h}_L^q}{h_L^q} 
	\sim \frac{\la\bar{q}gq\ra}{\la\bar{q}q\ra} \sim 
	\biggl(\frac{\rho}{R}\biggr)^{\!4} \log\biggl(\frac{\rho}{R}\biggr)
	\sim 10^{-2}
\ee
which strongly supports the generic approximation
in Eq.~(\ref{Eq:WW-generic}) with the instanton packing fraction providing 
the non-trivial small parameter justifying the neglect of tilde-terms. 
The predictions for $\tilde{g}_T^a(x)$ \cite{Balla:1997hf}
were made before the advent of the first precise data on $g_2(x)$
which we discuss next.
The instanton calculus has not yet been applied to $\tilde{e}^a(x)$.


%\newpage
\subsection{Tests of WW approximation in DIS experiments}
\label{Sec-3.4:WW-classic-experiment}

The presently available phenomenological information on $g_T^a(x)$ is due 
to measurements of the structure function $g_2(x)$, Eq.~(\ref{Eq:DIS-g2}),
in DIS off various transversely polarized targets. In the WW-approximation 
(\ref{Eq:WW-original1}) one can write $g_2(x)$ as a total derivative
expressed in terms of the experimentally well known twist-2
structure function $g_1(x)$ as follows
\be
    	g_2(x) \stackrel{\rm WW}{\approx} g_2(x)_{\rm WW} \equiv
	\frac{ d\;}{ d x}\;\Biggl[x\int_x^1\frac{ d y}{y}
	\;g_1(y)\Biggr]\,.\label{Eq:g2-in-WW-approximation}
\ee
Data support (\ref{Eq:g2-in-WW-approximation}) to a good accuracy
\cite{Anthony:2002hy,Abe:1998wq,Airapetian:2011wu}, although especially 
at smaller $x$ more stringent tests are not yet possible. Overall it has 
been estimated that the WW approximation for $g_2(x)$ and $g_T^a(x)$ works 
with an accuracy of about $40\,\%$ or better \cite{Accardi:2009au}. 

%%%%%%%%%%%%%%%%%%%%%%%%%%%%%%%%%%%%%%%%%%%%%%%%%%%%%%%%%%%%%%%%%%%%%%%%%%%%%%%
\begin{figure}[t!]
\centering
\includegraphics[width=0.48\textwidth]{\FigPath/g2.pdf} 
\includegraphics[width=0.48\textwidth]{\FigPath/g2-HERMES.pdf} 
\caption{\label{Fig:g2} 
The structure function $xg_2(x)$ in WW-approximation at $Q^2=7.1\,{\rm GeV}^2$,
Eq.~(\ref{Eq:g2-in-WW-approximation}), for proton (P) and neutron (N) targets. 
Left panel: data from E144 and E155 experiments at 
$\la Q^2\ra=7.1\,{\rm GeV}^2$ \cite{Anthony:2002hy,Abe:1998wq}. 
Right panel: HERMES data for $Q^2>1\,{\rm GeV}^2$ with
$\la Q^2\ra=2.4\,{\rm GeV}^2$ \cite{Airapetian:2011wu}.
The estimate of the theoretical uncertainty is described in the text.}
\end{figure}
%%%%%%%%%%%%%%%%%%%%%%%%%%%%%%%%%%%%%%%%%%%%%%%%%%%%%%%%%%%%%%%%%%%%%%%%%%%%%%%

We present calculations of $g_2(x)_{\rm WW}$ in Fig.~\ref{Fig:g2}.
This result is obtained using the LO $g_1^a(x)$-parametrization 
\cite{Gluck:1998xa}. In order to display the theoretical ``uncertainty 
band'' of this WW-approximation of about $40\,\%$ as deduced in 
Ref.~\cite{Accardi:2009au} we proceed as follows: we split the 
$40\,\%$ uncertainty in two parts, $\varepsilon_1=\pm 20\,\%$ and 
$\varepsilon_2(x)=\pm 20\,\%(1-x)^\epsilon$ with a small $\epsilon>0$,
and estimate the impact of this uncertainty as 
\be\label{Eq:g2-in-WW-model-violation}
    g_2(x)_{\rm WW} = (1\pm\varepsilon_1)\frac{ d\;}{ d x}\;\Biggl[
    x\int_x^1\frac{ d y}{y} \,\biggl(
    \frac12\sum_ae_a^2\,g_1^a(y(1\pm\varepsilon_2))\biggr)\Biggr]\,.
\ee
The effect of $\varepsilon_1$ is to change the magnitude
of $g_2(x)_{\rm WW}$, $\varepsilon_2$ varies the position of its zero.
The $x$--dependence of $\varepsilon_2$ preserves $\lim_{x\to1}g_2(x)= 0$; 
we use $\epsilon=0.05$ which yields $\varepsilon_2\approx 20\,\%$ up to 
the highest measured $x$--bin.
The good agreement of $g_2(x)_{\rm WW}$ with data is encouraging,
and in line with theory predictions \cite{Balla:1997hf}.
Our estimate with the splitted uncertainties 
$\varepsilon_{1,2}$ may overestimate in certain $x$--bins the 
$40\,\%$--``uncertainty band'' estimated in \cite{Accardi:2009au}. 
This however helps us to display a conservative estimate of possible 
uncertainties. 
We conclude that the WW--approximation works reasonably well, 
see Fig.~\ref{Fig:g2}.

Presently $h_L^a(x)$ is unknown.
With phenomenological information on $h_1^a(x)$
\cite{Efremov:2006qm,Anselmino:2007fs,Anselmino:2008jk}, 
the WW approximation (\ref{Eq:WW-original2}) for $h_L^a(x)$ could 
be tested experimentally in Drell-Yan \cite{Koike:2008du}.


\subsection{Tests in lattice QCD}
\label{Sec-3.5:WW-lattice}

The lowest Mellin moments of the PDF $g_T^q(x)$ were studied in
lattice QCD in the quenched approximation \cite{Gockeler:2000ja} 
and with $N_f = 2$ flavors of light dynamical quarks \cite{Gockeler:2005vw}.
The obtained results were compatible with a small $\tilde{g}_T^q(x)$. 
We are not aware of lattice QCD studies related to the PDF $h_L^a(x)$,
and turn now our attention to TMD studies in lattice QCD.

After first exploratory investigations of TMDs on the lattice
\cite{Hagler:2009mb,Musch:2010ka}, recent years have witnessed considerable
progress and improvements with regard to rigor, realism and methodology
\cite{Yoon:2017qzo, %PhysRevD.96.094508
Engelhardt:2015xja,%Phys.Rev. D93 (2016) no.5, 054501, 
Ji:2014hxa,%Phys.Rev. D91 (2015) 074009, 
Musch:2011er%Phys.Rev. D85 (2012) 094510
}.
However, numerical results from recent calculations are only available 
for a subset of observables, and the quantities calculated are not in a 
form that lends itself to straighforward tests of the WW-type relations 
as presented in this paper. Details about recent works and future 
perspectives are discussed at the end of this section.

For the time being, we content ourselves with rather crude comparisons 
based on the lattice data
published in Refs.~\cite{Hagler:2009mb,Musch:2010ka}. 
These early works explored all nucleon and quark polarizations, but 
they used a gauge link that does not incorporate the final or initial 
state interactions present in SIDIS or Drell-Yan experiments. In other 
words, the transverse momentum dependent quantities computed in
\cite{Hagler:2009mb,Musch:2010ka} are not precisely the TMDs measurable 
in experiment. More caveats will be discussed along the way.

Let us now translate the approximations
(\ref{Eq:WW-approx-g1T},~\ref{Eq:WW-approx-h1L}) into expressions
for which we have a chance to compare them with available lattice data.
For that we multiply the
Eqs.~(\ref{Eq:WW-approx-g1T},~\ref{Eq:WW-approx-h1L}) by $x^N$
with $N=0,\,1,\,2,\,\dots$ and integrate over $x\in[-1,1]$ which yields
\ba
        \int_{-1}^1 d x\;x^N
       g_{1T}^{\perp(1)q}(x)&\stackrel{\rm WW-type}{\approx}&
        \phantom{-} \,\frac{1}{N+2} \int_{-1}^1 d x \;x^{N+1}g_1^q(x)
        \;,
    \label{Eq:WW-approx-g1T-d}\\
        \int_{-1}^1 d x\;x^N
        h_{1L}^{\perp(1)q}(x)&\stackrel{\rm WW-type}{\approx}&
        -\,\frac{1}{N+3} \int_{-1}^1 d x\: x^{N+1}\,h_1^q(x)
        \;.
    \label{Eq:WW-approx-h1L-d}
\ea
with the understanding that
negative $x$ refer to antiquark distributions
$g_1^{\bar q}(x) = +\,g_1^{q}(-x)$,
$h_1^{\bar q}(x) = -\,h_1^{q}(-x)$,
$g_{1T}^{\perp(1)\bar q}(x) =- g_{1T}^{\perp(1)q}(-x)$,
$h_{1L}^{\perp(1)\bar q}(x) = +\,h_{1L}^{\perp(1)q}(-x)$
depending on $C$--parity of the involved operators \cite{Mulders:1995dh}.
The right hand sides of 
Eqs.~(\ref{Eq:WW-approx-g1T-d},~\ref{Eq:WW-approx-h1L-d}) are $x$--moments 
of parton distributions, and those can be obtained from lattice QCD using 
well-established methods based on operator product expansion. 
The left hand sides are moments of TMDs in $x$ and $\bfkperp$. We have to 
keep in mind that TMDs diverge for large $\bfkperp$. Therefore, without 
regularizing these divergences in a scheme suitable for the comparison of 
left and right hand side, a test of the above relations is meaningless, 
even before we get to address the issues of lattice calculations. Let us 
not give up at this point and take a look at the lattice observables of 
Ref.~\cite{Musch:2010ka}. Here, the TMDs are obtained from amplitudes 
$\tilde A_i(l^2,\ldots)$ in Fourier space, where $\bfkperp$ is encoded 
in the Fourier conjugate variable $\bflperp$, which is the transverse 
displacement of quark operators in the correlator evaluated on the lattice. 
In Fourier-space, the aforementioned divergent behavior for large $\bfkperp$ 
translates into strong lattice scale and scheme dependences at short distances
$\bflperp$ between the quark operators. The $\bfkperp$ integrals needed for 
the left hand sides of Eqs.~(\ref{Eq:WW-approx-g1T-d},~\ref{Eq:WW-approx-h1L-d})
correspond to the amplitudes at $\bflperp = 0$, where scheme and 
scale-dependence is greatest.  In Ref.~\cite{Musch:2010ka} Gaussian fits 
have been performed to the amplitudes \emph{excluding} data at short quark 
separations $\bflperp$. The Gaussians describe the long range data quite well 
and bridge the gap at short distances $\bflperp$. 
Taking the Gaussian fit at $\bflperp = 0$, we get a value which is 
(presumably) largely lattice scheme and scale independent. We have thus 
swept the problem of divergences under the rug. The Gaussian fit acts as 
a crude regularization of the divergences that appear in TMDs at large 
$\bfkperp$ and manifest themselves as short range artefacts on the lattice. 
Casting this line of thought into mathematics, we get
\ba
    	\int_{-1}^1 d x\; g_{1T}^{\perp(1)q}(x) 
	= & \int_{-1}^1 d x\; \int  d^2 \bfkperp 
	\frac{\kperp^2}{2M^2} g_{1T}^{\perp q}(x,\kperp) 
	= -2 \tilde{A}_{7,q}( \ell = 0 ) 
	\stackrel{\text{\scriptsize Gaussian}}{=} -c_{7,q} \quad \quad \\
    	\int_{-1}^1 d x\; h_{1L}^{\perp(1)q}(x) 
	= & \int_{-1}^1 d x\; \int  d^2 \bfkperp 
	\frac{\kperp^2}{2M^2} h_{1T}^{\perp q}(x,\kperp) 
	= -2 \tilde{A}_{10,q}( \ell = 0 )
	\stackrel{\text{\scriptsize Gaussian}}{=} -c_{10,q} \quad \quad
\ea
where the amplitudes $\tilde{A}$ and constants $c$ are those of Ref.~\cite{Musch:2010ka}.
We have thus expressed the left hand side of 
Eqs.~(\ref{Eq:WW-approx-g1T-d},~\ref{Eq:WW-approx-h1L-d}) in terms of 
amplitudes $c_{7,q}$ and $c_{10,q}$ of the Gaussian fits on the lattice.
Before quoting numbers, a few more comments are in order. The overall 
multiplicative renormalization in Ref.~\cite{Musch:2010ka} was fixed by 
setting the Gaussian integral $c_{2,u-d}$ of the unpolarized TMD $f_1$ 
in the isovector channel (u-d) to the nucleon quark content, namely, to 1.
 One then assumes that the normalization of the lattice results for the 
unpolarized TMD $f_1$ also fixes the normalization for polarized quantities 
correctly. This assumption holds if renormalization is multiplicative and 
flavor-independent for the non-local lattice operators. This is not true 
for all lattice actions \cite{Yoon:2017qzo} %PhysRevD.96.094508
but presumably it is true if the lattice action preserves chiral symmetry, 
as it does in the present case.
The Gaussian fits along with the normalization prescription serve as
a crude form of renormalization, and this is needed to attempt
a comparison of left and right hand sides of equations  
Eqs.~(\ref{Eq:WW-approx-g1T-d},~\ref{Eq:WW-approx-h1L-d}).

There is yet another issue to be discussed. 
The gauge link that goes into the evaluation of the quark-quark correlator 
introduces a power divergence that has to be subtracted. 
Ref.~\cite{Musch:2010ka} employs a subtraction scheme on the lattice
but establishes no connection with a subtraction scheme designed for 
experimental TMDs and the corresponding gauge link geometry. 
The gauge link renormalization mainly 
influences the width of the Gaussian fits; the amplitudes are only slightly 
affected, so it may not play a big role for our discussion. Altogether, the 
significance of our numerical ``tests'' of WW-relations should be taken 
with a grain of salt.

For the test of Eq.~(\ref{Eq:WW-approx-g1T-d}), we use the numbers 
$\int d x\;g_{1T}^{\perp(1)u}(x)\stackrel{\text{\scriptsize Gaussian}}{=}
-c_{7,u}= 0.1041(85)$ and
$\int d x\;g_{1T}^{\perp(1)d}(x)\stackrel{\text{\scriptsize Gaussian}}{=}
-c_{7,d}=-0.0232(42)$
from \cite{Musch:2010ka}. %[Phys.Rev. D83 (2011) 094507]:
Lattice data for
$\int d x \,x^{N}g_1^q(x)$
\cite{Hagler:2003is,Hagler:2007xi} and
$\int d x \,x^{N}h_1^q(x)$
\cite{Gockeler:2005cj} are available for $N=0,\,1,\,2,\,3$ .
These values have been computed using (quasi-) local operators which 
have been renormalized to the $\overline{MS}$ scheme at the scale 
$\mu^2 = 4\,\text{GeV}^2$.
According to \cite{Hagler:2007xi} (data set 4:
% lattice spacing $a$ and current quark masses $m_{u,d}$ such that
with $a\,m_{u,d} = 0.020$ with $m_\pi\approx 500\,{\rm MeV}$)
one has $\int d x \;x\,g_1^{u-d}(x)= 0.257(10)$ and
$\int d x \;x\,g_1^{u+d}(x)= 0.159(14)$.
Decomposing the results from  \cite{Hagler:2007xi} into
individual flavors, and inserting them into
Eq.~(\ref{Eq:WW-approx-g1T-d}) we obtain
\ba
        \underbrace{\int d x\;g_{1T}^{\perp(1)u}(x)}
        _{= 0.1041(85) \;\mbox{\footnotesize Ref.~\cite{Musch:2010ka}}}
        &\stackrel{!}{\approx}
        \underbrace{\frac{1}{2}\int d x\;x\,g_1^u(x)}
        _{= 0.104(9) \;\mbox{\footnotesize Ref.~\cite{Hagler:2007xi}}}
        \hspace{3mm} , \nonumber \\
        \underbrace{\int d x\;g_{1T}^{\perp(1)d}(x)}
        _{= -0.0232(42) \;\mbox{\footnotesize Ref.~\cite{Musch:2010ka}}}
        &\stackrel{!}{\approx}
        \underbrace{\frac{1}{2}\int d x\;x\,g_1^d(x)}
        _{= -0.025(9) \;\mbox{\footnotesize Ref.~\cite{Hagler:2007xi}}}
        \hspace{3mm}, 
        \label{Eq:test-WW-type-lattice-g1T}
\ea
which confirms the approximation (\ref{Eq:WW-approx-g1T-d}) for $N=0$
within the statistical uncertainties of the lattice calculations.
%
In order to test (\ref{Eq:WW-approx-h1L-d}) we use
% old numbers
% $\int d x\;h_{1L}^{\perp(1)u}(x) = -0.0931(73)$ and
% $\int d x\;h_{1L}^{\perp(1)d}(x) = 0.0130(40)$ from \cite{Hagler:2009mb}
$\int dx\;h_{1L}^{\perp(1)u}(x)\stackrel{\text{\scriptsize Gaussian}}{=}
-c_{10,u}=-0.0881(72)$ 
and
$\int dx\;h_{1L}^{\perp(1)d}(x)\stackrel{\text{\scriptsize Gaussian}}{=}
-c_{10,d}=0.0137(34)$
from \cite{Musch:2010ka} and the lattice data 
$\int d x \;x\,h_1^u(x)= 0.28(1)$ and
$\int d x \;x\,h_1^d(x)= -0.054(4)$
from QCDSF \cite{Gockeler:2005cj}.\footnote{
  These numbers are read off from a figure in \cite{Gockeler:2005cj},
  and were computed on a different lattice. We interpolate them to a
  common value of the pion mass $m_\pi\approx500\,{\rm MeV}$, and
  estimate the uncertainty conservatively in order to take systematic effects
  into account due to the use of a different lattice.}
Inserting these numbers into  (\ref{Eq:WW-approx-h1L-d}) for the case
$N=0$ we obtain
\ba
        \underbrace{\int d x\;h_{1L}^{\perp(1)u}(x)}
        _{= -0.0881(72)\;\mbox{\footnotesize Ref.~\cite{Musch:2010ka}}}
        &\stackrel{!}{\approx}&
        \underbrace{-\,\frac{1}{3}\int d x\;x\,h_1^u(x)}
        _{= -0.093(3) \;\mbox{\footnotesize Ref.~\cite{Hagler:2007xi}}}
        \hspace{3mm} , \hspace{7mm} \nonumber \\
        \underbrace{\int d x\;h_{1L}^{\perp(1)d}(x)}
        _{= 0.0137(34) \;\mbox{\footnotesize Ref.~\cite{Musch:2010ka}}}
        &\stackrel{!}{\approx}&
        \underbrace{-\,\frac{1}{3}\int d x\;x\,h_1^d(x)}
        _{= 0.018(1) \;\mbox{\footnotesize Ref.~\cite{Hagler:2007xi}}}
        \hspace{3mm}. 
        \label{Eq:test-WW-type-lattice-h1L}
\ea
which again confirms the WW-type approximation within the statistical
uncertainties of the lattice calculations.

Several more comments are in order concerning the, at first glance, remarkably
good confirmation of the  WW-type approximations by lattice data in
Eqs.~(\ref{Eq:test-WW-type-lattice-g1T},~\ref{Eq:test-WW-type-lattice-h1L}).

First, the relations refer to lattice parameters corresponding
to pion masses of $500\,{\rm MeV}$. We do not
need to worry about that too much. The lattice results do provide
a valid test of the approximations in a ``hadronic world'' with
somewhat heavier pions and nucleons. All that matters in our
context is that the relative size of $\bar{q}gq$--matrix elements
is small with respect to $\bar{q}q$--matrix elements.

Second, we have to revisit carefully which approximations the above
lattice calculations actually test. As mentioned above, in
the lattice study \cite{Hagler:2009mb,Musch:2010ka} a specific choice for
the path of the gauge link was chosen, which is actually different
from the paths required in SIDIS or Drell-Yan. With the path choice of
\cite{Hagler:2009mb,Musch:2010ka} there are effectively only (T-even)
$A_i$ amplitudes, the $B_i$ amplitudes are absent.
Therefore the test (\ref{Eq:test-WW-type-lattice-g1T}) of the WW-type
approximation (\ref{Eq:WW-approx-g1T-d}) actually constitutes a test
of the WW-approximation (\ref{Eq:WW-original1}) and confirms
earlier lattice work \cite{Gockeler:2000ja,Gockeler:2005vw},
cf.\ Refs.~\cite{Metz:2008ib,Teckentrup:2009tk} and
Sec.~\ref{Sec-3.6:models}.
Similarly, the test (\ref{Eq:test-WW-type-lattice-h1L}) of the
WW-type approximation (\ref{Eq:WW-approx-h1L-d}) actually constitutes
a test of the WW-approximation (\ref{Eq:WW-original2}). The latter
however has not been reported previously in literature, and is a
new result.

Third, to be precise:
Eqs.~(\ref{Eq:test-WW-type-lattice-g1T},~\ref{Eq:test-WW-type-lattice-h1L})
test the first Mellin moments of the WW approximations
(\ref{Eq:WW-original1},~\ref{Eq:WW-original2}), which corresponds to the
Burkhardt-Cottingham sum rule for $g_T^a(x)$ and an analogous sum rule for
$h_L^a(x)$, see \cite{Jaffe:1996zw} and references there in.
In view of the long debate on the validity of those sum rules
\cite{Burkardt:2001iy,Bass:2003vp,Efremov:2002qh}, this is in
an interesting result in itself.

It is important to stress that in view of the pioneering and
exploratory status of the TMD lattice calculations
\cite{Hagler:2009mb,Musch:2010ka}, this is already a remarkable and very
interesting result. Thus, apart from the instanton calculus
\cite{Dressler:1999hc} also lattice data provide support for
the validity of the WW approximation (\ref{Eq:WW-original2}).
At the same time, however, we also have to admit that we do
not really reach our goal of testing the WW-type approximations
on the lattice. We have to wait for better lattice data. 
Meanwhile we may try to gain insights into the quality of
WW-type approximations from models.




\subsection{Tests in models}
\label{Sec-3.6:models}

Effective approaches and models such as bag 
\cite{Jaffe:1991ra,Stratmann:1993aw,Signal:1996ct,Avakian:2010br},
spectator \cite{Jakob:1997wg}, chiral quark-soliton 
\cite{Wakamatsu:2000ex}, or light-cone 
constituent \cite{Pasquini:2008ax,Lorce:2011dv} models
support the approximations (\ref{Eq:WW-original1},~\ref{Eq:WW-original2}) 
for PDFs within an accuracy of $(10-30)\,\%$ at low hadronic scale 
below $1\,{\rm GeV}$. 

Turning to TMDs, we recall that in models without gluon 
degrees of freedom certain relations among TMDs hold, the 
so-called quark model Lorentz-invariance relations (qLIRs)
\cite{Tangerman:1994bb,Mulders:1995dh}.\footnote{Notice that
	the qLIRs of \cite{Tangerman:1994bb,Mulders:1995dh} are 
	valid only in quark models with no gluons and should not
	be confused with the LIRs of \cite{Kanazawa:2015ajw} which 
	are exact relations in QCD, see Sec.~\ref{Sec-3.1:WW-classic}.
	In literature both are often simply referred to as LIRs.
	This ambiguity is unfortunate.}
Initially thought to be exact \cite{Tangerman:1994bb,Mulders:1995dh}
qLIRs were shown  to be invalid in models with gluons 
\cite{Kundu:2001pk,Schlegel:2004rg} and in QCD \cite{Goeke:2003az}.
They originate from decomposing the (completely unintegrated)
quark correlator in terms of Lorentz-invariant amplitudes, and 
TMDs are certain integrals over those amplitudes.
When gluons are absent, the correlator consists
of 12 amplitudes \cite{Tangerman:1994bb,Mulders:1995dh}, i.e.\ fewer 
amplitudes than TMDs which implies relations: the qLIRs. 
In QCD the correct Lorentz decomposition requires the consideration of 
gauge links which introduces further amplitudes. As a result one has 
as many amplitudes as TMDs and no relations exist \cite{Goeke:2003az}. 
However, qLIRs ``hold'' in QCD in the WW-type approximation 
\cite{Metz:2008ib}. In models without gluon degrees of freedom 
they are exact
\cite{Metz:2008ib,Teckentrup:2009tk,Avakian:2010br,Jakob:1997wg}. 

The bag, spectator and light-cone constituent quark models support 
the approximations (\ref{Eq:WW-approx-g1T},~\ref{Eq:WW-approx-h1L}) 
within an accuracy of $(10-30)\,\%$ 
\cite{Jakob:1997wg,Pasquini:2008ax,Avakian:2010br,Lorce:2011dv}.
The spectator and bag model support WW-type approximations 
within $(10-30)\,\%$ \cite{Avakian:2010br}. 
As they are defined in terms of quark bilinear expressions 
(\ref{Eq:correlator}) it is possible to evaluate twist-3 functions
in quark models \cite{Jaffe:1991ra}. The tilde-terms arise due to
the different model interactions, and it is important to discuss
critically how realistically they describe the $\bar{q}gq$--terms
of QCD \cite{Lorce:2014hxa,Lorce:2016ugb}.

In the covariant parton model with intrinsic 3D-symmetric parton 
orbital motion \cite{Zavada:1996kp}  quarks are free, $\bar{q}gq$ 
correlations absent, and all WW and WW-type relations exact
\cite{Efremov:2010mt,Efremov:2009ze}.
The phenomenological success of this approach \cite{Zavada:1996kp} may 
hint at a general smallness of $\bar{q}gq$ terms, although many of the 
predictions from this model have yet to be tested \cite{Efremov:2010mt}.

Noteworthy is the result from the chiral quark soliton
model where the WW-type approximation (\ref{Eq:WW-type-Cahn})
happens to be exact: $xf^{\perp q}(x,\kperp^2)=f_{1}^q(x,\kperp^2)$
for quarks and antiquarks \cite{Lorce:2014hxa}. The degrees of freedom
in this model are quarks, antiquarks and Goldstone bosons which are 
strongly coupled (the coupling constant is $\sim 4$) and has to be
solved using nonperturbative techniques (expansion in $1/N_c$ where
$N_c$ is the number of colors) with the nucleon described as a 
chiral soliton. In general the model predicts non-zero tilde-terms, for 
instance $\tilde{e}^a(x)\neq 0$ 
\cite{Schweitzer:2003uy,Ohnishi:2003mf,Cebulla:2007ej}.
However, despite strong interactions in this effective theory, the tilde 
term $\tilde{f}^{\perp q}(x,\kperp^2)$ vanishes exactly in this model 
\cite{Lorce:2014hxa} and the WW-type approximation (\ref{Eq:WW-type-Cahn})
becomes exact at the low initial scale of this model of 
$\mu_0\sim 0.6\,{\rm GeV}$.

Let us finally discuss quark-target models, 
where gluon degrees of freedom are included and WW(-type)
approximations badly violated
\cite{Kundu:2001pk,Schlegel:2004rg,Meissner:2007rx,Mukherjee:2009uy}.
This is natural in this class of models for two
reasons. First, quark-mass terms are of ${\cal O}(m_q/M_N)$ 
and negligible in the nucleon case, but of ${\cal O}(100\,\%)$
in a quark target where $m_q$ plays also the role of $M_N$. 
Second, even if one refrains from mass terms the approximations are 
spoiled by gluon radiation, see for instance \cite{Harindranath:1997qn} 
in the context of (\ref{Eq:WW-original1}).
This means that perturbative QCD does not support the WW-approximations:
they certainly are not preserved by evolution. However, scaling violations
{\it per se} do not need to be large. What is crucial in this context are 
dynamical reasons for the smallness of the {\sl matrix elements} of
$\bar{q}gq$--operators. This requires the consideration of chiral symmetry 
breaking effects reflected in the hadronic spectrum, as considered in the
instanton vacuum model \cite{Balla:1997hf,Dressler:1999hc} but 
out of scope in quark-target models.

We are not aware of systematic tests of WW-type approximations for FFs. One 
information worth mentioning in this context is that in spectator models 
\cite{Jakob:1997wg} tilde-contributions to FFs are proportional to the 
offshellness of partons % in complete analogy to the case of TMDs 
\cite{Lorce:2014hxa,Lorce:2016ugb}. This
natural feature may indicate that in the region dominated by effects of
small $P_\perp$ tilde-terms might be small. On the other hand, quarks have 
sizable constituent masses of the order of few hundred MeV in spectator models 
and the mass-terms are not small. 
The applicability of WW-type approximations to FFs 
remains the least tested point in our approach.

\newpage

%\newpage
\subsection{Basis functions for the WW-type approximations}
\label{Sec-3.7:basis}

The leading--twist 6 TMDs 
$f_1^a, \; f_{1T}^{\perp a}, \; g_1^a, \; h_1^a, \;h_1^{\perp a},\; h_{1T}^{\perp a}$
and 2 FFs $D_1^a, \; H_1^{\perp a}$ provide a basis
\kt{in the sense that in WW-type approximation all other TMDs and FFs 
can either be expressed in terms of these basis functions or vanish.}
Below we shall see that, under the assumption of the validity of WW-type 
approximations, it is possible to express all SIDIS structure functions
in terms of the basis functions. 
Notice that SIDIS alone is of course not sufficient to determine the basis
functions uniquely: the 8 basis functions appear in 6 SIDIS structure functions.
It is crucial to take advantage of other processes: Drell-Yan for PDFs and TMDs
and hadron production in $e^+e^-$ annihilation for FFs. Other processes 
play also important roles.

These basis functions allow us to describe, in WW-type approximation,
all other TMDs. The experiment will tell us how well the approximations work.
In some cases, however, we know in advance that the WW-type approximations 
have limitations, see next section.

\subsection{Limitation of WW-type approximation}
\label{Sec-3.8:limitations}

The approximation may work in the case when: TMD or FF 
$=\la\bar{q}q\ra + \la\bar{q}gq\ra \approx \la\bar{q}q\ra \neq 0$ with 
a ``controlled approximation'' in the spirit of Eq.~(\ref{Eq:WW-generic}).
We know cases where this works, see
Secs.~\ref{Sec-3.3:WW-classic-instanton}, \ref{Sec-3.4:WW-classic-experiment},
but it has to be checked case by case whether
$| \la\bar{q}gq\ra  | \ll  |\la\bar{q}q\ra|$ for a given operator.
At least in such cases the approximation has a chance to work.

However, it may happen that after applying the QCD equations
of motion one ends up in the situation that: a given function 
$=\la\bar{q}q\ra + \la\bar{q}gq\ra$ with $\la\bar{q}q\ra = 0$.
This happens for the T-even TMD 
	$e^a$ in Eqs.~(\ref{Eq:WW-original-e},~\ref{Eq:WW-type-1}), 
for the T-odd TMDs 
	$e_L^q$, 
	$e_T^q$,
	$e_T^{\perp q}$, 
	$f_L^{\perp q}$,
	$g^{\perp q}$ in Eqs.~(\ref{Eq:WW-type-eLperp}--\ref{Eq:WW-type-fLperp}), 
and for the FFs
	$E^q$, 
	$G^{\perp q}$ in Eqs.~(\ref{Eq:WW-type-FF-1},~\ref{Eq:WW-type-FF-2})
(actually, all twist-3 FFs are affected, we will discuss this in detail below). 
In this situation the ``leading term'' is absent, so neglecting the 
``subleading (pure twist-3) term'' actually constitutes an error of $100\,\%$
even if the neglected matrix element  $\la\bar{q}gq\ra$ is very small.
Notice that this type occurs for all subleading twist FFs, which enter 
SIDIS structure functions only in the shape of tilde-FFs, see 
Sec.~\ref{Sec-2:SIDIS+TMDs+FF} and 
Eqs.~(\ref{Eqs:structure-functions-twist-3}). 
We shall see that some structure functions are potentially more and 
others potentially less affected by this generic limitation. In any 
case phenomenological work has to be carried out to find out whether 
or not the approximation works. 


For both FFs and TMDs there are also limitations 
which go beyond this generic issue. To illustrate this for FFs
we recall that both $H_1^{\perp(1)q}$ and $\tilde H_1^{q}$ are 
related to integrals of an underlying function $H_{FU}^{q,\Im}(z,z_1)$
as pointed out in Ref.~\cite{Kanazawa:2015ajw}. Therefore, if one 
literally assumed $\tilde H^q(z)$ to be zero, this would imply that 
also $H_1^{\perp(1)q}$ would vanish, indicating that the WW-type 
approximation has to be used with care in the chiral-odd FF sector.

Similar limitations exist also for TMDs. This is manifest in particular 
for those twist-3 T-odd TMDs which appear in the decomposition of the 
correlator (\ref{Eq:correlator}) with no prefactor of $\kperp$.
There are three cases: $f_T^a(x,k_\perp)$, $h^a(x,\kperp)$, $e_L^a(x,\kperp)$.
Such TMDs in principle survive integration of the correlator over $\kperp$
and would have PDF counterparts if there were not the sum rules in 
Eq.~(\ref{Eq:sum-rules-T-odd}). These sum rules arise because hypothetical
PDF versions of T-odd TMDs vanish: they have a simple straight gauge link
along the lightcone, and such objects vanish due to parity and time-reversal 
symmetry of strong interactions. This argument does not apply to other T-odd 
TMDs because they drop out from the $\kperp$--integrated correlator due to 
explicit factors of e.g.\ $\kperp^j$ in the case of the Sivers function.

Let us first discuss the case of $f_T^a(x,k_\perp)$. Taking the  
WW-type approximation (\ref{Eq:WW-type-fT}) literally means
$x\int d^2 k_\perp\,f_T^a(x,k_\perp)\,\stackrel{!?}{=}
-f_{1T}^{\perp(1)a}(x)\neq0$ 
at variance with the sum rule (\ref{Eq:sum-rules-T-odd}). We 
have $xf_T^a(x,k_\perp)=x\tilde{f}_T^a(x,k_\perp)-f_{1T}^{\perp(1)a}(x,k_\perp)$ 
from QCD equations of motion \cite{Bacchetta:2006tn} which yields
(\ref{Eq:WW-type-fT}). The point is that in this case it is 
essential to keep the tilde-function. 
The situation for the chirally and T-odd twist-3 
TMD $h^a(x,k_\perp)$ is analogous. The third 
function in (\ref{Eq:sum-rules-T-odd}) causes no issues since 
$e_L^a(x,k_\perp)=\tilde{e}_L^a(x,k_\perp)\approx0$ in WW-type approximation.

Does it mean WW-type approximations fail for $f_T^a(x,k_\perp)$ 
and $h^a(x,k_\perp)$? Not necessarily. The approximations may
work in some but not all regions of $\kperp$, but the sum rules 
(\ref{Eq:sum-rules-T-odd}) include integration over all $k_\perp$. 
Notice also that e.g.\ $f_{1T}^{\perp (1),q}(x)$ is related to the 
soft-gluon-pole matrix element $F_{FT}(x,x)$ \cite{Boer:2003cm,Ji:2006ub}
which is a $\bar{q}gq$-term that one would naturally neglect 
in WW-type approximation.
In this sense (\ref{Eq:WW-type-fT}) could be consistent.
Thus, issues with the sum rules 
(\ref{Eq:sum-rules-T-odd}) do not need to exclude the
possibility that the WW-type approximations for $f_T^a(x,k_\perp)$ and
$h^a(x,k_\perp)$ in (\ref{Eq:WW-type-fT},~\ref{Eq:WW-type-last}) 
may work at small $k_\perp$ where we use them in our TMD approach.
This would mean the UV region is essential to realize the sum rules 
(\ref{Eq:sum-rules-T-odd}). Alternatively, one could also envision 
the sum rules (\ref{Eq:sum-rules-T-odd}) to be sensitive to the
IR region through gluonic or fermionic pole contributions manifest
in tilde-terms. 

Presently too little is known in the theory of subleading twist TMDs. 
Below in Sec.~\ref{Sec-7.6:FUTsinphiS} and \ref{Sec-7.7:FUUcosphi} we 
will present a pragmatic solution how to deal with the TMDs 
$f_T^a(x,k_\perp)$ and $h^a(x,k_\perp)$ phenomenologically. 
For now let us keep in mind that 
one has to keep a vigilant eye on all WW-type approximations, and 
especially on those for $f_T^a(x,k_\perp)$ and $h^a(x,k_\perp)$.



%\newpage
%======= SECTION 4: SIDIS IN WW APPROXIMATION ========================
\section{SIDIS in WW-type approximation and Gaussian model}
\label{Sec-4:SIDIS-in-WW-approximation}

In this section we consequently apply the WW- and WW-type approximation
to SIDIS, and describe our procedure to evaluate the structure 
functions in this approximation and the Gaussian Ansatz which we use
to model the $k_\perp$ dependence of TMDs.

\subsection{Leading structure functions amenable to WW-type approximations}
\label{Sec-4.1:WW-twist-2}

The WW- and WW-type approximations are useful for the following
two leading--twist structure functions 
\begin{subequations}\ba
 F_{LT}^{\cos(\phi_h -\phi_S)}
	&\stackrel{\rm WW}{=}& 
	{\cal C}\biggl[\omega^{\{1\}}_{\rm B}\, {g_{1T}^\perp}D_1 \biggr]
        \with{ }{g_{1T}^{\perp a}\to g_1^a}{Eq.~(\ref{Eq:WW-approx-g1T})}
        \label{F_LTcos(phi-phiS)-WW} \\
 F_{UL}^{\sin 2\phi_h} 	
        &\stackrel{\rm WW}{=}& 
	{\cal C}\biggl[\omega^{\{2\}}_{\rm AB}\,
    	{h_{1L}^{\perp }} H_{1}^{\perp }\biggr]  
        \with{ }{h_{1L}^{\perp a}\to h_1^a}{Eq.~(\ref{Eq:WW-approx-h1L})} 
        \label{F_UUsin2phi-WW}
\ea\end{subequations}

\newpage
\subsection{Subleading structure functions in WW-type approximations}
\label{Sec-4.2:WW-twist-3}

In the case of the subleading twist structure functions the WW-type 
approximations in Eqs.~(\ref{Eq:WW-type-1}--\ref{Eq:WW-type-last})
lead to considerable simplifications. We obtain the approximations
\begin{subequations}\ba
F_{LU}^{\sin\phi_h} &\stackrel{\rm WW}{=}& 0\,, \phantom{\frac11}
	\label{Eq:WW-type-FLUsinphi}\\
F_{LT}^{\cos \phi_S}&\stackrel{\rm WW}{=}& \frac{2M}{Q}\,
	{\cal C}\biggl[-  \omega^{\{0\}}\, x\,g_T D_1 \biggr]
        \with{ }
	{g_T^a\to g_1^a}
	{Eq.~(\ref{Eq:WW-original1}),}\\
F_{LL}^{\cos \phi_h} &\stackrel{\rm WW}{=}& \frac{2M}{Q}\,{\cal C}\biggl[ 
   	-\omega^{\{1\}}_{\rm B}\,
   	xg_L^{\perp} D_1 \biggr]
        \with{ }
	{g_L^{\perp a}\to g_1^a}
	{Eq.~(\ref{Eq:WW-type-gLperp}),}\\
F_{LT}^{\cos(2\phi_h - \phi_S)} &\stackrel{\rm WW}{=}& \frac{2M}{Q}\,{\cal C}\biggl[
   	- \omega^{\{2\}}_{\rm C}\,
   	x g_T^{\perp } D_1 \biggr]
        \with{ }
	{g_T^{\perp a}\to g_1^a}
	{Eqs.~(\ref{Eq:WW-type-gTperp},~\ref{Eq:WW-approx-g1T}),}\\
F_{UL}^{\sin\phi_h} &\stackrel{\rm WW}{=}& \frac{2M}{Q}\,{\cal C}\biggl[
   	\phantom{-}\omega^{\{1\}}_{\rm A}\,
    	x\,h_L  H_1^{\perp } \biggr]
        \with{ }
	{h_L^a\to h_{1L}^{\perp a}}
	{with Eq.~(\ref{Eq:WW-type-6}),}
	\label{Eq:WW-type-FULsinphi}\\
F_{UU}^{\cos\phi_h} &\stackrel{\rm WW}{=}&\frac{2M}{Q}\,{\cal C}\biggl[\phantom{-}
	 \omega^{\{1\}}_{\rm A}\,x\,h\,H_{1}^{\perp } 
   	-\omega^{\{1\}}_{\rm B}\,x\,f^\perp D_1\biggr]
        \with{ }
	{f^{\perp a}\to f_1^a, \; h^a\to h_1^{\perp a}}
	{with Eqs.~(\ref{Eq:WW-type-Cahn},~\ref{Eq:WW-type-last}),}
	\label{Eq:WW-type-FUUcosphi}\\
F_{UT}^{\sin \phi_S } &\stackrel{\rm WW}{=}& \frac{2M}{Q}\,
	{\cal C}\biggl[ \phantom{-}\omega^{\{0\}} \, x\,f_TD_1
	-\frac{\omega^{\{2\}}_{\rm B}}{2}\,(xh_T-xh_T^\perp)\,H_{1}^{\perp } \biggr]
        \with
	{f_T^a \to f_{1T}^{\perp a},}
	{h_T^a-h_T^{\perp a}\to h_1^a}
	{(\ref{Eq:WW-type-fT},~\ref{Eq:WW-type-7},~\ref{Eq:WW-type-8}),}
	\label{Eq:WW-type-FUTsinphiS}\\
F_{UT}^{\sin(2\phi_h -\phi_S)} &\stackrel{\rm WW}{=}& \frac{2M}{Q}\,{\cal C}\biggl[
   	\;\omega^{\{2\}}_{\rm C}\,
   	{  x\,f_T^\perp}D_1
        + \frac{\omega^{\{2\}}_{\rm AB}}{2} 
	{x(h_T+h_T^\perp)}H_1^\perp \biggr]
        \with
	{f_T^{\perp a}\to f_{1T}^{\perp a},}
	{(h_T^a+h_T^{\perp a})\to h_{1T}^{\perp a}}{with 
	(\ref{Eq:WW-type-fTperp},~\ref{Eq:WW-type-7},~\ref{Eq:WW-type-8}).}
	\;\;\;\;
\ea\end{subequations}


\subsection{Gaussian Ansatz for TMDs and FFs}
\label{Sec-4.3:evaluation}

In this work we will use the so-called Gaussian Ansatz for the TMDs and FFs.
This Ansatz, which for a generic TMD or FF is given by
\be\label{Eq:Gauss-generic}
    f(x,\kperp^2) = f(x)\;
    \frac{e^{-\kperp^2/\avkperp}}{\pi\avkperp} \;,\;\;\;
    D(z,\pperp^2) = D(z)\,
    \frac{e^{-\pperp^2/\avpperp}}{\pi\avpperp}\;,
\ee
is popular not only because it considerably simplifies the
calculations. In fact, all convolution integrals of the type 
(\ref{Eq:def-convolution-integral}) can be solved analytically 
with this Ansatz. Far more important is the fact that it works 
phenomenologically with a good accuracy in many practical applications
\cite{Anselmino:2005nn,Collins:2005ie,D'Alesio:2007jt,Schweitzer:2010tt,
Signori:2013mda,Anselmino:2013lza}.
Of course this Ansatz is only a rough approximation. For instance, 
it is not consistent with general matching expectations when $\kperp$ 
becomes large \cite{Bacchetta:2008xw}. 

Nevertheless, if one limits oneself to work in a regime where the 
transverse momenta (of hadrons produced in SIDIS, dileptons produced
in the Drell-Yan process, etc) are small compared to the hard
scale in the process, then the Ansatz works quantitatively
very well. The most recent and detailed tests were reported in 
\cite{Schweitzer:2010tt}, where the Gaussian Ansatz was shown to
describe the most recent SIDIS data: no deviations were observed 
within the error bars of the data provided one takes into account 
the broadening of the Gaussian widths with increasing energy 
\cite{Schweitzer:2010tt} according
with expectations from QCD \cite{Aybat:2011zv}.
The Gaussian Ansatz is approximately compatible with 
the $\kperp$--shapes obtained from evolution \cite{Aybat:2011zv}
or fits to high-energy Tevatron data on weak-boson production
\cite{Landry:2002ix}. Effective models at 
low \cite{Pasquini:2008ax,Avakian:2010br,Lorce:2011dv} and 
high \cite{Efremov:2009ze} renormalization scales support this
Ansatz as a good approximation.

\subsection{Evaluation of structure functions in WW-type \&
 Gaussian approximation}
\label{Sec-4.4:evaluation}

The Gaussian Ansatz is compatible with many WW-type approximations
but not all. The trivial approximations (\ref{Eq:WW-type-1}) and
(\ref{Eq:WW-type-eLperp}--\ref{Eq:WW-type-fLperp}) of course 
cause no issue. 
The Gaussian Ansatz can also be applied to the nontrivial approximations 
in Eqs.~(\ref{Eq:WW-type-Cahn}--\ref{Eq:WW-type-gTperp})
and (\ref{Eq:WW-type-fTperp}), provided the corresponding Gaussian 
widths are defined to be equal to each other: for example, in the 
WW-type approximation (\ref{Eq:WW-type-Cahn}),
$xf^{\perp q}(x,\kperp^2)\approx f_1^q(x,\kperp^2)$, one may 
assume Gaussian $\kperp$--dependence for $f^{\perp q}(x,\kperp^2)$ 
and for $f_1^q(x,\kperp^2)$ as long as the Gaussian widths
of these two TMDs are assumed to be equal.

In the case of the approximations 
(\ref{Eq:WW-type-gT}--\ref{Eq:WW-type-8})
the situation is different because here twist-3 TMDs
are related to transverse moments of twist-2 TMDs. In such cases the 
Gaussian Ansatz is not compatible with the WW-type approximations:
for instance, the approximation (\ref{Eq:WW-type-gT}) relates
$xg_T^q(x,\kperp^2)\approx\frac{k_\perp^2}{2M_N^2}\,g_{1T}^{q}(x,\kperp^2)$,
i.e.\ if $g_{1T}^q(x,\kperp^2)$ was exactly Gaussian then 
$g_T^q(x,\kperp^2)$ certainly could not be Gaussian. If one wanted to take
the Gaussian Ansatz and WW-type approximations literally, one clearly
would deal with an incompatibility. However, we of course must keep 
in mind that both are approximations. 

Some comments are in order to understand how the usage of the Gaussian 
Ansatz and the WW-type approximations can be reconciled.
First, let us remark that the individual TMDs, say 
$g_T^q(x,\kperp^2)$ and $g_{1T}^{q}(x,\kperp^2)$ in our example,
may each by itself be assumed to be approximately Gaussian in $k_\perp$
which is supported by quark model calculations \cite{Avakian:2010br}.
Second, we actually do not need the unintegrated WW-type approximations. 
For phenomenological applications we can use the WW-type approximations 
in ``integrated form.'' 

Let us stress that if one took an unintegrated WW-type approximation of the 
type $xg_T^q(x,\kperp^2)\approx\frac{k_\perp^2}{2M_N^2}\,g_{1T}^{q}(x,\kperp^2)$
literally and assumed both TMDs to be exactly Gaussian, one would find
``incompatibilities:'' perhaps most strikingly in the limit $k_\perp\to 0$
where the left-hand side is finite while the right-hand side vanishes.
However, such incompatibilities are washed out and not apparent after
the convolution integrals defining the structure functions are solved.
Notice that the failure of the WW-type approximations
(\ref{Eq:WW-type-gT}--\ref{Eq:WW-type-8}) in the limit $k_\perp\to 0$ is 
not specific to the Gaussian model,
but a general feature caused by neglecting tilde-terms. This indicates 
a practical scheme how to use responsibly the WW-type approximations in 
Eqs.~(\ref{Eq:WW-type-gT}--\ref{Eq:WW-type-8}).

Our procedure is as follows. In a first step we assume that all TMDs and 
FFs are (approximately) Gaussian, and solve the convolution integrals.
In the second step we use the integrated WW-type approximations to
simplify the results for the structure functions.

Notice that in some cases (when T-even TMDs are involved) 
one could choose a different order of the steps: first apply 
WW-type approximations and then solve convolution integrals 
with Gaussian Ansatz.
In general this would yield different (and bulkier) analytical 
expressions, but we convinced ourselves that the differences 
are numerically within the accuracy expected for this approach.
However, for the structure functions discussed in
Secs.~\ref{Sec-7.6:FUTsinphiS} and \ref{Sec-7.7:FUUcosphi}
such an ``alternative scheme'' would give results at
variance with the sum rules in for the twist-3 T-odd TMDs
in Eq.~(\ref{Eq:sum-rules-T-odd}), as discussed in 
Sec.~\ref{Sec-3.8:limitations}. 
The scheme presented here will allow us to implement those sum 
rules in a convenient and consistent way. We will follow up on 
this in more detail in 
Secs.~\ref{Sec-7.6:FUTsinphiS} and \ref{Sec-7.7:FUUcosphi}.

To summarize, our procedure is to solve first the convolution 
integrals in Gaussian Ansatz, and use then WW-type approximations.
When implementing this procedure we will see that the results 
for the structure functions can be conveniently expressed in 
terms of the basis TMDs or their adequate transverse moments. 





\subsection{Phenomenological information on basis functions}
\label{Sec-4.3:plot-basis-functions}

%%%%%%%%%%%%%%%%%%%%%%%%%%%%%%%%%%%%%%%%%%%%%%%%%%%%%%%%%%%%%%%%%%%%%%%%%%%%%%%
\begin{figure}[b!]
\centering
\includegraphics[width=0.37\textwidth]{\FigPath/f1} 
\includegraphics[width=0.37\textwidth]{\FigPath/f1Tperp}
  
\includegraphics[width=0.37\textwidth]{\FigPath/g1}  
\includegraphics[width=0.37\textwidth]{\FigPath/h1perpBM}  

\includegraphics[width=0.37\textwidth]{\FigPath/h1}  
\includegraphics[width=0.37\textwidth]{\FigPath/h1Tperp}

\includegraphics[width=0.37\textwidth]{\FigPath/D1} 
\includegraphics[width=0.37\textwidth]{\FigPath/H1perp} 
\caption{\label{basis} 
	The basis functions $f_1^a, \; g_1^a, \; h_1^a, 
	f_{1T}^{\perp a}, \;h_1^{\perp a},\; h_{1T}^{\perp a}; \; 
	D_1^a, \; H_1^{\perp a} \,$. For details see App.~\ref{App:basis}.}
\end{figure}
%%%%%%%%%%%%%%%%%%%%%%%%%%%%%%%%%%%%%%%%%%%%%%%%%%%%%%%%%%%%%%%%%%%%%%%%%%%%%%%


We have seen that the following 6 TMDs and 2 FFs provide a basis 
(Sec.~\ref{Sec-3:WW}) and allow us to express all SIDIS structure 
functions (Sec.~\ref{Sec-4:SIDIS-in-WW-approximation})  
in WW-type approximation: 
\be\label{Eq:basis}
   \mbox{basis: \ \ } 
   f_1^a, \; f_{1T}^{\perp a}, \; g_1^a, \; h_1^a, \;h_1^{\perp a},\; h_{1T}^{\perp a};
   \; D_1^a, \; H_1^{\perp a} \, .
\ee
Phenomenological information is available for all basis functions at 
least to some extent. 
In Fig.~\ref{basis} we present plots of the basis functions, and refer
to App.~\ref{App:basis} for details.
The four functions $f_1^a, \; g_1^a, \; h_1^a,\; D_1^a$  are related to 
twist-2 collinear functions. All collinear functions are calculated at 
$Q^2 = 2.4$ GeV$^2$. Collinear $f_1^a(x)$ are from Ref.~\cite{Martin:2009iq},  
$g_1^a(x)$ are from Ref.~\cite{Gluck:1998xa}, and $D_1^a$ are from 
Ref.~\cite{deFlorian:2007aj}. The other four TMDs 
have no collinear counterparts. 
For $f_{1T}^{\perp a},\;h_1^{\perp a},\;H_1^{\perp a}$ it is convenient to 
consider their (1)-moments, for $ h_{1T}^{\perp a}$  the (2)-moment,
see Eq.~(\ref{eq:moments}) for definitions.
This has two important advantages. First, this step simplifies 
the Gaussian model expressions, and the Gaussian width parameters are
largely absorbed in the definitions of the transverse moments.
% which helps to minimize the model-dependence. 
Second,
the $k_\perp$--moments of these TMDs have in principle simple definitions
in QCD (whereas e.g.\ the function $f_{1T}^{\perp q}(x)$ can be computed in
models but is very cumbersome to define in QCD).
The parametrizations for the basis functions read
\begin{subequations}\ba
	f^a_1(x,\kperp^2) &=& f^a_1(x)\;
    	\frac{1}{\,\pi\avkperp_{f_1}}\;e^{-\kperp^2/\avkperp_{f_1}} \, ,
	\label{Eq:Gauss-f1}\\
    	D^a_1(z,\pperp^2) &=& D_1^a(z)\,
    	\frac{1}{\,\pi\avpperp_{D_1}}\;e^{-\pperp^2/\avpperp_{D_1}} \, ,
	\label{Eq:Gauss-D1}\\
	g^a_1(x,\kperp^2) &=& g^a_1(x)\;
    	\frac{1}{\,\pi\avkperp_{g_1}}\;e^{-\kperp^2/\avkperp_{g_1}} \, ,
	\label{Eq:Gauss-g1}\\
	h_{1}^{q} (x, \kperp^2) &=& h_{1}^{q} (x)\;
  	\frac{1}{\,\pi \avkperp_{h_1}}\;e^{-{\kperp^2}/{\avkperp_{h_1} }} \, ,
	\label{Eq:Gauss-h1}\\
	H_{1}^{\perp}(z,\pperp^2) &=&  H_{1}^{\perp (1)}(z) \;  
	\frac{2 z^2 m_h^2}{\pi \avpperp_{H_{1}^\perp}^2} \;
	e^{-\pperp^2/{\avpperp_{H_{1}^\perp}}}\, ,\\
	f_{1T}^{\perp q}(x,\kperp^2) &=&  f_{1T}^{\perp (1) q}(x)   \;
	\frac{2 M^2}{\pi \avkperp_{f_{1T}^\perp}^2} \;
	e^{-\kperp^2/{\avkperp_{f_{1T}^\perp}}} 
	\label{Eq:Gauss-f1Tperp}\, ,\\
	h_{1}^{\perp q}(x,\kperp^2) &=&  h_{1}^{\perp (1) q}(x)\;
   	\frac{2 M^2}{\pi \avkperp_{h_{1}^\perp}^2}\;
 	e^{-\kperp^2/{\avkperp_{h_{1}^\perp}}}\,
	\label{Eq:Gauss-h1perp}\, ,\\
	h_{1T}^{\perp q}(x,\kperp^2) &=&  h_{1T}^{\perp (2) q}(x)\;
   	\frac{2 M^4}{\pi \avkperp_{h_{1T}^\perp}^3} \;
	e^{-\kperp^2/{\avkperp_{h_{1T}^\perp}}}
	\label{Eq:Gauss-h1Tperp}\, .
\ea\end{subequations}
The parametrizations of the basis functions and the Gaussian model
parameters are described in detail in App.~\ref{App:basis}.




%\newpage
%======= SECTION 5: TWIST-2 AND BASIS FUNCTIONS ======================
\section{Leading twist asymmetries and basis functions}
\label{Sec-5:twist-2+basis}
In this section we review how the basis functions describe available
SIDIS data. This is of importance to asses the reliability of the
predictions presented in the next sections.

\subsection{\boldmath Leading twist $F_{UU}$ and Gaussian Ansatz}
\label{Sec-5.1:FUU-basis}

As explained in Sec.~\ref{Sec-4.3:evaluation} the Gaussian Ansatz is chosen
not only because it considerably simplifies the calculations, but more 
importantly because it works phenomenologically with a good accuracy 
in many processes including SIDIS
\cite{Anselmino:2005nn,Collins:2005ie,D'Alesio:2007jt,Schweitzer:2010tt,
Signori:2013mda,Anselmino:2013lza}.

The Gaussian Ansatz for the unpolarized TMD and FF 
is given by Eqs.~(\ref{Eq:Gauss-f1},~\ref{Eq:Gauss-D1}).
The parameters $\avkperp_{f_1}$ and $\avpperp_{D_1}$ can be 
assumed to be flavor- and $x$-- or $z$--independent, as present
data hardly allow us to constrain too many parameters, see
App.~\ref{App:basis-f1-D1} for a review. This assumption can be
relaxed, e.g.\ theoretical studies in chiral effective theories 
predict a strong flavor-dependence in the $\kperp$--behavior
of sea and valence quark TMDs \cite{Schweitzer:2012hh}.

%%%%%%%%%%%%%%%%%%%%%%%%%%%%%%%%%%%%%%%%%%%%%%%%%%%%%%%%%%%%%%%%%%%%%%%%%%%%%%%
\begin{figure}[b!]
\centering
\includegraphics[height=3.2cm]{\FigPath/Fig02a-JLab-ratio-pT-v5-NEW.pdf}  \quad
\includegraphics[height=3.2cm]{\FigPath/hermes_pip.pdf} \quad
\includegraphics[height=3.2cm]{\FigPath/compass_hp.pdf} 
\caption{\label{FUU-show-pT-dependence}
Examples of the 
	description of transverse momenta of hadrons in unpolarized SIDIS. 
Left panel: 
	$F_{UU}(P_{hT}^2)$ for $\pi^+$ production normalized with 
	respect to its value at zero transverse momentum at 
	$\la Q^2\ra=2.37\,{\rm GeV}^2$, $\la x\ra=0.24$, $\la z\ra=0.30$ 
	at JLab with 5.75 GeV beam \cite{Osipenko:2008aa}, cf.\  
	\cite{Schweitzer:2010tt}.
Middle panel: 
	HERMES multiplicity (\ref{Eq:multiplicity-HERMES}) at 
	$\la Q^2\ra=2.87\,{\rm GeV}^2$, $\la x\ra  =0.15$, $\la z\ra  =0.22$
	from \cite{Airapetian:2012ki}.
Right panel: 
	COMPASS multiplicity (\ref{Eq:multiplicity-COMPASS}) at 
	$\la Q^2\ra=20\,{\rm GeV}^2$, $\la x\ra  =0.15$, $\la z\ra  =0.2$
	from \cite{Aghasyan:2017ctw}.}
\end{figure}
%%%%%%%%%%%%%%%%%%%%%%%%%%%%%%%%%%%%%%%%%%%%%%%%%%%%%%%%%%%%%%%%%%%%%%%%%%%%%%%


The structure function $F_{UU}$ needed for our analysis reads
\begin{subequations}\ba
	F_{UU}(x,z,\Phperp) 
	&=& x \sum_q e_q^2\,f^q_1(x)\,D_1^q(z)\,{\cal G}(\Phperp)\,, 
	\label{Eq:FUU-Phperp}\\
	F_{UU}(x,z) %\equiv \int d^2 \Phperp \, F_{UU}(x,z,\Phperp) 
	&=& x \sum_q e_q^2\,f^q_1(x)\,D_1^q(z)  \, ,
	\label{Eq:FUU}
\ea\end{subequations}
where we introduce the notation ${\cal G}(\Phperp)$, which is defined as
\be\label{Eq:def-Gaussian-lambda}
	{\cal G}(\Phperp) = \frac{\exp(-\Phperp^2/\lambda)}{\pi\,\lambda}
	\, , \;\;\; 
	\lambda = z^2\,\la\kperp^2\ra_{f_1} + \la\pperp^2\ra_{D_1} \,,
\ee
with the understanding that the convenient abbreviation $\lambda$ is expressed 
in terms of the Gaussian widths of the {\it preceeding} TMD and FF. Notice 
that ${\cal G}(\Phperp)\equiv {\cal G}(x,z,\Phperp)$ and that in general
${\cal G}(\Phperp)$ appears under the flavor sum due to a possible 
flavor-dependence of the involved Gaussian widths.
The normalization $\int d^2\Phperp \,{\cal G}(\Phperp)=1$ 
correctly connects the structure function $F_{UU}(x,z,\Phperp)$ 
in (\ref{Eq:FUU-Phperp}) with its $\Phperp$--integrated counterpart
(\ref{Eq:FUU}). In our effective description this step is trivial. In 
QCD the connection of TMDs to PDFs is subtle \cite{Collins:2016hqq}.
Figure~\ref{FUU-show-pT-dependence} illustrates how the Gaussian Ansatz
describes selected SIDIS data.

Let us begin with JLab where, in the pre-12$\,$GeV era, electron beams 
from CEBAF with energies in the range $4.3$ to $5.7$ GeV were scattered 
off proton or deuterium targets in the typical kinematics 
$1 \,{\rm GeV}^2 < Q^2 < 4.5 \,{\rm GeV}^2$, $W > 2\,{\rm GeV}$, 
$0.1 < x < 0.6$, $y < 0.85$, $0.5<z<0.8$.
The left panel of Fig.~\ref{FUU-show-pT-dependence} shows basically
the SIDIS structure function $F_{UU}(P_{hT}^2)$ normalized with respect
to its value at zero transverse hadron momentum\footnote{Strictly 
	speaking in \cite{Osipenko:2008aa} data for the normalized 
	SIDIS cross section was presented. But these data correspond 
	to $F_{UU}(P_{hT}^2)/F_{UU}(0) \equiv
	F_{UU}(\la x\ra,\la z\ra,P_{hT}^2)/F_{UU}(\la x\ra,\la z\ra,0)$
	up to $1/Q^2$-suppressed terms.}
for $\pi^+$ production from a proton target measured in the CLAS experiment 
with a $5.75$ GeV beam for the kinematics $\la Q^2\ra=2.37\,{\rm GeV}^2$, 
$\la x\ra=0.24$, $\la z\ra=0.30$ \cite{Osipenko:2008aa}. Clearly, the 
Gaussian model works for the entire region of $P_{hT}$ covered in this 
experiment where structure function $F_{UU}$ falls down by 2 orders of 
magnitude \cite{Schweitzer:2010tt}.

Next we discuss a representative plot from the HERMES experiment 
where pions or kaons were measured in the scattering of 27.6 GeV 
positrons from the HERA polarized positron storage ring at DESY 
off proton and deuteron targets in the SIDIS kinematics 
$Q^2 > 1 \,{\rm GeV}^2$, $W > 2\,{\rm GeV}$, 
$0.023 < x < 0.4$, $y < 0.85$, $0.2<z<0.7$. 
The middle panel of Fig.~\ref{FUU-show-pT-dependence} displays the 
HERMES multiplicity~\cite{Airapetian:2012ki}
\be\label{Eq:multiplicity-HERMES}
	M_n^h(x,z,\Phperp) \equiv 
	\frac{d\sigma_{\rm SIDIS}(x,z,\Phperp)/dx\,dz\,d\Phperp}
	{d\sigma_{\rm DIS}(x,z)/dx} =
	2 \pi \Phperp \frac{F_{UU}(x,z,\Phperp) }{ x \sum_q e_q^2\,f^q_1(x)}
\ee
at $\la Q^2\ra=2.87\,{\rm GeV}^2$, $\la x\ra=0.15$, $\la z\ra=0.22$ 
for $\pi^+$ production on the proton target \cite{Airapetian:2012ki}. 

Finally we show also a representative plot from the COMPASS experiment 
where charged pions, kaons, or hadrons were measured with 160 GeV 
longitudinally polarized muons scattered off proton and deuteron 
targets in the typical SIDIS kinematics 
$Q^2 > 1 \,{\rm GeV}^2$, $W > 5\,{\rm GeV}$, 
$0.003 < x < 0.7$, $0.1<y < 0.9$, $0.2<z<1$. 
The right panel of
Fig.~\ref{FUU-show-pT-dependence} shows the COMPASS multiplicity 
\cite{Aghasyan:2017ctw}
\be\label{Eq:multiplicity-COMPASS}
	n^h(x,z,\Phperp^2)  \equiv 
	\frac{d\sigma_{\rm SIDIS}(x,z,\Phperp^2)/dx\,dz\,d\Phperp^2}
	{d\sigma_{\rm DIS}(x,z)/dx} =
	\pi \frac{F_{UU}(x,z,\Phperp^2) }{ x \sum_q e_q^2\,f^q_1(x)}\;
\ee
at $\la Q^2\ra=20\,{\rm GeV}^2$, $\la x\ra  =0.15$, $\la z\ra  =0.2$ 
for $h^+$ production on the deuterium target \cite{Aghasyan:2017ctw}.

To streamline the presentation we refer to the comprehensive Appendix 
on the used parametrizations (App.~\ref{App:basis}), and for 
technical details on the Gaussian Ansatz (App.~\ref{App:factor}).

The description of the HERMES and COMPASS multiplicities in
Fig.~\ref{FUU-show-pT-dependence} is good and sufficient for our
purposes, but it is not perfect. The descriptions of the COMPASS 
data in the region of small $\Phperp^2$ and that of the HERMES data 
for $\Phperp \gtrsim 0.3\,{\rm GeV}$ are not ideal.
However, notice that in our description we use the Gaussian widths 
as fitted and employed in the original extractions of the TMDs. These 
values were not optimised to fit the HERMES or COMPASS multiplicities.
Keeping this in mind, the description in Fig.~\ref{FUU-show-pT-dependence}
can be considered as satisfactory. We also remark that we do not take 
into account $\kperp$-broadening effects between HERMES and 
COMPASS energies \gs{and that the HERMES data actually represent 
multiplicities integrated (separately for numerator and denominator) 
over the kinematic ranges of each bin while the curve is plotted for 
a fixed set of kinematics.} Through dedicated fits to the HERMES, COMPASS
(and other) data and consideration of $\kperp$-evolution effects
it is possible to obtain a better description than in
Fig.~\ref{FUU-show-pT-dependence}, see \cite{Bacchetta:2017gcc}.

\newpage
%========= SECTION 6: TESTING APPROACH WITH A_LL =====================
\subsection{\boldmath Leading twist $A_{LL}$ and first test of Gaussian Ansatz
	in polarized scattering}
\label{Sec-5.2:FLL-basis}

The Gaussian Ansatz is useful in unpolarized case 
\cite{Anselmino:2005nn,Collins:2005ie,D'Alesio:2007jt,Schweitzer:2010tt,
Signori:2013mda,Anselmino:2013lza}, but nothing is known about its 
applicability to spin asymetries. The JLab data \cite{Avakian:2010ae} 
on $A_{LL}(\Phperp)$ put us in the position to conduct a first ``test'' 
for polarized partons. We assume Gaussian form for $g_{1}^{a}(x,\kperp^2)$, 
Eq.~(\ref{Eq:Gauss-g1}), and use lattice QCD results \cite{Hagler:2009mb} 
to estimate the width $\avkperp_{g_{1}}$, see App.~\ref{App:basis-g1}.
With $\lambda=z^2\la\kperp^2\ra_{g_1}+\la\pperp^2\ra_{D_1}$ implicit 
in ${\cal G}(\Phperp)$, the structure function $F_{LL}$  reads
\begin{subequations}\ba
	F_{LL}(x,z,\Phperp) 
	&=& x \sum_q e_q^2\,g^q_1(x)\,D_1^q(z)\,{\cal G}(\Phperp)\,, 
	\label{Eq:FLL-Phperp}\\
	F_{LL}(x,z)  
	&=& x \sum_q e_q^2\,g^q_1(x)\,D_1^q(z)  \, .
	\label{Eq:FLL}
\ea\end{subequations}

%%%%%%%%%%%%%%%%%%%%%%%%%%%%%%%%%%%%%%%%%%%%%%%%%%%%%%%%%%%%%%%%%%%%%%%%%%%%%%%
\begin{figure}[b]
\centering 
\includegraphics[width=0.3\textwidth]{\FigPath/all_jlab_pip}  
\includegraphics[width=0.3\textwidth]{\FigPath/all_jlab_pi0}  
\includegraphics[width=0.3\textwidth]{\FigPath/all_jlab_pim}  
\caption{\label{jlab_ALL} 
	$A_{LL,\la y\ra}$ asymmetry compared to JLab data \cite{Avakian:2010ae}
 	for $\pi^+$, $\pi^0$, $\pi^-$. The solid lines are 
	our results for the mean values of kinematical variable
	$\langle x \rangle = 0.25$, 
	$\langle z \rangle = 0.5$, $\langle Q^2 \rangle = 1.67$ GeV$^2$.
}
\end{figure}
%%%%%%%%%%%%%%%%%%%%%%%%%%%%%%%%%%%%%%%%%%%%%%%%%%%%%%%%%%%%%%%%%%%%%%%%%%%%%%%

The definition of the asymmetry in  the JLab experiment \cite{Avakian:2010ae} 
was
\be\label{Eq:ALLy}
	A_{LL,\la y\ra}(x,z,\Phperp) 
	= \la p_2 \,A_{LL}(x,z,\Phperp) \ra \, 
	= \frac{\la y (2-y) \; F_{LL}(x,z,\Phperp)\ra}
	{\la(1+(1-y)^2) \; F_{UU}(x,z,\Phperp)\ra} 
\ee 
where $p_2 = y (2-y)/(1+(1-y)^2)$ and averaging (separately in numerator 
and denominator) over the kinematics of \cite{Avakian:2010ae} is implied. 
We use the lattice data \cite{Hagler:2009mb} to
constrain the Gaussian width $\la\kperp^2\ra_{g_1}$ as described in 
App.~\ref{App:basis-g1}. All other ingredients in (\ref{Eq:ALLy}) are known 
and tested through other observables in Sec.~\ref{Sec-5.1:FUU-basis}.
Therefore the comparison of our results to JLab data \cite{Avakian:2010ae} 
shown in Fig.~\ref{jlab_ALL} provides several important tests.
First, the JLab data \cite{Avakian:2010ae} are compatible
with the Gaussian Ansatz within error bars. Second, the lattice results
in the way we use them in App.~\ref{App:basis-g1} give an appropriate 
description of the data.
	(Another important test was already presented in 
	\cite{Avakian:2010ae}: the $\Phperp$--integrated (``collinear'')
	asymmetry (\ref{Eq:FLL}) is compatible with data
	from other experiments and theoretical results obtained from
	parametrizations of $f_1^a(x)$, $g_1^a(x)$, $D_1^a(z)$. This 
	shows that even at the moderate beam energies of the pre-12GeV 
	era one was, to a good approximation, indeed probing DIS at JLab
	\cite{Avakian:2010ae}.)


Encouraged by these findings we will use lattice predictions from 
Ref.~\cite{Hagler:2009mb} below also for the Gaussian widths of 
$g_{1T}^{\perp(1)a}$ and $h_{1L}^{\perp(1)a}$.
Of course, at this point one could argue that the WW- and WW-type 
approximations (\ref{Eq:WW-approx-g1T},~\ref{Eq:WW-approx-h1L}) also
dictate that $g_{1T}^\perp$ and $h_{1L}^\perp$ have the same Gaussian
widths as $g_1$ and $h_1$. In fact, the lattice results for the 
respective widths are numerically similar, which can be interpreted as 
yet another argument in favor of the usefulness of the approximations. 
The practical predictions depend only weakly on the choice of parameters.


\subsection{\boldmath Leading twist $A_{UT}^{\sin(\phi_h-\phi_S)}$ Sivers asymmetry}
\label{Sec-5.3:Sivers-basis}

The $F_{UT}^{\sin(\phi_h-\phi_S)}$ structure function is related to the 
Sivers function~\cite{Sivers:1989cc}, which describes the distribution
of unpolarized quarks inside a transversely polarized proton, has so far 
received the widest attention, from both phenomenological and experimental 
points of view. 

The Sivers function $f_{1T}^\perp$ is related to initial and final state 
interactions of the struck quark and the rest of the nucleon and could 
not exist without the contribution of the orbital angular momentum of 
partons to the spin of the nucleon. As such it encodes the correlation 
between the partonic intrinsic motion and the transverse spin of the 
nucleon, and it generates a dipole deformation in momentum space.
The Sivers function has been extracted from SIDIS data
by several groups, with consistent results 
\cite{Anselmino:2010bs,Anselmino:2005ea,Anselmino:2005an,Collins:2005ie,Vogelsang:2005cs,Anselmino:2008sga,Bacchetta:2011gx,Echevarria:2014xaa}. 

The structure function $F_{UT}^{\sin(\phi_h-\phi_S)}$ reads
\begin{subequations}\ba
	F_{UT}^{\sin(\phi_h-\phi_S)}(x,z,\Phperp) 
	&=& - x\sum_q e_q^2\,f_{1T}^{\perp (1) q}(x)\,D_1^{q}(z)\; 
	b^{(1)}_{\rm B}\,\biggl(\frac{z \Phperp} {\lambda}\biggr)\,
	{ \cal G}(\Phperp ) \, , \label{FUTsiv-Gauss-Phperp}\\ 
	F_{UT}^{\sin(\phi_h-\phi_S)}(x,z,\la\Phperp\ra) 
	&=& - x\sum_q e_q^2\,f_{1T}^{\perp (1) q}(x)\,D_1^{q}(z)\;
	c^{(1)}_{\rm B}\,\biggl(\frac{z} {\lambda^{1/2}}\biggr)\,,
	\label{FUTsiv-Gauss}
\ea\end{subequations}
where $\lambda=z^2 \avkperp_{f_{1T}^\perp} + \avpperp_{D_1}$ and
$b^{(1)}_{\rm B}=2M_N$ and $c^{(1)}_{\rm B} = \sqrt{\pi}\,M_N$, 
see App.~\ref{App:convol-details} for details.

%%%%%%%%%%%%%%%%%%%%%%%%%%%%%%%%%%%%%%%%%%%%%%%%%%%%%%%%%%%%%%%%%%%%%%%%%%%%%%%
\begin{figure}[b!]
\centering
\includegraphics[width=0.45\textwidth]{\FigPath/AUTSivers_x.pdf}  \hspace{5mm}
\includegraphics[width=0.45\textwidth]{\FigPath/AUTSivers_COMPASS_x.pdf}
\caption{\label{aut_f1t_jlab} Sivers asymmetry 
	$A_{UT}^{\sin(\phi_h-\phi_S)}$ from proton target as function of $ x $ 
	based on the fit \cite{Anselmino:2011gs} in comparison to 
	(left panel) HERMES \cite{Airapetian:2009ae}
	and (right panel) COMPASS data \cite{Adolph:2012sp}.}
\end{figure}
%%%%%%%%%%%%%%%%%%%%%%%%%%%%%%%%%%%%%%%%%%%%%%%%%%%%%%%%%%%%%%%%%%%%%%%%%%%%%%%

Notice that integrating structure functions over $\Phperp$
is different from integrating the cross section over $\Phperp$
where azimuthal hadron modulations drop out. 
Only if the relevant weight is $\omega^{\{0\}}$ we obtain
``collinear structure functions:''  $F_{UU}(x,z)$, $F_{LL}(x,z)$ 
in Secs.~\ref{Sec-5.1:FUU-basis}, \ref{Sec-5.2:FLL-basis}
and below in Secs.~\ref{Sec-7.2:FLTcosphiS}, \ref{Sec-7.6:FUTsinphiS}.
In all other cases, despite integration over $\Phperp$, we end up 
always with true convoluted TMDs (here within Gaussian model).
We stress this important point by displaying the dependence of 
the structure functions on the mean hadron momenta, for instance
$F_{UT}^{\sin(\phi_h-\phi_S)}(x,z,\la\Phperp\ra) = 
\int d^2\Phperp F_{UT}^{\sin(\phi_h-\phi_S)}(x,z,\Phperp)$
in (\ref{FUTsiv-Gauss}).

The asymmetries $A_{UT}^{\sin(\phi_h-\phi_S)}= F_{UT}^{\sin(\phi_h-\phi_S)}/F_{UU}$  
obtained from the fit \cite{Anselmino:2011gs}
are plotted in Fig.~\ref{aut_f1t_jlab} as functions of $x$ in comparison 
to HERMES \cite{Airapetian:2009ae} and COMPASS \cite{Adolph:2012sp} data 
on respectively charged pion and hadron production from a proton target.
The $P_{hT}$--dependencies of the data are equally well described which
confirms that the Gaussian model works also in this case.

\newpage
\subsection{\boldmath Leading twist $A_{UT}^{\sin(\phi_h+\phi_S)}$ Collins asymmetry}
\label{Sec-5.4:Collins-basis}

The $F_{UT}^{\sin(\phi_h+\phi_S)}$ modulation of the SIDIS cross section is due 
to the convolution of the transversity distribution $h_1$ and the Collins 
FF $H_1^\perp$. Transversity can in principle be accessed also as a PDF in
Drell-Yan or in dihadron production 
\cite{Bacchetta:2002ux,Bacchetta:2003vn,Bacchetta:2011ip,Bacchetta:2012ty,
Radici:2015mwa,Radici:2018iag}.
It describes the distribution of transversely polarized quarks
in a transversely polarized nucleon, and is the only source of information 
on the tensor charge of the nucleon. The Collins FF $H_1^\perp$ decodes the 
fundamental correlation between the transverse spin of a fragmenting quark 
and the transverse momentum of the final produced hadron \cite{Collins:1992kk}.

There are many extractions of $h_1$ and $H_1^\perp$ from a 
combined fit of SIDIS and $e^+e^-$ data, for instance those of 
Refs.~\cite{Anselmino:2013vqa,Kang:2014zza,Anselmino:2015sxa}.
In this work we will use the extractions of $h_1$ and $H_1^\perp$ 
from Ref.~\cite{Anselmino:2013vqa}.

The structure function $F_{UT}^{\sin(\phi_h+\phi_S)}$ reads
\begin{subequations}\ba
	F_{UT}^{\sin(\phi_h+\phi_S)}(x,z,\Phperp) 
	&=& x\sum_q e_q^2\,h_{1}^{q}(x)\,H_1^{\perp(1)q}(z)\; 
	b^{(1)}_{\rm A}\,\biggl(\frac{z \Phperp} {\lambda}\biggr)\,
	{ \cal G}(\Phperp ) \, , \\
	F_{UT}^{\sin(\phi_h+\phi_S)}(x,z,\la\Phperp\ra) 
	&=& x\sum_q e_q^2\,h_{1}^{q}(x)\,H_1^{\perp(1)q}(z)\;  
	c^{(1)}_{\rm A}\,\biggl(\frac{z} {\lambda^{1/2}}\biggr)\,,
	\label{eq:asymmetry_aut_h1}
\ea\end{subequations}
where $\lambda=z^2 \avkperp_{h_1} + \avpperp_{H_1^\perp}$ and
$b^{(1)}_{\rm A}=2m_h$ and $c^{(1)}_{\rm A} = \sqrt{\pi}\,m_h$,
see App.~\ref{App:convol-details} for details.

The asymmetries $A_{UT}^{\sin(\phi_h+\phi_S)}= F_{UT}^{\sin(\phi_h+\phi_S)}/F_{UU}$  
are plotted in Fig.~\ref{aut_h1_jlab} as functions of $x$ in comparison 
to HERMES \cite{Airapetian:2010ds} and COMPASS \cite{Adolph:2014zba} 
data on charged pion production from proton targets.
We remark that the description of the $P_{hT}$--dependencies of 
this azimuthal spin asymmetry is equally satisfactory by the 
fit of Ref.~\cite{Anselmino:2013vqa}, which implies that the 
data are compatible with the Gaussian Ansatz also in this case. 

%%%%%%%%%%%%%%%%%%%%%%%%%%%%%%%%%%%%%%%%%%%%%%%%%%%%%%%%%%%%%%%%%%%%%%%%%%%%%%%
\begin{figure}[b!]
\centering
\includegraphics[width=0.45\textwidth]{\FigPath/AUTCollins_x.pdf}  
\includegraphics[width=0.45\textwidth]{\FigPath/AUTCollins_COMPASS_x.pdf}
\caption{\label{aut_h1_jlab}  Collins asymmetry 
	$A_{UT}^{\sin(\phi_h+\phi_S)}$ from proton target as function of $ x $ 
	based on the fit \cite{Anselmino:2013vqa} in comparison to 
	(left panel) HERMES \cite{Airapetian:2010ds} and 
	(right panel) COMPASS data \cite{Adolph:2014zba} (where
	$(-1)A_{UT}^{\sin(\phi_h+\phi_S)}$ is shown since for historical
	reasons the COMPASS collaboration likes to analyze their data
	with an opposite sign convention.)}
\end{figure}
%%%%%%%%%%%%%%%%%%%%%%%%%%%%%%%%%%%%%%%%%%%%%%%%%%%%%%%%%%%%%%%%%%%%%%%%%%%%%%%

\newpage
\subsection{\boldmath Leading twist $A_{UU}^{\cos(2\phi_h)}$ Boer-Mulders asymmetry}
\label{Sec-5.5:BM-basis}

The structure function $F_{UU}^{\cos(2\phi_h)}$ arises from a convolution of
the Collins fragmention function and the TMD $h_{1}^{\perp q}$ which describes
the distribution of transversely polarized partons inside an unpolarized
target. The expression of this structure function is given by
\begin{subequations}\ba
	F_{UU}^{\cos(2\phi_h)}(x,z,\Phperp) 
	&=& x \sum_q e_q^2\,h_{1}^{\perp (1) q}(x)\,H_1^{\perp(1) q}(z)\; 
	b^{(2)}_{\rm AB}\,\biggl(\frac{z \Phperp} {\lambda}\biggr)^{\!2}\,
	{ \cal G}(\Phperp)\, , \\
	F_{UU}^{\cos2\phi_h}(x,z,\la\Phperp\ra) 
	&=& x\sum_q e_q^2\,h_{1}^{\perp(1) q}(x)\,H_1^{\perp(1)q}(z)\;  
	c^{(2)}_{\rm AB}\,\biggl(\frac{z} {\lambda^{1/2}}\biggr)^{\!2}\,,
	\label{eq:asymmetry_auu_cos2phi}
\ea\end{subequations}
where $\lambda=z^2 \avkperp_{h_1^\perp} + \avpperp_{H_1^\perp}$ and
$b^{(2)}_{\rm AB}=4M_Nm_h$ and $c^{(2)}_{\rm AB} = 4M_Nm_h$,
see App.~\ref{App:convol-details}.

%%%%%%%%%%%%%%%%%%%%%%%%%%%%%%%%%%%%%%%%%%%%%%%%%%%%%%%%%%%%%%%%%%%%%%%%%%%%%%%
\begin{figure}[b!]
\centering
\includegraphics[width=0.45\textwidth]{\FigPath/AUUcos2phi_x.pdf} 
\includegraphics[width=0.45\textwidth]{\FigPath/AUUcos2phi_COMPASS_x.pdf}
\caption{\label{auu_cos2phi_jlab} The asymmetry 
	$A_{UU}^{\cos(2\phi_h)}$ from proton target as function of $ x $ 
	based on the fit \cite{Barone:2015ksa} in comparison to 
	(left panel) HERMES \cite{Airapetian:2012yg} and 
	(right panel) COMPASS data \cite{Adolph:2014pwc}.
	\TD{I thought we use now different parametrization, 
	not \cite{Barone:2015ksa}. please, check!}}
\end{figure}
%%%%%%%%%%%%%%%%%%%%%%%%%%%%%%%%%%%%%%%%%%%%%%%%%%%%%%%%%%%%%%%%%%%%%%%%%%%%%%%

Asymmetries $A_{UU}^{\cos(2\phi_h)}=F_{UU}^{\cos(2\phi_h)}/F_{UU}$ for 
HERMES \cite{Airapetian:2012yg} and COMPASS \cite{Adolph:2014pwc}
are plotted in Fig.~\ref{auu_cos2phi_jlab} where we only considered the
Boer-Mulders contribution to $A_{UU}^{\cos(2\phi_h)}$ which does not describe
the data accurately. In fact, it is suspected that this observable receives 
a significant contribution from the Cahn effect \cite{Cahn:1978se}, a 
term of higher twist character of the type $\la\Phperp^2\ra/Q^2$ which 
is not negligible in fixed target experiments \cite{Schweitzer:2010tt}. 
This contribution was estimated and corrected for in the phenomenological 
works \cite{Barone:2009hw,Barone:2010gk,Barone:2015ksa} which was of
importance to obtain a picture of the Boer-Mulders function undistorted 
from Cahn effect. The point is that this substantial twist-4 contamination 
can be estimated phenomenologically, even though there is no rigorous 
theoretical basis for the description of such power-suppressed terms. 
In this work we consistently neglect power-suppressed contributions of 
order $1/Q^2$, and do so also in Fig.~\ref{auu_cos2phi_jlab}. 
Nevertheless we of course use the parametrizations of 
\cite{Barone:2009hw,Barone:2010gk,Barone:2015ksa} offering
the best currently available parametrizations for $h_1^{\perp}$
which were corrected for Cahn effect as good as it is possible at
the current state of art. It is unknown whether other asymmetries
could be similarly effected by such type of power-corrections.
This is an important point to be kept in mind as the lesson 
from Fig.~\ref{auu_cos2phi_jlab} shows.


\newpage
\subsection{\boldmath Leading twist $A_{UT}^{\sin(3\phi_h-\phi_S)}$  asymmetry}
\label{Sec-5.6:pretzel-basis}

The pretzelosity TMD $h_{1T}^{\perp q}$ is the least known of the basis 
functions. It is of interest as it allows one to measure the deviation 
of the nucleon spin density from spherical shape \cite{Miller:2007ae},
is related to the only leading-twist SIDIS structure function where the 
small-$P_{hT}$ description in terms of TMDs and the large-$P_{hT}$ expansion 
in perturbative QCD mismatch \cite{Bacchetta:2008xw}, and is the only
TMD where a clear relation to quark orbital angular momentum
could be established (albeit only within quark models)
\cite{Avakian:2008dz,She:2009jq,Avakian:2010br,Lorce:2011kn}.

The structure function $F_{UT}^{\sin(3\phi_h-\phi_S)}$ reads
\begin{subequations}\ba
	F_{UT}^{\sin(3\phi_h-\phi_S)}(x,z,\Phperp)
	&=& x \sum_q e_q^2\,h_{1T}^{\perp (2) q}(x)\,H_1^{\perp(1) q}(z)\; 
	b^{(3)}\,\biggl(\frac{z \Phperp} {\lambda}\biggr)^{\!3}\,
	{ \cal G}(\Phperp) \, , \;\;\;\;
	\label{eq:asymmetry_aut_h1tp} \\
	F_{UT}^{\sin(3\phi_h-\phi_S)}(x,z,\la\Phperp\ra) 
	&=& x\sum_q e_q^2\,h_{1T}^{\perp (2) q}(x)\,H_1^{\perp(1)q}(z)\;  
	c^{(3)}\,\biggl(\frac{z} {\lambda^{1/2}}\biggr)^{3}\,,
\ea\end{subequations}
where $\lambda=z^2 \avkperp_{h_{1T}^\perp} + \avpperp_{H_1^\perp}$ and
$b^{(3)}=2M_N^2m_h$ and $c^{(3)}_{\rm  } = 3/2\sqrt{\pi} \,M_N^2m_h$,
see App.~\ref{App:convol-details}.
In Eq.~(\ref{eq:asymmetry_aut_h1tp}) we see that this structure 
function suffers a cubic suppression for small transverse hadron
momenta. This is the strongest $P_{hT}$-suppression of all SIDIS 
structure functions through leading and subleading twist, making
it challenging to measure.

%%%%%%%%%%%%%%%%%%%%%%%%%%%%%%%%%%%%%%%%%%%%%%%%%%%%%%%%%%%%%%%%%%%%%%%%%%%%%%%
\begin{figure}[b!]
\centering
%\includegraphics[width=0.45\textwidth]{\FigPath/AUTPretzelosity_x.pdf}  
%\includegraphics[width=0.45\textwidth]{\FigPath/AUTPretzelosity_PT.pdf}
\includegraphics[width=0.37\textwidth]{\FigPath/Screenshot-Fig-COMPASS-A_UT_Pr10-left-panel-up.png} \quad
\includegraphics[width=0.37\textwidth]{\FigPath/Screenshot-Fig-COMPASS-A_UT_Pr10-left-panel-down.png}
\caption{\label{aut_h1tp_jlab} $A_{UT}^{\sin(3 \phi_h - \phi_S)}$  
	as a function of $ x $ 
	\ps{from COMPASS \cite{Parsamyan:2010se} and the central value 
	of the best fit (whose 1-$\sigma$ uncertainty band is compatible 
	with zero) from \cite{Lefky:2014eia}. For comparison we show the
	model results \cite{Kotzinian:2008fe,Boffi:2009sh}.}
	\TD{\sf screenshot of 
	{\tt Fig-COMPASS-A\_UT\_Pr10.pdf} which was previously shown in 
	Fig.~\ref{autsinphi_jlab}. Bakur: please, can you produce such plot 
	with Alexei's ``best fit'' from \cite{Lefky:2014eia}?}}
\end{figure}
%%%%%%%%%%%%%%%%%%%%%%%%%%%%%%%%%%%%%%%%%%%%%%%%%%%%%%%%%%%%%%%%%%%%%%%%%%%%%%%


The preliminary COMPASS data \cite{Parsamyan:2010se} for 
$A_{UT}^{\sin(3 \phi_h - \phi_S)}=F_{UT}^{\sin(3 \phi_h - \phi_S)}/F_{UU}$ 
are plotted in Fig.~\ref{aut_h1tp_jlab}.
Clearly, the pretzelosity TMD is the least known of the basis TMDs.
Nevertheless it is of fundamental importance, as it provides one of the
basis functions in our approach. It is so difficult to access it 
experimentally, because its contribution to the SIDIS cross section
is proportional to $\Phperp^3$, the TMD approach requires us to
necessarily operate at $\Phperp\ll Q$, and so far only moderate
values of $Q$ could be achieved in the fixed target experiments.
A notable exception is COMPASS where the largest $x$--bins 
(where $Q^2$ is largest) bear the best hints on this TMD,
see Fig.~\ref{aut_h1tp_jlab}.
Future high luminosity data from JLab 12 are expected 
to significantly improve our knowledge of this TMD.


\subsection{Statistical and systematic uncertainties of basis functions}

Even the well known colinear functions $f_1^a(x)$, $g_1^a(x)$ have 
statistical uncertainties and systematic uncertainties (the latter
introduced by choosing a certain fit Ansatz which however can be 
avoided through neural network techniques \cite{Ball:2014uwa}).
These uncertainties as well as those of $D_1^a(z)$ can safely
be neglected for our purposes. 
For TMDs the situation is different. Already the transverse
momentum descriptions of $f_1^a(x,\kperp)$ and $D_1^a(z,\pperp)$ 
are associated with non-negligible statistical uncertainties
which are reviewed in App.~\ref{App:basis}, and with systematic
uncertainties which are very difficult to assess as they are
related to model bias (because of Gaussian model and its limitations).
The statistical and systematic uncertainties are significant
when we deal with the basis functions 
$f_{1T}^{\perp q}$, $h_{1}^{q}$, $H_{1}^{\perp q}$.
The least well-controlled uncertainties are associated with the 
Boer-Mulders function $h_1^{\perp q}$ and pretzelosity $h_{1T}^{\perp q}$.

In this work we are not interested in these uncertainties, which
future data will allow us to diminish --- even though in pratice 
they may be sizable. Rather in this work we are interested in
making projections based on the WW-type approximation. To avoid
cumbersome and difficult to interpret figures we will therefore
refrain from indicating the uncertainties associated with our
current knowledge of the basis functions. In the following
we will only display the estimated systematic uncertainty
associated with the WW-type approximations {\it assuming}
it works within a relative accuracy of $\pm40\%$.
We stress that this is only to simplify the presentation. The 
actual uncertainty of the predictions may be larger.



%======= SECTION 6: TWIST-2 AND WW-APROXIMATION ======================
\section{Leading twist asymmetries in WW-type approximation}
\label{Sec-6:twist-2-and-WW}

\begin{figure}[b!]
\centering
\includegraphics[width=0.45\textwidth]{\FigPath/g1t.pdf} \quad
\includegraphics[width=0.45\textwidth]{\FigPath/h1Lperp.pdf}
	\caption{\label{g1t_h1l_functions} 
	$g^{\perp (1) q}_{1T}(x)$ (left panel) and
	$h^{\perp (1) q}_{1L}(x)$ distributions (right panel) 
	for $u$-- and $d$--flavor 
	as predicted by the WW-ype approximations in
	Eqs.~(\ref{Eq:WW-approx-g1T},~\ref{Eq:WW-approx-h1L}).}
\end{figure}

Two out of the 8 leading--twist structure functions 
can be described in WW-type approximation thanks to 
Eqs.~(\ref{Eq:WW-approx-g1T},~\ref{Eq:WW-approx-h1L}) which 
relate $g_{1T}^{\perp (1) q}(x)$ and $h_{1L}^{\perp (1) q}(x)$ to the
basis functions $g_1^q(x)$ and $h_1^a(x)$, see Fig.~\ref{g1t_h1l_functions}.
These TMDs are sometimes referred to as ``gear worms:'' 
$g_{1T}^{\perp q}$ describes the distributions of longitudinally 
polarized quarks inside a transversely polarized nucleon,
$h_{1L}^{\perp q}$ the opposite constallation, namely transversely 
polarized quarks inside a longitudinally polarized nucleon.
It is interesting that both cases can be expressed in the
WW-type approximation in terms of the basis functions.
In this section we discuss the associated asymmetries.

\newpage
\subsection{\boldmath Leading twist $A_{LT}^{\cos(\phi_h-\phi_S)}$}
\label{Sec-6.1:FLTcosphi-phiS}

We assume for $g^{\perp}_{1T}$ the Gaussian Ansatz as shown 
in Eq.~(\ref{eq:g1t}) of App.~\ref{App-B:Gauss-Ansatz-non-basis-TMDs}, 
see also \cite{Kotzinian:2006dw}, and evaluate $g^{\perp (1) q}_{1T}(x)$ 
using Eq.~(\ref{Eq:WW-approx-g1T}) which yields the result 
shown in Fig.~\ref{g1t_h1l_functions}.
For our numerical estimates we use $\avkperp_{g_{1T}^\perp} = \avkperp_{g_{1}}$ 
which is supported by lattice results \cite{Hagler:2009mb}.

In the Gaussian Ansatz the structure function $F_{LT}^{\cos(\phi_h -\phi_S)}$ 
has the form
\begin{subequations}\ba
	F_{LT}^{\cos(\phi_h -\phi_S)}(x,z,\Phperp) 
	&=& x\sum_q e_q^2\,g_{1T}^{\perp (1) q}(x)\,D_1^{q}(z)\; 
	b^{(1)}_{\rm B}\,\biggl(\frac{z \Phperp} {\lambda}\biggr)\,
	{ \cal G}(\Phperp )  \\
	F_{LT}^{\cos(\phi_h-\phi_S)}(x,z,\la\Phperp\ra) 
	&=&  x\sum_q e_q^2\,g_{1T}^{\perp (1) q}(x)\,D_1^{q}(z)\;
	c^{(1)}_{\rm B}\,\biggl(\frac{z} {\lambda^{1/2}}\biggr)\,
	\label{eq:asymmetry_g1t_pt-II}
\ea\end{subequations}
where 
$\lambda  = z^2 \avkperp_{g_{1T}^\perp} + \avpperp_{D_1}$,
$b^{(1)}_{\rm B} = 2M_N$,
$c^{(1)}_{\rm B} = \sqrt{\pi}\,M_N$,
see App.~\ref{App:convol-details} for details.

%%%%%%%%%%%%%%%%%%%%%%%%%%%%%%%%%%%%%%%%%%%%%%%%%%%%%%%%%%%%%%%%%%%%%%%%%%%%%%%
\begin{figure}[b!]
\centering
%\includegraphics[width=0.3\textwidth]{\FigPath/ALT_COMPASS_x.pdf}  
\includegraphics[width=0.9\textwidth]{\FigPath/Fig-COMPASS-A_LT_Pr10.pdf}
	\caption{\label{g1t_jlab} 
	Asymmetries as functions of $x$ from scattering of 160 GeV
	longitudinally polarized muons off a transversely polarized 
	proton target \cite{Parsamyan:2015dfa}.
	Left panel: 
	leading-twist asymmetry $A_{LT}^{\cos(\phi_h-\phi_S)}$ 
	described in Sec.~\ref{Sec-6.1:FLTcosphi-phiS}.
	Middle panel: 
	subleading-twist $A_{LT}^{\cos\phi_S}$ 
	discussed in Sec.~\ref{Sec-7.2:FLTcosphiS}.
	Right panel: 
	subleading-twist $A_{LT}^{\cos(2\phi_h-\phi_S)}$
	on which we comment in Sec.~\ref{Sec-7.3:FLTcos2phi-phiS}.
	\TD{(Bakur: can you prepare separate plots for these asymmetries?)}}
\end{figure}
%%%%%%%%%%%%%%%%%%%%%%%%%%%%%%%%%%%%%%%%%%%%%%%%%%%%%%%%%%%%%%%%%%%%%%%%%%%%%%%

This asymmetry was measured at JLab and COMPASS 
\cite{Huang:2011bc,Parsamyan:2015dfa}. Left panel of Fig.~\ref{g1t_jlab} 
shows 2010 COMPASS data \cite{Parsamyan:2015dfa}. We approximate the 
charged hadrons ($90\,\%$ of which are $\pi^\pm$ at COMPASS) by 
charged pions, see App.~\ref{App:basis-f1-D1}.
Our results are shown as a shaded area which indicates a rough estimate
of the uncertainty of the WW-type approximation.
We observe that the WW-type approximation describes the data within 
their experimemental uncertainties.
For comparison results from the theoretical works 
\cite{Kotzinian:2006dw,Kotzinian:2008fe,Boffi:2009sh} are shown.
Our results are also compatible with the JLab data which was 
taken with a neutron ($^3$He) target \cite{Huang:2011bc} and has 
larger statistical uncertainty than COMPASS data. The middle and right 
panels of Fig.~\ref{g1t_jlab} will be discussed in 
Secs.~\ref{Sec-7.2:FLTcosphiS},~\ref{Sec-7.3:FLTcos2phi-phiS}.



\newpage

\subsection{\boldmath 
	Leading twist $A_{UL}^{\sin2\phi_h}$ Kotzinian-Mulders  asymmetry}
	\label{Sec-6.2:FULsin2phi}

We use the Gaussian form for the Kotzinian-Mulders function 
$h_{1L}^{\perp a}$, Eq.~(\ref{eq:h1l_final}) in
App.~\ref{App-B:Gauss-Ansatz-non-basis-TMDs}, with
$\avkperp_{h_{1L}^\perp} = \avkperp_{h_{1}}$ as
supported by lattice data \cite{Hagler:2009mb}.
>From (\ref{Eq:WW-approx-h1L}) we obtain the WW-type estimate
for $h_{1L}^{\perp(1) a}(x)$ shown in Fig.~\ref{g1t_h1l_functions}. 
The structure function $F_{UL}^{\sin(2\phi_h)}$ reads
\begin{subequations}\ba
	F_{UL}^{\sin(2\phi_h)}(x,z,\Phperp) 
	&=& 
	x \sum_q e_q^2\,h_{1L}^{\perp (1) q}(x)\,H_1^{\perp(1) q/h}(z)  
	\biggl(\frac{z\Phperp}{\lambda}\biggr)^{\!\!2} \;
	b^{(2)}_{\rm AB}\;{ \cal G}(\Phperp )\, , \\
	F_{UL}^{\sin(2\phi_h)}(x,z,\la\Phperp\ra) 
	&=& 
	x \sum_q e_q^2\,h_{1L}^{\perp (1) q}(x)\,H_1^{\perp(1) q/h}(z)  
	\biggl(\frac{z}{\lambda^{1/2}}\biggr)^{\!\!2} \;
	c^{(2)}_{\rm AB}\,,
\ea\end{subequations}
where $\lambda= z^2 \avkperp_{h_{1L}^\perp} + \avpperp_{H_1^\perp}$ and
$b^{(2)}_{\rm AB}=c^{(2)}_{\rm AB}=4M_Nm_h$,
see App.~\ref{App:convol-details} for details.

%%%%%%%%%%%%%%%%%%%%%%%%%%%%%%%%%%%%%%%%%%%%%%%%%%%%%%%%%%%%%%%%%%%%%%%%%%%%%%%
\begin{figure}[b!]
\centering
\includegraphics[height=3.1cm]{\FigPath/Fig-COMPASS-A_2_Pr0711_Zgt02.pdf} \quad
\includegraphics[height=3.0cm]{\FigPath/AULsin2phi_JLAB_x.pdf} 

\vspace{-5mm}
 
	\caption{\label{aul_jlab} 
	$A_{UL}^{\sin(2\phi_h)}$ as function of $x$ from HERMES
	\cite{Airapetian:1999tv} and COMPASS \cite{Parsamyan:2015dfa} 
	experiments (left panel) and at JLab 
	\cite{Jawalkar:2017ube} in comparison to predictions from
	WW-type approximation. Predictions from \cite{Avakian:2007mv}
	are shown for comparison.}
\end{figure}
%%%%%%%%%%%%%%%%%%%%%%%%%%%%%%%%%%%%%%%%%%%%%%%%%%%%%%%%%%%%%%%%%%%%%%%%%%%%%%%

The asymmetry $A_{UL}^{\sin(2\phi_h)}=F_{UL}^{\sin(2\phi_h)}/F_{UU}$  was 
studied at HERMES \cite{Airapetian:1999tv,Airapetian:2002mf}, COMPASS 
\cite{Parsamyan:2015dfa}, and JLab \cite{Avakian:2010ae,Jawalkar:2017ube}.
In Fig.~\ref{aul_jlab} proton data are shown for $\pi^\pm$ in the
HERMES experiment which were measured with the 27.6 GeV positron
beam from the HERA polarized positron storage ring at DESY for 
$1\,{\rm GeV}^2 < Q^2 < 15 \,{\rm GeV}^2$, $W > 2\,{\rm GeV}$,
$0.023 < x < 0.4$ and $y < 0.85$. 
The COMPASS data were taken with a 160 GeV muon beam and show 
the asymmetry for charged hadrons (in practice mainly pions).
Since $y$--dependent prefactors were included in the analyses 
(see Sec.~\ref{Sec-2.1:SIDIS+structure-functions}),
the HERMES data is adequately (``$D(y)$--'')rescaled.
The CLAS $\pi^0$ data in the right panel were measured using 6$\,$GeV 
longitudinally polarized electrons scattering off 
longitudinally polarized protons in a cryogenic $^{14}$NH$_3$ 
target in the kinematics $1.0\,{\rm GeV}^2 < Q^2 < 3.2\,{\rm GeV}^2$, 
$0.12 < x < 0.48$ and $0.4 < z < 0.7$ \cite{Jawalkar:2017ube}.

$A_{UL}^{\sin(2\phi_h)}$  can be expected to be smaller than 
$A_{LT}^{\cos(\phi_h -\phi_S)}$ discussed in previous section,
even though both are leading twist. This is because 
$F_{UL}^{\sin(2\phi_h)}$ is quadratic in the hadron transverse 
momenta $P_{hT}\ll Q$, while $F_{LT}^{\cos(\phi_h -\phi_S)}$ is linear. 
In addition, the former is proportional to the Collins function 
which is smaller than $D_1^q(z)$, and the WW-type approximation 
predicts the magnitude of $h_{1L}^{\perp(1)q}(x)$ to be about half 
of the size of $g_{1T}^{\perp(1)q}(x)$.
The data support these expectations. HERMES and JLab data are compatible
with zero for this asymmetry. So are COMPASS data except for the region
$x>0.1$ for negative hadrons, where the trend of the data provides a first 
encouraging confirmation for our results. The current data are compatible
with the WW-type approximation for $h_{1L}^{\perp(1)q}(x)$. 




\newpage

\subsection{Inequalities and a cross check}

We discussed WW-type approximations for the twist-2 TMDs
$g^{\perp q}_{1T}$ and $h^{\perp q}_{1L}$ 
in Secs.~\ref{Sec-6.1:FLTcosphi-phiS}, \ref{Sec-6.2:FULsin2phi}.
Before proceeding with twist-3 let us pause and revisit positivity 
bounds \cite{Bacchetta:1999kz}. % which offer unique cross checks. 

The Kotzinian-Mulders function $h^{\perp q}_{1L}$ in conjuction with 
the Boer-Mulders function, and the TMD $g^{\perp q}_{1T}$ in conjuction 
with the Sivers function obey the positivity bounds 
\cite{Bacchetta:1999kz}
\begin{subequations}\ba
	\frac{\kperp^2}{4M_N^2}\;
	\left((f_{1}^{q}(x,\kperp^2))^2 -(g_{1}^{q}(x,\kperp^2))^2\right)
	- (h^{\perp(1)q}_{1L}(x,\kperp^2))^2 
	- (h_{1}^{\perp(1)q}(x,\kperp^2))^2
	& \ge & 0\,, \quad \label{eq:positivity}\\
	\frac{\kperp^2}{4M_N^2}\;
	\left((f_{1}^{q}(x,\kperp^2))^2 -(g_{1}^{q}(x,\kperp^2))^2\right)
	- (f_{1T}^{\perp(1)q}(x,\kperp^2))^2
	- (g^{\perp(1)q}_{1T}(x,\kperp^2))^2 
	& \ge & 0\,, \quad \label{eq:positivity1}
\ea\end{subequations}
where $f^{(1)}(x,\kperp^2) \equiv \frac{\kperp^2}{2M_N^2} f(x,\kperp^2)$.
The inequalities provide a non-trivial test of our approach.
The inequalities (\ref{eq:positivity},~\ref{eq:positivity1}) 
must be strictly satisfied by the {\it exact} leading--order QCD 
expressions for the TMDs.  
(For PDFs it is known that positivity can be preserved in some 
schemes and violated in others. For TMDs not much is 
known about positivity conditions beyond leading order.)
However, here we do not deal with exact TMDs but (i) we invoked 
strong model assumptions (WW-type approximations for $g^{\perp q}_{1T}$ 
and $h^{\perp q}_{1L}$ and Gaussian Ansatz for all TMDs), and (ii) we deal 
with first extractions which have sizable uncertainties including 
poorly controlled systematic uncertainties.
Therefore, considering that we deal with approximations 
(WW-type, Gauss) and considering the current state of TMD-extractions, 
the inequalities (\ref{eq:positivity},~\ref{eq:positivity1}) constitute
a challenging test for the approach.

%%%%%%%%%%%%%%%%%%%%%%%%%%%%%%%%%%%
\begin{figure}[b!]
\centering
\includegraphics[width=0.45\textwidth]{\FigPath/h1lperp_positivity.pdf} \quad  
\includegraphics[width=0.45\textwidth]{\FigPath/g1t_positivity.pdf}  
	\caption{\label{h1l_pos} 
	The normalized inequalities for $g^{\perp(1)q}_{1T}(x)$ and
	$h^{\perp(1)q}_{1L}(x)$ vs $x$ which are obtained by integrating
	(\ref{eq:positivity}) and (\ref{eq:positivity1}) over $\kperp$,
	and normalizing with respect to the sum of the absolute
	values of the individual terms. The result must be positive and 
	smaller than unity {\it if} the WW-type approximations
	and the application of the Gaussian model are compatible
	with positivity, see text. Clearly, our approach respects 
	this test of the positivity conditions.}
\end{figure}
%%%%%%%%%%%%%%%%%%%%%%%%%%%%%%%%%%%%%%%%%%%%

To conduct a test we use the Gaussian Ans\"atze 
(\ref{Eq:Gauss-f1}, \ref{Eq:Gauss-h1}, \ref{Eq:Gauss-f1Tperp}, 
\ref{Eq:Gauss-h1perp}, \ref{eq:g1t}, \ref{eq:h1l_final}) for the
TMDs and integrate over $k_\perp$. The results are shown in 
Fig.~\ref{h1l_pos} where we plot the ``normalized inequalities'' 
defined as follows:
given an inequality $a-b-\dots \ge 0$, the normalized inequality 
is defined as: $0 \le (a-b-\dots)/(|a|+|b|+\dots) \le 1$.

Fig.~\ref{h1l_pos} shows that the results of our approach for the 
``normalized inequalities'' for both TMDs lie between between
zero and one, as it is dictated by positivity constraints.
This is an important consistency cross-check for our approach.

\newpage
%========= SECTION 8: TWIST-3 IN WW-TYPE APPROXIMATION ===============
\section{Subleading twist asymmetries in WW-type approximation}
\label{Sec-7:twist-3-and-WW}

WW-type approximations can be applied to all 8 subleading--twist asymmetries,
see Sec.~\ref{Sec-4.2:WW-twist-3}. In this section we discuss all of them,
starting with less complex cases and proceeding then to those structure 
functions whose description in WW-type approximation is more involved.

\subsection{\boldmath Subleading twist  $A_{LU}^{\sin\phi_h}$}
\label{Sec-7.1:FLU}

We start our discussion with the structure function $F_{LU}^{\sin\phi_h}$,
Eq.~(\ref{FLUsinphi}), containing 4 terms: 
2 term are proportional to pure twist-3 fragmentation functions 
$\tilde{G}^{\perp a}$ and $\tilde{E}^a$ and neglected; the other 2
terms are proportional to the twist-3 TMDs $e^a$ and $g^{\perp a}$ which
also turn out to be given in terms of pure twist-3 terms upon the 
inspection of Eqs.~(\ref{Eq:WW-type-1},~\ref{Eq:WW-type-gperp}).
Hence, after consequently applying the WW-type approximation we are left 
with no term. Our approximation predicts this structure function to be zero.

In this asymmetry we encounter the generic limitation of the 
WW-type approximation in most extreme form. As discussed in 
Sec.~\ref{Sec-3.8:limitations}, if we have a function
$=\la\bar{q}q\ra + \la\bar{q}gq\ra$ the necessary condition for 
the approximation to work is that $\la\bar{q}q\ra \neq 0$ and the 
sufficient condition would be $|\la\bar{q}q\ra|\gg|\la\bar{q}gq\ra|$.
Remarkably, none of the twist-3 TMDs or FFs entering this structure 
function satisfy even the necessary condition. In this situation we 
do not expect the WW-type approximation to be applicable. 

Indeed, data from JLab, HERMES and COMPASS show a clearly non-zero 
asymmetry $A_{LU}^{\sin\phi_h} = F_{LU}^{\sin\phi_h}/F_{UU}$ of the order of 2\,$\%$ 
\cite{Avakian:2003pk,Airapetian:2006rx,Gohn:2009zz,
Aghasyan:2011ha,Adolph:2014pwc,Gohn:2014zbz}. 
On the one hand this is interesting: this observable provides a unique 
opportunity to learn about the physics of $\bar{q}gq$-terms. On the other
hand, this observable is beyond the applicability of the WW-type approximation
and we cannot make a non-trivial prediction for this asymmetry.

With the numerator of the asymmetry proportional to $\bar{q}gq$--terms
and the denominator given in terms of $\bar{q} q$--terms, one could be 
tempted to interpret this observation as
\be\label{Eq:ALU-small}
    	A_{LU}^{\sin\phi_h}
	\;\; \propto \;\;\frac{\la\bar{q}gq\ra}{\la\bar{q}q\ra}
    	\;\;\stackrel{\rm exp}{\sim} \;\; 
	{\cal O}(2\,\%)
    	\;.
\ee
Thus, in some sense the observed effect hints at the smallness of the 
involved $\bar{q}gq$--terms. While in principle a correct observation,
one should keep in mind several reservations. First,
the experimental result (\ref{Eq:ALU-small})
contains kinematic prefactors which help to reduce the value. 
Second, the denominator contains $f_1^a$ and $D_1^a$ which are the
largest TMD and FF because of positivity constraints. Third, the 
numerator is a sum of 4 terms, so its overall smallness could result 
from cancellation of different terms, rather than indicating that 
each single $\bar{q}gq$--term is small. Fourth, some asymmetries 
predicted to be non-zero in WW-approximation are not much larger and 
in some cases even smaller than the result in (\ref{Eq:ALU-small}).

To conclude, the WW-type approximation predicts $A_{LU}^{\sin\phi}\approx 0$
in contradiction with experiment. The size of the observed effect seems
in line with the WW-type approximation, as
$A_{LU}^{\sin\phi_h}\sim\la\bar{q}gq\ra/\la\bar{q}q\ra \sim {\cal O}(2\,\%)$
is not large, although this interpretation has reservations.
$F_{LU}^{\sin\phi}$ is the only twist-3 SIDIS structure function not 
``contaminated'' by leading twist. Attempts to describe this observable
and relevant model studies have been reported 
\cite{Efremov:2002qh,
Lorce:2014hxa,
Schweitzer:2003uy,Ohnishi:2003mf,Cebulla:2007ej,
Efremov:2002ut,Afanasev:2003ze,Yuan:2003gu,Gamberg:2003pz,Metz:2004je,Afanasev:2006gw,Mao:2012dk,Mao:2014dva,Courtoy:2014xea,Yang:2016mxl}.
But more phenomenological work and dedicated studies on the basis of models 
of $\bar{q}gq$ terms are needed to fully understand this asymmetry.



\newpage

\subsection{\boldmath Subleading twist  $A_{LT}^{\cos\phi_S}$}
\label{Sec-7.2:FLTcosphiS}

In  WW-type approximation the structure function
$F_{LT}^{\cos\phi_S}$ arises from $g_T^a(x,\kperp)$ and $D_1^a(z,\pperp)$ 
whose collinear counterparts are more or less known, see
Secs.~\ref{Sec-3.4:WW-classic-experiment} and \ref{Sec-5.1:FUU-basis}.
We assume the Gaussian Ansatz for $g_T^a(x,\kperp)$ shown in
Eq.~(\ref{eq:gtnew}) of App.~\ref{App-B:Gauss-Ansatz-non-basis-TMDs} 
with ${\avkperp_{g_T}}={\avkperp_{g_1}}$. We then determine $g^{q}_{T}(x)$ 
according to Eq.~(\ref{Eq:WW-original1}) which is a well-tested
approximation in DIS, see Sec.~\ref{Sec-3.4:WW-classic-experiment}.
In this way we obtain for $F_{LT}^{\cos\phi_S}$ the result
\begin{subequations}\ba
	F_{LT}^{\cos\phi_S}(x,z,\Phperp) 
	&=& -\frac{2M}{Q}\; x^2 \sum_q e_q^2\,g_T^q(x)\,D_1^q(z)\;   
	{ \cal G}(\Phperp)\, , \label{Eq:FLTcosphiS-Gauss}\\
	F_{LT}^{\cos\phi_S}(x,z)
	&=& -\frac{2M}{Q}\; x^2 \sum_q e_q^2\, g_T^q(x)\,D_1^q(z)\, ,
	\label{Eq:FLTcosphiS-Gauss-b}
\ea\end{subequations}
with the width $\lambda= z^2 \avkperp_{g_T} + \avpperp_{D_1}$
in the Gaussian ${\cal G}(\Phperp)$. 

Notice that we followed here the scheme explained in 
Sec.~\ref{Sec-4.4:evaluation}: first assume Gaussian Ansatz, then apply 
WW-type approximation. For some structure functions the order of 
these steps is not relevant, but here it is. It is instructive to 
discuss what the opposite order of these steps yields. 
One may first plug in the WW-type approximation (\ref{Eq:WW-type-gT}) 
in the convolution integral (\ref{Eq:WW-original1}) and then solve the
convolution integral with Gaussian Ansatz. The result is an
analytical expression which is bulkier than (\ref{Eq:FLTcosphiS-Gauss})  
though it yields a numerically similar result. But there are 2 critical
issues with that. First, the WW-type approximation (\ref{Eq:WW-type-gT}) 
relates $g^{q}_{T}(x,\kperp^2)$ to $g^{\perp(1)q}_{1T}(x,\kperp^2)$ which
(due to the weight $\kperp^2/(2M_N^2)$ in the (1)-moment) implies a kinematical 
zero in $\kperp$ as $g^{q}_{T}(x, 0) = 0$  which is not supported by model 
calculations \cite{Avakian:2010br}. Second, the more economic (because 
less bulky) expression in (\ref{Eq:FLTcosphiS-Gauss}) automatically 
yields (\ref{Eq:FLTcosphiS-Gauss-b}) which is the correct colinear
result for this SIDIS function in Eq.~(\ref{Eq:FLT-collinear})
in WW-type approximation. This technical remark confirms that
consistency of the scheme suggested in Sec.~\ref{Sec-4.4:evaluation}.

In the middle panel of Fig.~\ref{g1t_jlab} our predictions
for $A_{LT}^{\cos\phi_S}$ are shown in comparison to preliminary
COMPASS data from Ref.~\cite{Parsamyan:2015dfa}. The predicted
asymmetry is small and compatible with the COMPASS data within
error bars. More precise data are necessary to judge how
well the WW-type approximation works in this case. Such data
could come from JLab 12 experiments, see Fig.~\ref{altcosphi_jlab}.

%%%%%%%%%%%%%%%%%%%%%%%%%%%%%%%%%%%%%%%%%%%%%%%%%%%%%%%%%%%%%%%%%%%%%%%%%%%%%%%
\begin{figure}[b!]
\centering
\includegraphics[width=0.37\textwidth]{\FigPath/ALTcosPhi_COMPASS_x.pdf} 
\caption{\label{altcosphi_jlab} Predictions for the double--spin asymmetry
	$A_{LT}^{\cos\phi_S}$  as a function of $x$ from a proton target
	at JLab 12.
	\TD{Bakur: can you prepare a separate plot for this asymmetry 
	based on the middle panel of Fig.~\ref{g1t_jlab}?
	Then we could skip here the prediction.}}
\end{figure}
%%%%%%%%%%%%%%%%%%%%%%%%%%%%%%%%%%%%%%%%%%%%%%%%%%%%%%%%%%%%%%%%%%%%%%%%%%%%%%%


\newpage
\subsection{\boldmath Subleading twist  $A_{LT}^{\cos(2\phi_h - \phi_S)}$}
\label{Sec-7.3:FLTcos2phi-phiS}

In the WW-type approximation this asymmetry is expressed in terms of
$g_T^{\perp a}(x,\kperp)$ for which we assume a Gaussian Ansatz according to 
Eq.~(\ref{eq:gtperpnew}) in App.~\ref{App-B:Gauss-Ansatz-non-basis-TMDs},
and use the WW-type approximation (\ref{Eq:WW-type-gTperp}) as
\be
	xg_T^{\perp(2)q}(x) = \frac{\la\kperp^2\ra_{g_{1T}^\perp}}{M_N^2}\;
	g_{1T}^{\perp (1)q}(x)\,,
\ee
where we finally express $g_{1T}^{\perp (1)q}(x)$ in terms of $g_1^q(x)$ 
according to Eq.~(\ref{Eq:WW-approx-g1T}). For the Gaussian widths
we assume $\avkperp_{g_{T}^\perp}=\avkperp_{g_{1T}^\perp}=\avkperp_{g_1}$.
This yields for the structure function 
\begin{subequations}\ba
	F_{LT}^{\cos(2\phi_h - \phi_S)}(x,z,\Phperp) 
	&=& -\frac{2M}{Q} x \sum_q e_q^2\,x\,g_{T}^{\perp (2) q}(x)\,D_1^q(z)\; 
	b^{(2)}_{C}\,\biggl(\frac{z \Phperp} {\lambda}\biggr)^{\!2}\,
	{ \cal G}(\Phperp)\, , \\
	F_{LT}^{\cos(2\phi_h - \phi_S)}(x,z,\la\Phperp\ra) 
	&=& -\frac{2M}{Q} x \sum_q e_q^2\,x\,g_{T}^{\perp (2) q}(x)\,D_1^q(z)\;  
	c^{(2)}_{C}\,\biggl(\frac{z} {\lambda^{1/2}}\biggr)^{\!2}\,
	\label{eq:asymmetry_auu_cos2phi-phiS}
\ea\end{subequations}
where $\lambda=z^2 \avkperp_{g_{T}^\perp} + \avpperp_{D_1}$ and 
$b^{(2)}_{\rm C}=c^{(2)}_{\rm C} = M_N^2$, 
see App.~\ref{App:convol-details} for details. 

Our predictions for the asymmetry  
$A_{LT}^{\cos(2\phi_h -\phi_S)}=F_{LT}^{\cos(2\phi_h -\phi_S)}/F_{UU}$ as function 
of $x$ are displayed in the right panel of Fig.~\ref{g1t_jlab} in
comparison to preliminary COMPASS data from Ref.~\cite{Parsamyan:2015dfa}. 
The asymmetry is very small, so we show our predictions on a larger
scale in Fig.~\ref{altcos2phi_jlab}. At the current stage one may
conclude that the WW-type approximation for the asymmetry 
$A_{LT}^{\cos(2\phi_h -\phi_S)}$ is compatible with available data. In view 
of the smallness of the effect, cf.\  Fig.~\ref{altcos2phi_jlab}, 
it might be difficult to obtain more quantitative insights
in near future. 

\


%%%%%%%%%%%%%%%%%%%%%%%%%%%%%%%%%%%%%%%%%%%%%%%%%%%%%%%%%%%%%%%%%%%%%%%%%%%%%%%
\begin{figure}[ht]
\centering
\includegraphics[width=0.45\textwidth]{\FigPath/ALTcos2PhiminusPhiS_COMPASS_x.pdf} 
\caption{\label{altcos2phi_jlab} $A_{LT}^{\cos(2\phi_h - \phi_S)}$ 
	as a function of $ x $ from a proton target for the COMPASS
	kinematics (enlarged detail of right panel Fig.~\ref{g1t_jlab} 
	where also the preliminary COMPASS data \cite{Parsamyan:2015dfa}
	are shown). The smallness
	of this asymmetry has two natural reasons. First, being twist-3
	it is $M_N/Q$ suppressed. Second, it is proportional to
	$P_{hT}^2$ with $P_{hT}\ll Q$ in TMD approach.
	\TD{Bakur: can you prepare a separate plot for this asymmetry 
	based on the right panel of Fig.~\ref{g1t_jlab}?
	Then we could skip here the prediction.}}
\end{figure}
%%%%%%%%%%%%%%%%%%%%%%%%%%%%%%%%%%%%%%%%%%%%%%%%%%%%%%%%%%%%%%%%%%%%%%%%%%%%%%%

\newpage

\subsection{\boldmath   Subleading twist  $A_{LL}^{\cos\phi_h}$}
\label{Sec-7.3:FLLcosphi}

In WW-type approximation the only contribution to $F_{LL}^{\cos\phi_h}$
is due to $g_{L}^{\perp q}(x,k_\perp)$ which we assume to have a
Gaussian $k_\perp$--behavior according to Eq.~(\ref{eq:gLperp})
in App.~\ref{App-B:Gauss-Ansatz-non-basis-TMDs} with 
the Gaussian width $\avkperp_{g_{L}^\perp}=\avkperp_{g_1}$. 
The structure function $F_{UL}^{\cos\phi_h}$ reads
\begin{subequations}\ba
	F_{LL}^{\cos \phi_h}(x,z,\Phperp)
	&=& -\,\frac{2M}{Q}\;x\sum_q e_q^2\,x\,g_{L}^{\perp (1) q}(x)\,D_1^{q}(z)\; 
	b^{(1)}_{\rm B}\,\biggl(\frac{z \Phperp} {\lambda}\biggr)\,
	{ \cal G}(\Phperp ) \, , \\ 
	F_{LL}^{\cos\phi_h}(x,z,\la\Phperp\ra) 
	&=& -\,\frac{2M}{Q}\;x\sum_q e_q^2\,x\,g_{L}^{\perp (1) q}(x)\,D_1^{q}(z)\;
	c^{(1)}_{\rm B}\,\biggl(\frac{z} {\lambda^{1/2}}\biggr)\, , 
\ea\end{subequations}
where $\lambda=z^2 \avkperp_{g_{L}^\perp} + \avpperp_{D_1}$, $b^{(1)}_{\rm B}=2M_N$, 
$c^{(1)}_{\rm B} = \sqrt{\pi}\,M_N$, see App.~\ref{App:convol-details}. Finally 
we explore the WW-type approximation (\ref{Eq:WW-type-gLperp}) to relate 
$x\,g_L^{\perp(1) a}(x) = \frac{\la \kperp^2\ra_{g_1}}{2\,M_N^2}\,g_1^a(x)$.

The asymmetry $A_{LL}^{\cos \phi_h}=F_{LL}^{\cos \phi_h}/F_{UU}$ predicted by the
WW-type approximation in this case is small and compatible with presently 
available data, see Fig.~\ref{allcosphi_jlab}.
We remark that previously this asymmetry was studied in
\cite{Anselmino:2006yc} and more recently in \cite{Mao:2016hdi}.

%%%%%%%%%%%%%%%%%%%%%%%%%%%%%%%%%%%%%%%%%%%%%%%%%%%%%%%%%%%%%%%%%%%%%%%%%%%%%%%
\begin{figure}[h!]
\centering

\vspace{1cm}
\includegraphics[width=0.8\textwidth]{\FigPath/Fig-COMPASS-A_LL_Pr0711_Zgt02.pdf}
	\caption{\label{allcosphi_jlab} $A_{LL}^{\cos(\phi_h)}$  
	as a function of $ x $ from a proton target for the COMPASS
	kinematics. The theoretical curves obtained from the WW-type 
	approximation are compared to the preliminary COMPASS data 
	\cite{Parsamyan:2015dfa}. }
\end{figure}
%%%%%%%%%%%%%%%%%%%%

\newpage
\subsection{\boldmath Subleading twist $A_{UL}^{\sin\phi_h}$ }
\label{Sec-7.4:FULsinphi}

In the WW-type approximation the asymmetry is described by 
$h_L^q(x,\kperp)$ for which we assume the Gaussian Ansatz 
(\ref{eq:hLnew}) in App.~\ref{App-B:Gauss-Ansatz-non-basis-TMDs}
with $\avkperp_{h_L}=\avkperp_{h_1}$. We explore (\ref{Eq:WW-type-6}) 
to estimate $xh_L^q(x) = -2 h_{1L}^{\perp(1)q}(x)$ and express
$h_{1L}^{\perp(1)q}(x)$ through $h_1^a(x)$ according to 
(\ref{Eq:WW-approx-h1L}). 
This yield for the structure function the result
\begin{subequations}\ba
	F_{UL}^{\sin\phi_h}(x,z,\Phperp) 
	&=& \frac{2M}{Q}\,x\sum_q e_q^2\,x\,h_{L}^{q}(x)\,H_1^{\perp(1)q}(z)\; 
	b^{(1)}_{\rm A}\,\biggl(\frac{z \Phperp} {\lambda}\biggr)\,
	{ \cal G}(\Phperp ) \, , \\
	F_{UL}^{\sin\phi_h}(x,z,\la\Phperp\ra) 
	&=& \frac{2M}{Q}\,x\sum_q e_q^2\,x\,h_{L}^{q}(x)\,H_1^{\perp(1)q}(z)\;  
	c^{(1)}_{\rm A}\,\biggl(\frac{z} {\lambda^{1/2}}\biggr)\,
\ea\end{subequations}
where $\lambda=z^2 \avkperp_{h_L} + \avpperp_{H_1^\perp}$ and
$b^{(1)}_{\rm A}=2m_h$ and $c^{(1)}_{\rm A} = \sqrt{\pi}\,m_h$,
see App.~\ref{App:convol-details} for details. 

%%%%%%%%%%%%%%%%%%%%%%%%%%%%%%%%%%%%%%%%%%%%%%%%%%%%%%%%%%%%%%%%%%%%%%%%%%%%%%%
\begin{figure}[b]
\centering
\includegraphics[width=0.35\textwidth]{\FigPath/AULsinphi_HERMES_x.pdf}
\includegraphics[width=0.35\textwidth]{\FigPath/AULsinphi_JLAB_x.pdf} 
\includegraphics[width=0.7\textwidth]{\FigPath/Fig-COMPASS-A_UL1_Pr0711_Zgt02.pdf}
\caption{\label{aulsinphi_jlab} The asymmetry $A_{UL}^{\sin\phi_h}$ 
	as function of $x$ from WW-type approximation in comparison to
	proton target data from: 
	(upper panel left) HERMES \cite{Airapetian:2005jc},
	(upper panel right) JLab \cite{Jawalkar:2017ube}, and 
	(lower panel) preliminary COMPASS \cite{Parsamyan:2015dfa} data.
	\TD{Bakur: could you remove the ``rescaled HERMES data''?}
	\TD{Gunar sent $\pi^0$ data, email from Thu, July 12, 2018,
	file: sinphi\_UL\_pi0\_HERMES.txt. Alexei, please
	can you include these data?}}
\end{figure}
%%%%%%%%%%%%%%%%%%%%%%%%%%%%%%%%%%%%%%%%%%%%%%%%%%%%%%%%%%%%%%%%%%%%%%%%%%%%%%%

The asymmetries $A_{UL}^{\sin\phi_h}=F_{UL}^{\sin\phi_h}/F_{UU}$ are compared
to HERMES, JLab, and preliminary COMPASS data in Fig.~\ref{aulsinphi_jlab}.
The WW-type approximation reproduces the positive sign of the asymmetry
seen consistently at HERMES and COMPASS for $\pi^+$ and positive hadrons
but underestimates its magnitude. The results for negative pions and 
hadrons at HERMES and COMPASS are compatible. An important test for this
asymmetry is provided by neutral pions where in WW-type approximation
the contributions for Collins fragmentation function largely cancel.
The WW-type approximation is not able to explain the large effect
observed at JLab for $\pi^0$ in the large--$x$ region $0.2< x < 0.4$
in Fig.~\ref{aulsinphi_jlab}.

\newpage

\subsection{\boldmath Subleading twist  $A_{UT}^{\sin\phi_S}$}
\label{Sec-7.6:FUTsinphiS}

In the structure function $F_{UT}^{\sin\phi_S}$ several interesting features
occur not encountered before. Let us summarize the result and follow up 
with some comments.
Assuming Gaussian Ans\"atze (\ref{eq:hTperpnew},~\ref{eq:hTnew}) 
for the TMDs $h_T^{\perp q}(x,\kperp)$, $h_T^q(x,\kperp)$ 
in App.~\ref{App-B:Gauss-Ansatz-non-basis-TMDs}, we obtain 
\begin{subequations}\begin{alignat}{1}
	F_{UT}^{\sin\phi_S}(x,z,\Phperp) 
	&= \,\frac{2M}{Q}\;x\sum_q e_q^2\,
	h_1^{(1)q}(x)\,H_1^{\perp(1)q}(z)\; \frac{4z^2 m_h\,M_N}{\lambda} 
	\left(1-\frac{\Phperp^2}{\lambda}\right) {\cal G}(\Phperp) 
	\label{Eq:FUTsinphiS-final-PhT}\\
  	F_{UT}^{\sin\phi_S}(x,z) 
	&= 0 \, .	\label{Eq:FUTsinphiS-final}
\end{alignat}\end{subequations}
with $\lambda=z^2\la\kperp^2\ra_{h_T^\perp}+\la\pperp^2\ra_{H_1^\perp}$ and
$\la\kperp^2\ra_{h_T^\perp}=\la\kperp^2\ra_{h_T^{ }}=\la\kperp^2\ra_{h_1^{ }}$. 
We can then use (\ref{Eq:WW-type-7},~\ref{Eq:WW-type-8}) to relate 
in the WW-type approximation 
$-\,\frac12\,x (h_T^{(1)q}(x) - h_T^{\perp(1)q}(x))
= h_1^{(1)q}(x) = \frac{\la\kperp^2\ra_{h_1}}{2M_N^2}\,h_1^q(x)$.

Some comments are in order. The first interesting feature 
is that after applying the WW-type approximation more than 
a single term is left.  Out of the 6 terms contributing to this 
structure function in (\ref{FUTsinphiS}) after WW-type approximation 
initially the 3 terms shown in Eq.~(\ref{Eq:WW-type-FUTsinphiS}) 
remain one of which is proportional to the TMD $f_T^q(x,\kperp)$.

The second interesting feature not encountered before is 
associated with $f_T^q(x,\kperp)$. This T-odd TMD must 
satisfy the sum rule (\ref{Eq:sum-rules-T-odd}), see the
discussion in Sec.~\ref{Sec-3.8:limitations}. This could be
implemented in two ways: one could describe it with a superposition 
of Gaussians, see App.~\ref{App-B:comment-Todd-twist-3}.
At this point, however, we have no guidance from phenomenology or theory 
to fix additional parameters. Alternatively one could choose a pragmatic 
and simple solution, namely to neglect the contribution of the TMD
$f_T^q(x,\kperp)$.\footnote{\label{Footnote:fT-single-Gauss} Notice 
	that this would correspond to using a ``single Gaussian'' as 
	$f_T^q(x,\kperp) = f_T^q(x)\;
	\frac{\exp(-\kperp^2/\la\kperp^2\ra_{f_T}^{ })}
	{\pi\la\kperp^2\ra_{f_T}^{ }}$
	with the ``coeffiecient'' $f_T^q(x)=0$ as dictated 
	by the sum rule (\ref{Eq:sum-rules-T-odd}).}
The result shown in Eq.~(\ref{Eq:FUTsinphiS-final-PhT})
is obtained using this second ``pragmatic'' solution.

Interestingly, for the $x$-- and $z$--dependence of the
structure function $F_{UT}^{\sin\phi_S}(x,z)$ it plays no role
(footnote~\ref{Footnote:fT-single-Gauss} 
vs App.~\ref{App-B:comment-Todd-twist-3}) how we model $f_T^q(x,\kperp)$: 
the result is Eq.~(\ref{Eq:FUTsinphiS-final}) in either case due to the 
requirement of the sum rule (\ref{Eq:sum-rules-T-odd}).
Notice that most of the time we focus on the $x$--dependence of 
asymmetries since this allows us to more directly test the WW-type 
approximations for TMDs, as opposed to e.g.\ the $P_{hT}$--dependence 
where in addition we would test at the same time also the Gaussian Ansatz.

The third interesting feature not seen previously is the occurence of a 
term which drops out upon integrating the structure function over $\Phperp$, 
cf.\ Eq.~(\ref{Eq:FUTsinphiS-final-PhT}) vs.\ (\ref{Eq:FUTsinphiS-final}).
This is a property of the weight $\omega^{\{2\}}_{\rm B}$, see 
Eq.~(\ref{Eq:wi}) and App.~\ref{App:factor} (which appears also in 
$F_{LT}^{\cos\phi_S}$, Eq.~(\ref{FLTcosphiS}), where it drops
out in WW-type approximation). In principle, this property could  help to 
discriminate experimentally the terms associated with this weight.

The final result in Eq.~(\ref{Eq:FUTsinphiS-final}) is the
consistent result for the structure function $F_{UT}^{\sin\phi_S}(x,z)$ in 
WW-type approximation, see Eq.~(\ref{Eq:FUT-collinear}). Our prediction 
is therefore $A_{UT}^{\sin\phi_S}(x)=0$. \gs{For positive pions or hadrons
respectively preliminary HERMES \cite{Schnell:2010zza} and COMPASS 
\cite{Parsamyan:2015dfa} data are compatible with this prediction.
However, for negative pions or hadrons the signal is clearly non-zero
and thus inconsistent with the WW prediction,} see Fig.~\ref{autsinphi_jlab}. 


\newpage

%%%%%%%%%%%%%%%%%%%%%%%%%%%%%%%%%%%%%%%%%%%%%%%%%%%%%%%%%%%%%%%%%%%%%%%%%%%%%%%
\begin{figure}[b]
\centering
\includegraphics[width=0.6\textwidth]{\FigPath/Fig-COMPASS-A_UT_Pr10.pdf}

\caption{\label{autsinphi_jlab} 
%	Preliminary data for azimuthal transverse target single--spin 
%	asymmetries as functions of $x$ from the COMPASS experiment 
%	\cite{Parsamyan:2015dfa}.
%	Left panel: 
%	leading--twist $A_{UT}^{\sin(3\phi_h-\phi_S)}(x)$ which 
%	determines the pretzelosity TMD, one of our basis functions. 
%	This asymmetry was discussed in Sec.~\ref{Sec-5.6:pretzel-basis} 
%	and is included here for completeness. 
	Left panel: 
        \TD{remove and show in Fig.~\ref{aut_h1tp_jlab} instead.}
	Middle panel:
	subleading--twist $A_{UT}^{\sin\phi_S}(x)$ which is predicted to vanish
	in the WW-type approximation, see Sec.~\ref{Sec-7.6:FUTsinphiS}.
	Right panel: 
	leading--twist $A_{UT}^{\sin(2\phi_h-\phi_S)}(x)$ in comparison to the 
	WW-type prediction described in Sec.~\ref{Sec-7.8:FUTsin2phi-phiS}.
	\TD{Bakur: please, can you re-do these plots in separate figures?}}
\end{figure}
%%%%%%%%%%%%%%%%%%%%%%%%%%%%%%%%%%%%%%%%%%%%%%%%%%%%%%%%%%%%%%%%%%%%%%%%%%%%%%%


\subsection{\boldmath Subleading twist  $A_{UT}^{\sin(2\phi_h-\phi_S)}$ }
\label{Sec-7.8:FUTsin2phi-phiS}

Also in this asymmetry we end up more than one surviving 
contribution in our treatment. We assume Gaussian Ans\"atze for 
$f_T^{\perp q}(x,\kperp)$, $h_T^{\perp q}(x,\kperp)$, $h_T^q(x,\kperp)$ 
according to Eqs.~(\ref{eq:hTperpnew},~\ref{eq:hTnew},~\ref{eq:ftperpnew}) 
in App.~\ref{App-B:Gauss-Ansatz-non-basis-TMDs} with
$\la\kperp^2\ra_{h_T^\perp}=\la\kperp^2\ra_{h_T^{ }}=\la\kperp^2\ra_{h_{1T}^\perp}$
and $\la\kperp^2\ra_{f_T^\perp}=\la\kperp^2\ra_{f_{1T}^\perp}$ and obtain
\begin{subequations}\begin{alignat}{1}
	F_{UT}^{\sin(2\phi_h -\phi_S)}(x,z,\Phperp) \; 
	=
	\frac{2M}{Q}\;x\sum_q e_q^2\,\Biggl[
	x \, f_T^{\perp(2)q}(x) \; D_1(z) \;b^{(2)}_{\rm C}\;
	\left(\frac{z\Phperp}{\lambda}\right)^{\!\!2}\,{ \cal G}(\Phperp)&
	\nonumber\\
	+ \;
	\frac{x}{2}\left(h_T^{(1)q}(x)+h_T^{\perp(1)q}(x)\right)H_1^{\perp(1)q}(z)\,
	b^{(2)}_{\rm AB}
	\left(\frac{z\Phperp}{\lambda}\right)^{\!\!2}\,{\cal G}(\Phperp)&\Biggr]
	\label{Eq:FUTsin2phi-phiS-Gauss-PhT}\\
{ }
	F_{UT}^{\sin(2\phi_h -\phi_S)}(x,z,\la\Phperp\ra) \; 
	= \; 
	\frac{2M}{Q}\;x\sum_q e_q^2\,\Biggl[
	x \; f_T^{\perp(2)q}(x) \; D_1(z) \; c^{(2)}_{\rm C} \;
	\biggl(\frac{z}{\lambda^{1/2}}\biggr)^{\!\!2} &
	\nonumber\\
	+ \;
	\frac{x}{2}\left(h_T^{(1)q}(x)+h_T^{\perp(1)q}(x)\right)H_1^{\perp(1)q}(z)\,
	c^{(2)}_{\rm AB}
	\biggl(\frac{z}{\lambda^{1/2}}\biggr)^{\!\!2} & \Biggr] 
	\label{Eq:FUTsin2phi-phiS-Gauss}
\end{alignat}\end{subequations}
with respectively 
$\lambda=z^2\la\kperp^2\ra_{f_T^\perp}+\la\pperp^2\ra_{D_1}$ in the first, and  
$\lambda=z^2\la\kperp^2\ra_{h_T^\perp}+\la\pperp^2\ra_{H_1^\perp}$ in the second 
terms in 
Eqs.~(\ref{Eq:FUTsin2phi-phiS-Gauss-PhT},~\ref{Eq:FUTsin2phi-phiS-Gauss}).
The coefficients 
$b^{(2)}_i$ and $c^{(2)}_i$ are defined in App.~\ref{App:convol-details}.
In the next step we explore the WW-type approximations
(\ref{Eq:WW-type-7},~\ref{Eq:WW-type-8},~\ref{Eq:WW-type-fTperp}) to 
relate 
$x \, f_T^{\perp(2)q}(x) = 
\frac{\la\kperp^2\ra_{f_{1T}^\perp}}{M_N^2}\,f_{1T}^{\perp (1)q}(x)$ and
$-\,\frac12\,x \left(h_T^{(1)q}(x) + h_T^{\perp(1)q}(x)\right)
= h_{1T}^{\perp(2)q}(x)$.

The asymmetries $A_{UT}^{\sin (2 \phi_h-\phi_S)}=F_{UT}^{\sin (2 \phi_h-\phi_S)}/F_{UU}$  
are plotted in the right panel of Fig.~\ref{autsinphi_jlab} in comparison 
to preliminary COMPASS data \cite{Parsamyan:2015dfa}. The predicted
asymmetry is small and compatible with data which are consistent
with a zero effect within error bars.


\newpage
\subsection{\boldmath Subleading twist  $A_{UU}^{\cos\phi_h}$ }
\label{Sec-7.7:FUUcosphi}
 
Historically this was the earliest azimuthal SIDIS asymmetry to be
discussed in literature: the first prediction for this asymmetry from 
intrinsic $k_\perp$ was made in \cite{Cheng:1972sy,Cahn:1978se}, a 
first measurement
was reported in \cite{Aubert:1983cz}.\footnote{First hints 
	\cite{Dakin:1972db} of azimuthal modulations in SIDIS
	date back to the early 1970s, i.e., 10 years before 
	the CERN measurements, but (unfortunately) were 
	discarded by the authors.}
This structure function contains after the WW-type approximation initially
two contributions, proportional to $f^\perp(x,\pperp)$ and $h^{q}(x,\kperp)$. 
The latter is T-odd and obeys the sum rule (\ref{Eq:sum-rules-T-odd}). 
We treat $h^{q}(x,\kperp)$ exactly as $f_T^q(x,\kperp)$ in 
Sec.~\ref{Sec-7.6:FUTsinphiS}.
Using the Gaussian Ansatz for $f^\perp(x,\kperp)$ in Eq.~(\ref{eq:fperpnew})
of App.~\ref{App-B:Gauss-Ansatz-non-basis-TMDs} we obtain 
\begin{subequations}\begin{alignat}{1}
	F_{UU}^{\cos\phi_h}(x,z,\Phperp) 
	= \frac{2M}{Q}\;x\sum_q e_q^2 & \Biggl[
	- x\,f^{\perp(1)q}(x)\,D_1^q(z)\;b^{(1)}_{\rm B}\,
	  \biggl(\frac{z \Phperp} {\lambda}\biggr)\, { \cal G}(\Phperp ) 
	\Biggr] \label{Eq:XXXa}\\
	F_{UU}^{\cos\phi_h}(x,z,\la\Phperp\ra) 
	= \frac{2M}{Q}\;x\sum_q e_q^2 & \Biggl[
	-x\,f^{\perp(1)q}(x)\,D_1^q(z)\;c^{(1)}_{\rm B}\;
	  \biggl(\frac{z}{\lambda^{1/2}}\biggr)
	\Biggr] \label{Eq:XXXb}
\end{alignat}\end{subequations}
with $\lambda=z^2\la\kperp^2\ra_{f^\perp}+\la\pperp^2\ra_{D_1}$. The coefficients
$b^{(1)}_i$ and $c^{(1)}_i$ are defined in App.~\ref{App:convol-details}.
Note that Eq.~(\ref{Eq:XXXa}) is valid in the ``scheme'' of 
footnote~\ref{Footnote:fT-single-Gauss}, but Eq.~(\ref{Eq:XXXa})
holds independently how one implements the sum rule (\ref{Eq:sum-rules-T-odd})
(as in footnote~\ref{Footnote:fT-single-Gauss} or 
App.~\ref{App-B:comment-Todd-twist-3}).

For $f^{\perp(1)}(x)$ we explore Eq.~(\ref{Eq:WW-type-Cahn}) as
$xf^{\perp(1)q}(x) = \frac{\la\kperp^2\ra_{f_1}}{2M_N^2}\,f_{1}^q(x)$ and 
assume for its Gaussian width $\la\kperp^2\ra_{f^\perp}=\la\kperp^2\ra_{f_1}$.
The latter means the Gaussian factors of 
$F_{UU}^{\cos\phi_h}$ and $F_{UU}$ cancel out, i.e.\ at some point 
for $\Phperp\gtrsim1\,$GeV we would obtain from (\ref{Eq:XXXa}) 
an asymmetry $A_{UU}^{\cos\phi_h}=F_{UU}^{\cos\phi_h}/F_{UU}$ exceeding
100$\,\%$ and violating unitarity. This is of course an artifact of our 
approximations, and reminds us that Gaussian and WW-type approximations 
as well as the entire TMD formalism must be applied to small $\Phperp\ll Q$.

The asymmetries $A_{UU}^{\cos\phi_h}$ were measured in experiments at
EMC \cite{Aubert:1983cz}, JLab \cite{Osipenko:2008aa,Mkrtchyan:2007sr}, 
HERMES \cite{Airapetian:2012yg}, and COMPASS \cite{Adolph:2014pwc}.
In Fig.~\ref{auucosphi_jlab} we compare our predictions 
to the HERMES and COMPASS data. Clearly, the WW-type
approximation strongly overestimates the data.


%%%%%%%%%%%%%%%%%%%%%%%%%%%%%%%%%%%%%%%%%%%%%%%%%%%%%%%%%%%%%%%%%%%%%%%%%%%%%%%
\begin{figure}[ht]
\centering
\includegraphics[width=0.37\textwidth]{\FigPath/AUUcosphi_COMPASS_x.pdf} 
\caption{\label{auucosphi_jlab} 
	Left panel: asymmetry $A_{UU}^{\cos\phi_h}$ for positive and negative 
	hadrons at COMPASS for a proton target \cite{Adolph:2014pwc}. Right 
	panel: the same for $\pi^\pm$ from HERMES \cite{Airapetian:2012yg}.
	\TD{Gunar sent data, email from Thu, July 12, 2018,
	file: cosNphi\_UU\_HERMES.txt. Maybe we show HERMES $\pi^\pm$ from 
	proton or deuterium (but not both).
	\underline{SUGGESTIONS}: include labels ``COMPASS proton data'' and 
	``HERMES proton (or deuteron) data'' on the figures.
	Plot range on $y$-axis: -0.35 to 0.}}
\end{figure}


%======= SECTION 9: CONCLUSIONS ======================================
\newpage
\section{Conclusions}
\label{Sec-8:conclusions}

In this work a comprehensive and complete treatment of SIDIS
spin and azimuthal asymmetries was presented. The leading--twist
SIDIS structure functions for the production of unpolarized hadrons,
which were proven to factorize, are unambiguously expressed in terms 
of one of 8 twist-2 TMDs convoluted with one of 2 twist-2 FFs.
For the subleading--twist SIDIS structure functions the situation
is far more complex for two reasons. First, factorization is not 
proven and must be assumed. 
Second, each of the subleading--twist structure functions
receives 4 or 6 contributions from various TMDs and FFs one of
which is twist-2 and the other twist-3. Clearly, to make 
predictions for new experiments or interpret early data, an
organizing theoretical guideline is needed.

In this work we have explored the so--called WW--type approximation
as a candidate guideline for the description of SIDIS structure
functions. This approximation consists of a systematic neglect 
of $\bar{q}gq$-terms in the
correlators defining the TMDs and FFs. We have shown that in such
an approach all twist-2 and twist-3 structure functions can be
described in terms of 8 basis functions: 6 TMDs and 2 FFs
which are each twist-2. All other TMDs and FFs, assuming this
approximation, are either related to the basis functions or
vanish. 

To make this work self-contained we included a review of the 
available phenomenological information on the basis functions 
which is given in terms of 6 SIDIS structure functions. 
(Of course, one cannot extract 8 basis
functions from 6 observables: the extraction of 2 basis functions,
the unpolarized TMD and FF $f_1^q(x,\kperp)$ and $D_1^q(x,\pperp)$,
makes also use of other experiments, most notably Drell-Yan
and hadron production in $e^+e^-$ annihilations.)

The WW-type approximation for TMDs and FFs is inspired by the
observation that the classical WW-approximation for the twist-3
SIDIS structure function $g_2(x)$ (or PDF $g_T^q(x)$) works well. 
This was predicted in theoretical studies in the instanton model 
of the QCD vacuum, and confirmed by data and lattice QCD studies.
The classic WW-approximation for $g_2(x)$ works with a relative
accuracy of $\pm\,40\,\%$ or better. This is remarkable.
The instanton vacuum model predicts an analogous WW-approximation
for the chirally odd twist-3 PDF $h_L^q(x)$ to work similarly
well. This prediction remains to be tested in experiment.

In each case, $g_T^q(x)$ and $h_L^q(x)$, we deal with nucleon
matrix elements of different $\bar{q}gq$ correlators, which are 
assumed to be small. Can one generalize these approximations to
TMDs? This is a key question, which has been addressed in the
past in literature in selected cases. This work is the first
systematic investigation of this question.
As in the unintegrated correlators one deals with different
classes of operators, we prefer to speak of the
WW-type approximation to distinguish from the colinear case.

We have studied in detail all SIDIS structure function in 
this approximation. This includes a review of results from 
lattice QCD calculations, effective theories and models.
We found that from theoretical point of view the WW-type
approximations receive certain support, though there is 
less evidence than in the colinear case.
Most importantly, we have conducted systematic tests of
WW-approximations with available published or 
preliminary (and soon to be published) SIDIS data
from HERMES, COMPASS, and JLab.

We found the following results. The two leading--twist 
structure functions amenable to WW-type approximation 
	$F_{LT}^{\cos(\phi_h -\phi_S)}$ and 
	$F_{UL}^{\sin(2\phi_h)}$ 
are well--described (the former) or at least compatible 
(the latter) with the data in this approximation. For 
$F_{UL}^{\sin(2\phi_h)}$ more precise data are needed, 
but also in this case the trend is encouraging
especially thanks to the recent COMPASS data.
We have also shown that our approach satisfies positivity
inequalities which is a non--trivial consistency check 
considering the crude approximations (WW-type, Gaussian 
Ansatz for TMDs) in our approach.

At subleading twist the WW-type approximation for the
structure functions 
	$F_{LT}^{\cos\phi_S}$,
	$F_{LT}^{\cos(2\phi_h-\phi_S)}$,
	$F_{LL}^{\cos\phi_h}$,
	$F_{UT}^{\sin\phi_S}$,
	$F_{UT}^{\sin(2\phi_h-\phi_S)}$
is compatible with data, too. Some of these asymmetries are
predicted to be very small in the WW-type approximation,
sometimes smaller than a fraction or a percent. This is
compatible with the available data in the sense that the
data are consistent with zero within their statistical
accuracies. This cannot be considered a thorough evidence
for the applicability of the WW-type approximations, but
on the positive site we also observe no alarming hints
that the WW-type approximations fail in these cases.
Merely for $F_{UT}^{\sin\phi_S}$ in the region of large-$x$
there is an indication that the asymmetry could be 
non-zero while the WW-type approximation proposes
a much smaller effect. But more higher-statistic data 
are needed before we can draw definitive conclusions.

In the case of the three subleading--twist structure functions 
	$F_{UU}^{\cos\phi_h}$, 
	$F_{UL}^{\sin\phi_h}$,
	$F_{LU}^{\sin\phi_h}$
the situation is clearer and indicates that here the WW-type
approximations do not work.
Incidently, these asymmetries include: 	$F_{UU}^{\cos(2\phi_h)}$ which 
is the very first non--zero azimuthal asymmetry measured in unpolarized SIDIS,
$F_{UL}^{\sin\phi_h}$ which is the very first non--zero azimuthal target spin 
asymmetry measured in SIDIS, and $F_{LU}^{\sin\phi_h}$ which is the 
very first non--zero azimuthal beam spin asymmetry measured in SIDIS.
The WW-type prediction for $F_{UU}^{\cos(2\phi_h)}$ suggest a large asymmetry
which overshoots data by a factor of three. In the case of $F_{UL}^{\sin\phi_h}$
the WW-type approximation undershoots data by a factor of 2 or so.
Most interestingly, in the case of $F_{LU}^{\sin\phi_h}$ the WW-type
approximation predicts exactly a zero asymmetry, but in experiment
a small but clearly non-zero effect is seen.

The non--aplicability of the WW--type approximation in these three
cases should not be too surprizing. After all it is a crude method
to model TMDs and FFs and an uncontrollable ``expansion''
(in nuclear physics the concept of 2--body, 3--body, etc operators
is well-justified and an effective expansion can be conducted; in
the case of TMDs however it is less appropriate to speak about a
systematic expansion in terms of $\bar{q}q$, $\bar{q}gq$, etc 
correlators). It will be very interesting to learn whether e.g.\ 
in $F_{UL}^{\sin\phi_h}$ or $F_{LU}^{\sin\phi_h}$ a single $\bar{q}gq$--term 
is anomalously large, or whether it is a accumulative effect of 
several small terms $\bar{q}gq$--terms adding up to the observed asymmetry.

Among all SIDIS structure functions $F_{LU}^{\sin\phi_h}$ emerges as
a particularly interesing case: this asymmetry is due to $\bar{q}gq$
only with no contamination from $\bar{q}q$ terms.
So $F_{LU}^{\sin\phi_h}$ offers a unique view on the physics of $\bar{q}gq$ 
correlators which is worth--exploring for its own sake.

The results presented in this work are of importance for several reasons.
First, to the best of our knowledge it is the first complete study 
of all SIDIS structure functions up to twist-3 in a unique approach. 
Second, the results are of use for experiments prepared in the near-term 
(JLab 12) or proposed in the long-term (Electron Ion Collider),
and provide helpful input for Monte Carlo event generators 
\cite{Avakian:2015vha}.
Third, our predictions will help to interpret data.

It is important to remark that the generalized parton model 
approach of Ref.~\cite{Anselmino:2011ch} provides a description,
which is largely equivalent to ours. As the quality of the 
approximations can only be determined experimentally, future
high--precision data will shed new light and allow us to 
access new insights abour the quark-gluon structure
of the nucleon.


\section{Acknowledgments}
First and foremost, the authors would like to thank their families 
for patience and constant support throughout writing this paper. 
The authors would like to thank Werner Vogelsang for many useful 
discussions. This work was partially supported by the U.S.\ 
Department of Energy under Contract No.~DE-AC05-06OR23177 (A.P.) 
and within TMD Collaboration framework, and by the National 
Science Foundation under Contract No.\ PHY-1623454 (A.P.)
and Contract No.\ PHY-1406298 (P.S.).


 
\newpage
\appendix

\section{\boldmath The ``minimal basis" of TMDs and FFs}
\label{App:basis}

This Appendix describes the technical details of the parametrizations
used in this work.

\subsection{\boldmath Unpolarised functions $f_1^a(x,k_\perp^2)$ 
			and $D_1(z,P_\perp^2)$}
\label{App:basis-f1-D1}

In this work we use the leading-order parametrizations 
from \cite{Martin:2009iq} for the unpolarized PDF $f_1^a(x)$ and 
from \cite{deFlorian:2007aj} for the unpolarized FF $D_1^a(z)$.
If not otherwise stated the parametrizations are taken at the scale 
$Q^2=2.4\,{\rm GeV}^2$ typical for many currently available SIDIS data.
These parametrizations were used in \cite{Anselmino:2005nn} and other 
works whose extractions we adopt. 

To describe the transverse momentum dependence of $f_1^a(x,k_\perp^2)$ 
and $D_1(z,P_\perp^2)$ we use the Gaussian Ansatz in 
Eqs.~(\ref{Eq:Gauss-f1},~\ref{Eq:Gauss-D1}). All early 
\cite{Anselmino:2005nn,Collins:2005ie,D'Alesio:2007jt,Schweitzer:2010tt}
and some recent \cite{Signori:2013mda} analyses employed flavor and 
$x$-- or $z$--independent widths $\avkperp$ and $\avpperp$.
In the analysis \cite{Anselmino:2013lza} 
of HERMES multiplicities flavor-independence of the widths was 
assumed. On long run one may anticipate that new precision data will 
require to relax these assumptions. However, one may also expect that the 
Gaussian Ansatz will remain a useful {\it approximation} as long as one is 
interested in describing data on transverse hadron momenta $\Phperp \ll Q $. 

The parameters resulting from calculations or extractions are presented in
Table~\ref{Table:Gauss-paramaters}. 
As most extractions of TMDs that we will use are done with the choice of 
$\avkperp_{f_1} = 0.25$ GeV$^2$, $\avpperp_{D_1}= 0.2$ GeV$^2$, for our numerical estimates in this work  we will use these fixed widths.

Some comments are in order.
In \cite{Anselmino:2005nn} no attempt was made to assign an uncertainty of the 
best fit result. The uncertainty of the numbers from \cite{Schweitzer:2010tt}
includes only the statistical uncertainty, but no systematic uncertainty.
For comparison lattice results from \cite{Hagler:2009mb} are included
whose range indicates flavor-dependence 
(first number $u$--flavor,
second number $d$--flavor). 
Notice that this is the contribution of the flavor averaged
over contributions from the respective quarks and antiquarks.
Chiral theories predict significant differences in the $\kperp$--behavior
of sea and valence quarks \cite{Schweitzer:2012hh}.
We will comment more on the lattice results in the next section.
In view of the large (and partly underestimated) uncertainties and the 
fact that those parameters are anti-correlated the numbers from the 
different approaches quoted in Table~\ref{Table:Gauss-paramaters} 
can be considered to be in good agreement. 

\begin{table}[h!]
\centering
\begin{tabular}{|r|l|c|c|c|}
\hline
  & & & & \cr
  study \ \ \ \ 
	& $\;\;\la Q^2\ra$, $\la x\ra$, $\la z\ra$ 
	& $\avkperp_{f_1}$ 
  	& $\avpperp_{D_1}$ 
	& $\avkperp_{g_1}$ \cr
  	& {\footnotesize $[{\rm GeV}^2]$} 
  	& {\footnotesize $[{\rm GeV}^2]$} 
  	& {\footnotesize $[{\rm GeV}^2]$} 
  	& {\footnotesize $[{\rm GeV}^2]$} \cr
  & & & & \cr
\hline
  & & & & \cr
\ fit of \cite{Anselmino:2005nn} & 5.0, 0.1, 0.3
	& $\sim 0.25$ 
	& $\sim 0.2$ 
	& -- \cr
\ fit of \cite{Schweitzer:2010tt} & 2.5, 0.1, 0.4 
	& $0.38\pm0.06$ 
	& $0.16\pm0.01$ 
	& -- \cr
\ fit of \cite{Anselmino:2013lza} & 2.4, 0.1, 0.3 
	& $0.57\pm0.08$ 
	& $0.12\pm0.01$ 
	& -- \cr
\ fit of \cite{Signori:2013mda} & 2.4, 0.1, 0.5 
	& $\sim0.3$
	& $\sim0.18$ 
	& -- \cr
\ lattice \cite{Hagler:2009mb}  & 4.0, -- , --
	& 0.14--0.15   
	& -- 
	& 0.11-0.15 \cr
  & & & & \cr 
\hline
\end{tabular}
\caption{\label{Table:Gauss-paramaters}
  	Gaussian model parameters for $f_1^a(x,k_\perp)$, $D_1^a(z,P_\perp)$, 
 	$g_{1}^a(x,k_\perp)$ from phenomenological and lattice QCD studies.
  	The kinematics to which the phenomenological results and the
	renormalization scale of the lattice results are indicated.
	The range of lattice values indicates flavor dependence
        (first number refers to $u$--flavor, second number to $d$--flavor).}
\end{table}


\subsection{\boldmath Helicity distribution $g_1^a(x,k_\perp^2)$}
\label{App:basis-g1}

For the helicity PDF $g_1^a(x)=\int d^2k_\perp\,g_1^a(x,k_\perp^2)\equiv 
\int d^2k_\perp\,g_{1L}^a(x,k_\perp^2)$ we use in this work the leading-order 
parametrizations from \cite{Gluck:1998xa}.
If not otherwise stated the parametrizations are taken at the scale 
$Q^2=2.4\,{\rm GeV}^2$.

In lack of phenomenological information on the $k_\perp$--dependence of
$g_1^a(x,k_\perp^2)$ we explore lattice QCD results from \cite{Hagler:2009mb}
to constrain the Gaussian width in Eq.~(\ref{Eq:Gauss-g1}). 
On a lattice with pion and nucleon masses 
$m_\pi\approx 500\,{\rm MeV}$ and $M_N=1.291(23)\,{\rm GeV}$ 
and with an axial coupling constant $g_A^{(3)}= 1.209(36)$ reasonably
close to its physical value $1.2695(29)$ the following results were
obtained for the mean square transverse parton momenta \cite{Hagler:2009mb}.
For the unpolarized TMDs it was found
$\langle \kperp^2 \rangle_{f_1^u} = (0.3741\,{\rm GeV})^2$ and
$\langle \kperp^2 \rangle_{f_1^d} = (0.3839\,{\rm GeV})^2$.
For the helicity TMDs it was found
$\avkperp_{g_1^u} = (0.327\,{\rm GeV})^2$ and
$\avkperp_{g_1^d} = (0.385\,{\rm GeV})^2$. 
These values are quoted in Table~\ref{Table:Gauss-paramaters}.

The lattice values for $\langle \kperp^2 \rangle_{f_1}$ consistently 
underestimate the phenomenological numbers, see 
Table~\ref{Table:Gauss-paramaters}.
The exact reasons for that are unknown, but it is natural to think it
might be related to the fact that the lattice predictions \cite{Hagler:2009mb}
do not refer to TMDs entering in SIDIS (or Drell-Yan or other process) 
because a different gauge link was chosen, see Sec.~\ref{Sec-3.5:WW-lattice}. 
Still one may expect these results to bear considerable information 
on QCD dynamics.\footnote{
	The results \cite{Hagler:2009mb} refer also to pion masses above the 
	physical value. This caveat is presumably less critical and
	will be overcome as lattice QCD simulations are becoming feasible 
	at physical pion masses.}
To make use of this information we shall assume that the lattice results
\cite{Hagler:2009mb} provide robust predictions for the {\it ratios}
$\langle \kperp^2 \rangle_{g_1^u}/\langle \kperp^2 \rangle_{f_1^u}\approx 0.76$.
With the phenomenological value $\langle \kperp^2 \rangle_{f_1} = 0.25$ GeV$^2$ 
we then obtain the estimate for the width of the helicity TMD
$\langle \kperp^2 \rangle_{g_1} = 0.19$ GeV$^2$.
In our phenomenological study we use this value for $u$--quarks and for 
simplicity also for $d$--quarks. Even though the lattice values indicate 
an interesting flavor dependence, see Table~\ref{Table:Gauss-paramaters},
for a proton target this is a very good approximation due to $u$--quark 
dominance. When precision data for deuterium and especially for $^3$He
from JLab become available, it will be interesting to 
re-investigate this point in detail.

\subsection{\boldmath Sivers function $f_{1T}^{\perp q}(x,k_\perp)$}
\label{App:basis-f1Tperp}

Sivers distribution function was studied in Refs.~\cite{Efremov:2004tp,Anselmino:2008sga,Anselmino:2011gs,Anselmino:2011ch, Aybat:2011ge,Gamberg:2013kla,Bacchetta:2011gx,Anselmino:2012aa,Sun:2013dya,Echevarria:2014xaa}. 
We will use parametrizations from  Refs.~\cite{Anselmino:2008sga,Anselmino:2011gs,Anselmino:2011ch}:
\ba
 \avkperp_{f_{1T}^\perp} &\equiv& \frac{\avkperp M_1^2}{\avkperp + M_1^2} \\
f_{1T}^{\perp}(x, \kperp^2)& =& - \frac {M}{M_1} \sqrt{2e} \;
{\cal N}_q(x) \, f_{q/p} (x,Q) \, \frac{e^{-\kperp^2/{\avkperp_{f_{1T}^\perp}}}}{\pi \avkperp} ,
\label{fiTp}
\ea
%
where $M_1$ is a mass parameter, $M$ the proton mass and
%
\ba
{\cal N}_q(x)= N_q \, x^{\alpha} (1-x)^{\beta} \,
\frac{(\alpha + \beta)^{(\alpha +\beta)}}
{\alpha^{\alpha} \beta^{\beta}}
 \ea
The first moment of Sivers function is:
\ba
f_{1T}^{\perp (1) q}(x)  = -\frac{\sqrt{\frac{e}{2}} \ \avkperp M_1^3}{M (\avkperp + M_1^2)^2}  \ {\cal N}_q(x)  f_q(x, Q) = -\sqrt{\frac{e}{2}} \frac{1}{M M_1}  \frac{\avkperp_{f_{1T}^\perp}^2}{\avkperp}    \ {\cal N}_q(x)  f_q(x, Q)
\label{siversm} \ .
\ea

We can rewrite parametrizations of Sivers functions as

\ba
f_{1T}^{\perp q}(x,\kperp^2) =  f_{1T}^{\perp (1) q}(x)   \frac{2 M^2}{\pi \avkperp_{f_{1T}^\perp}^2} e^{-\kperp^2/{\avkperp_{f_{1T}^\perp}}}
\label{sivers_new} \ .
\ea

The fit the HERMES proton and COMPASS deuteron data from 
including only Sivers functions for $u$ and $d$ quarks was done in Ref.~\cite{Anselmino:2011gs},
corresponding to seven free parameters, and parameters are shown in Table~\ref{tab:a}.


\begin{table}
\centering
\begin{tabular}{ccc}
\hline
$N_u=0.40$ & $\alpha_u=0.35$ & $\beta_u=2.6$ \\
$N_d=-0.97$ & $\alpha_d=0.44$ & $\beta_d=0.90$\\
& $M_1^2=0.19\; \textrm{(GeV}^2$) &   \\
\hline
\end{tabular}
\caption{Best values of the fit of the Sivers functions. Table from Ref.~\cite{Anselmino:2011gs}}
\label{tab:a}
\end{table}



\subsection{\boldmath Transversity $h_{1}^{q}(x,k_\perp)$ and 
Collins function $H_{1}^{\perp q}(x,P_\perp)$}
\label{App:basis-h1-H1perp}

These functions were studied in  
Refs.~\cite{Anselmino:2007fs,Anselmino:2008jk,Anselmino:2013vqa,
Kang:2014zza,Kang:2015msa,Anselmino:2015sxa}.
The following shape was assumed for parametrizations in
Refs.~\cite{Anselmino:2007fs,Anselmino:2008jk,Anselmino:2013vqa}:
 \ba
h_{1}^{q} (x, \kperp^2) &=&h_{1}^{q} (x)  \frac{e^{-{\kperp^2}/{\avkperp_{h_1} }}}{\pi \avkperp_{h_1}} \; ,\label{tr-funct}\\
h_{1}^{q} (x) &=&\frac{1}{2} {\cal N}^{\T}_q(x)\,
[f_{1}(x)+g_1(x)]\; ,\\
H_{1 h/q}^{\perp}(z, \pperp^2) = \frac{z m_h}{2 \pperp} \Delta^N \! D_{h/q^\uparrow}(z,\pperp^2) &=&  \frac{z m_h}{M_C} e^{-p_{\perp}^2/M_C^2} \sqrt{2 e} H_{1 h/q}^{\perp}(z) \,\frac{e^{-\pperp^2/{\avpperp}}}{\pi \avpperp}\,,
\label{coll-funct}
 \ea
 with $m_h$ the mass of the produced hadron and
 \ba
 {\cal N}^{\T}_q(x)= N^{\T}_q
\,x^{\alpha} (1-x)^{\beta} \, \frac{(\alpha + \beta)^{(\alpha
+\beta)}} {\alpha^{\alpha} \beta^{\beta}}\; ,
\\
H_{1 h/q}^{\perp}(z) =  {\cal N}^{\C}_q(z) D_{h/q}(z)\, , \\
{\cal N}^{\C}_q(z)= N^{\C}_q \, z^{\gamma} (1-z)^{\delta} \,
\frac{(\gamma + \delta)^{(\gamma +\delta)}}
{\gamma^{\gamma} \delta^{\delta}}\, ,
 \ea
and $-1\le N^{\T}_q\le 1$, $-1 \le N^{\C}_q \le 1$. The helicity distributions $g_1(x)$ are taken
from Ref.~\cite{Gluck:2000dy}, parton distribution and fragmentation functions are the GRV98LO PDF set~\cite{Gluck:1998xa} and the
DSS fragmentation function set~\cite{deFlorian:2007aj}. Notice that with these choices both
the transversity and the Collins function automatically obey their
proper positivity bounds. Note that as in Ref.~\cite{Anselmino:2013vqa} we use two 
Collins fragmentation functions, {\it favored} and {\it disfavored} ones, see Ref.~\cite{Anselmino:2013vqa} for details on implementation, and corresponding parameters ${N}^{\C}_a$ are then  ${N}^{\C}_{fav}$ and ${N}^{\C}_{dis}$. For numerical estimates we use parameters extracted in Ref.~\cite{Anselmino:2013vqa}, see Table~\ref{fitpar}.

\begin{table}[h]
\centering
\renewcommand{\tabcolsep}{0.4pc} % enlarge column spacing
\renewcommand{\arraystretch}{1.2} % enlarge line spacing
\begin{tabular}{@{ }ll}
 \hline
 $N_{u}^{\T}$ = $0.46^{+0.20}_{-0.14}$ & $N_{d}^{\T}$ = $ -1.00^{+1.17}_{-0.00}$ \\
 $\alpha$ =  $1.11^{+0.89}_{-0.66}$ & $\beta$  = $3.64^{+5.80}_{-3.37}$ \\
 $\avkperp_{h_1} = 0.25$ (GeV$^2$) & \\
 \hline
 $N_{\rm fav}^{\C}$  = $0.49^{+0.20}_{-0.18}$ & $N_{\rm dis}^{\C}$  = 
 $-1.00^{+0.38}_{-0.00}$ \\
 $\gamma$  = $1.06^{+0.45}_{-0.32}$  & $\delta$   = $0.07^{+0.42}_{-0.07}$    \\
 $M^2_C = 1.50^{+2.00}_{-1.12}$ (GeV$^2$) & \\
 \hline
\end{tabular}
\caption{
Best values of the 9 free parameters fixing the $u$ and $d$ quark
transversity distribution functions and the favored and
disfavored Collins fragmentation functions. The table is from Ref.~\cite{Anselmino:2013vqa}.
\label{fitpar}}
\end{table}


According to Eq.~(\ref{eq:moments}) we obtain the following expression for the first moment of 
Collins fragmentation function: 
\ba
H_{1 h/q}^{\perp (1)}(z) = \frac{H_{1 h/q}^{\perp}(z) \sqrt{e/2}  \avpperp M_C^3}{z m_h  (M_C^2+\avpperp)^2}\; .
\ea 
We also define the following variable:
\ba
\avpperp_{H_1^\perp} = \frac{\avpperp M_C^2 }{\avpperp + M_C^2} .  
\ea

We can rewrite the parametrizations of Collins FF as

\ba
H_{1}^{\perp}(z,\pperp^2) =  H_{1}^{\perp (1)}(z)   \frac{2 z^2 m_h^2}{\pi \avpperp_{H_{1}^\perp}^2} e^{-\pperp^2/{\avpperp_{H_{1}^\perp}}}
\label{coll-funct_new} \, .
\ea

\subsection{\boldmath Boer-Mulders function $h_{1}^{\perp}(x,k_\perp)$} 
\label{App:basis-h1perp}

The Boer-Mulders function $h_{1}^{\perp}$~\cite{Boer:1997nt} measures 
the transverse polarization asymmetry of quarks inside an unpolarized 
nucleon. The Boer-Mulders functions were studied phenomenologically in 
Refs.~\cite{Barone:2009hw,Barone:2010gk,Barone:2015ksa}

Ref.~\cite{Barone:2010gk} used the parametrization in which Boer-Mulders function is proportional to the Sivers functions, such that:
\ba
\avkperp_{h_1^\perp} &=& \frac{\avkperp \, M^2_{BM}}{\avkperp + M^2_{BM}} \, , \\
h_{1}^{\perp}(x, \kperp^2) &= &
- \,\frac{M}{M_{BM}}  
\sqrt{2e}\; N_{BM}^q N_q (x)
\, f_{q/p} (x, Q)\frac{e^{-\kperp^2/\avkperp_{h_{1}^{\perp}}}}{\pi\avkperp},  
\label{BM-dist}
\ea
 where  
%
\ba
{\cal N}_q(x)= N_q \, x^{\alpha} (1-x)^{\beta} \,
\frac{(\alpha + \beta)^{(\alpha +\beta)}}
{\alpha^{\alpha} \beta^{\beta}}\; .
 \ea

The first moment of Boer-Mulders function is:
\ba
h_{1}^{\perp (1) q}(x)  = -\frac{\sqrt{\frac{e}{2}} \ \avkperp M_{BM}^3}{M (\avkperp + M_{BM}^2)^2}  \ {N}_q f_q(x, Q) = -\sqrt{\frac{e}{2}} \frac{1}{M M_{BM}}  \frac{\avkperp_{h_1^\perp}^2}{\avkperp}    \ {N}_q  f_q(x, Q)
\label{bm} \ .
\ea
 
We can rewrite parametrizations of Boer-Mulders functions as
\ba
h_{1}^{\perp q}(x,\kperp^2) =  h_{1}^{\perp (1) q}(x)   \frac{2 M^2}{\pi \avkperp_{h_{1}^\perp}^2} e^{-\kperp^2/{\avkperp_{h_{1}^\perp}}}\label{bm_new} \ .
\ea
 
 \begin{table}
\centering
\begin{tabular}{crl}
\hline
$N_{BM}^u=2.1\pm0.1$ & $N_{BM}^d=$&$-1.111\pm0.001$   \\
Fixed & parameters: &  \\
$N_u=0.35$ & $\alpha_u=0.73$ & $\beta_u=3.46$ \\
$N_d=-0.9$ & $\alpha_d=1.08$ & $\beta_d=3.46$\\
& $M_{BM}^2=$&$0.34\; \textrm{(GeV}^2$)    \\
\hline
\end{tabular}
\caption{Fitted parameters of Boer-Mulders quark distributions. Values are from Ref.~\cite{Barone:2009hw}.}
\label{fitparbm}
\end{table}
%
%\begin{table}[htb]
%\centering
%\begin{tabular}{l c l l c l}
%\hline
 %$N_{u}$ &=& $-0.49 \pm 0.15$ & $N_{d}$ &=& $-1 \pm 0.2$\\
% $M_{BM}^2$ &=& $0.1 \pm  0.2$&\multicolumn{3}{l}{(GeV$^2$)}\\ 
%\hline
%\end{tabular}
%\caption{Fitted parameters of Boer-Mulders quark distributions. Values are from Ref.~\cite{Barone:2015ksa}}
%\label{fitparbm}
%\end{table}
 
\subsection{\boldmath Pretzelosity distribution $h_{1T}^{\perp}(x,k_\perp)$}
\label{App:basis-h1Tperp}

Pretzelosity   distribution function   
$h_{1T}^{\perp}$~\cite{Lefky:2014eia} describes transversely polarized quarks 
inside a transversely polarized nucleon.
We use the following form of $h_{1T}^{\perp a}$ ~\cite{Lefky:2014eia}:
\ba
h_{1T}^{\perp a}(x,k_{\perp}^2) = \frac{M^2}{M_{TT}^2} e^{-\kperp^2/M_{TT}^2} h^{\perp a}_{1T}(x) \frac{e^{-{\kperp^2}/{\avkperp}}}{\pi \avkperp}=\frac{M^2}{M_T^2} h^{\perp a}_{1T}(x) \frac{e^{-{\kperp^2}/{\avkperp_{h_{1T}^\perp}}}}{\pi \avkperp}\,\;,
\label{eq:h1Tperp}
\ea
where
\ba
h^{\perp a}_{1T}(x) = e  \ {\cal N}^a(x) (f_{1}^{a}(x, Q) - g_{1}^{a}(x, Q)), \label{eq:hx_par}\\
{\cal N}^a(x) = N^{a} x^{\alpha} (1-x)^{\beta} \frac{(\alpha + \beta)^{\alpha + \beta}}{\alpha^{\alpha} \beta^{\beta}}\, ,  \\
\avkperp_{h_{1T}^\perp}  = \frac{\avkperp M_{TT}^2}{\avkperp + M_{TT}^2}\, ,
\ea
where ${N}^a$, $\alpha$, $\beta$, and $M_T$ are parameters fitted to data that can be found in Table~\ref{fitparI}.

  We use Eq.~(\ref{eq:moments}) to calculate the second moment of $ h_{1T}^{\perp a}(x,\kperp^2)$
of Eq.~(\ref{eq:h1Tperp}) and obtain:
\be
h_{1T}^{\perp (2) a}(x) =  \frac{h^{\perp a}_{1T}(x) \avkperp_{h_{1T}^\perp}^3}{2 M^2 M_{TT}^2 \avkperp} \, .
\ee

We can rewrite parametrization of pretzelosity   functions as

\ba
h_{1T}^{\perp q}(x,\kperp^2) =  h_{1T}^{\perp (2) q}(x)   \frac{2 M^4}{\pi \avkperp_{h_{1T}^\perp}^3} e^{-\kperp^2/{\avkperp_{h_{1T}^\perp}}}
\label{pretzelosity_new} \ .
\ea

%
\begin{table}[htb]
\centering
\begin{tabular}{l c l l c l}
\hline
$\alpha$ &=& $2.5\pm1.5$ & $\beta$ &=& $2$ fixed \\
 $N_{u}$ &=& $1 \pm 1.4$ & $N_{d}$ &=& $-1 \pm 1.3$\\
 $M_{TT}^2$ &=& $0.18 \pm  0.7$&\multicolumn{3}{l}{(GeV$^2$)}\\ 
\hline
\end{tabular}
\caption{Fitted parameters of the pretzelosity quark distributions. Table from Ref.~\cite{Lefky:2014eia}}
\label{fitparI}
\end{table}
%
 
\newpage
\section{Convolution integrals and expressions in Gaussian Ansatz}
\label{App:factor}

In this Appendix we explain the notation for convolution integrals
of TMDs and FFs and give the explicit results obtained assuming the
Gaussian Ansatz.

\subsection{Notation for convolution integrals \label{ApendixB1}}
 
Structure functions are expressed as convolutions of TMDs and FFs 
in the Bjorken limit at tree level. For reference we quote the 
convolution integrals in ``Amsterdam notation'' \cite{Bacchetta:2006tn}
\be
	{\cal C}\bigl[ w\slim f\slim D \bigr]
	=  x \, \sum_a e_a^2 \int d^2 {\bm p}_T \,  d^2 {\bm k}_T
	\, \delta^{(2)}\bigl({\bm p}_T - {\bm k}_T - {\bm P}_{h \perp}/z \bigr)
	\,w({\bm p}_T,{\bm k}_T)\,
	f^a( x ,p_T^2)\,D^a(z,z^2 k_T^2) , \label{eq:convolution-Amsterdam}
\ee
where all transverse momenta refer to the virtual photon-proton 
center-of-mass frame and $\bfhp  ={\bm P}_{h \perp}/{P}_{h\perp}$. 
Hereby ${\bm p}_T$ is 
the transverse momentum of quark with respect to nucleon, 
${\bm k}_T$ is the transverse momentum of the fragmenting quark 
with respect to produced hadron. The notation is not unique. 
The one chosen in this work, in comparison to other works, is 
\begin{alignat}{3} 
	\mbox{transverse momentum in TMD:}\;\;\;&
    	\left[\bfkperp\right]_{\rm our}
    &=& 	\left[{\bm k}_\perp\right]_{\mbox{\tiny Ref.\cite{Anselmino:2011ch}}} \;
    =  	\;\;\;\;\;\left[{\bm p}_T\right]_{\mbox{\tiny Ref.\cite{Bacchetta:2006tn}}}\,,
	\\
	\mbox{transverse momentum in \ FF:}\;\;\;&
	\left[\bfpperp\right]_{\rm our}
    &=& 	\left[{\bm p}_\perp\right]_{\mbox{\tiny Ref.\cite{Anselmino:2011ch}}} \;
    =  	-z\left[{\bm k}_T\right]_{\mbox{\tiny Ref.\cite{Bacchetta:2006tn}}}\,, 
	\\
	\mbox{transverse hadron momenta:}\;\;\;&
    	\left[\bfPhperp\right]_{\rm our}
    &=&	\left[{\bm P}_T\right]_{\mbox{\tiny Ref.\cite{Anselmino:2011ch}}} \;
    =  	\;\;\;\left[{\bm P}_{h\perp}\right]_{\mbox{\tiny Ref.\cite{Bacchetta:2006tn}}}\,.
\end{alignat}
Notice that 
$\left[\bfpperp\right]_{\rm our}=
-z\left[{\bm k}_T\right]_{\mbox{\tiny Ref.\cite{Bacchetta:2006tn}}}$ 
is the transverse momentum the hadron acquires in the fragmentation process.
The normalization for unpolarized fragmentation functions is 
\be
	D_1^a(z) 
	= \left[\,\int d^2{\bm P}_\perp D_1^a(z,P_\perp^2)\right]_{\rm our}
	= \left[z^2\int d^2{\bm k}_TD_1^a(z, z^2  k_T^2)
	  \right]_{\mbox{\tiny Ref.~\cite{Bacchetta:2006tn}}}
	\label{eq:D1}\;. 
\ee
The ``Amsterdam'' convolution integral (\ref{eq:convolution-Amsterdam})
reads in our notation
\be
	{\cal C}\bigl[ w\slim f\slim D \bigr]
	=  x \,
	\sum_a e_a^2 \int d^2 \bfkperp\,  d^2 \bfpperp^{ }
	\, \delta^{(2)}\bigl(z\bfkperp + \bfpperp - \bfPhperp \bigr)
	\,w\left(\bfkperp,-\frac{\bfpperp^{ }}{z}\right)
	f^a( x ,\kperp^2)\,D^a(z,\pperp^2) . \label{eq:convolution_our}
\ee

\subsection{Gaussian Ansatz}

For a generic TMD and FF the Gaussian Ansatz is given by 
\be
    f^a(x,\kperp^2)=
    f^a(x)\,\frac{\exp(-\kperp^2/\la\kperp^2\ra)}{\pi\la\kperp^2\ra}\,,\;\;\;
    D^a(z,\pperp^2)=
    D^a(z)\,\frac{\exp(-\pperp^2/\la\pperp^2\ra)}{\pi\la\pperp^2\ra}
\ee
where 
$\la\kperp^2\ra$ could be $x$--dependent,  
and $\la\pperp^2\ra$ $z$--dependent. 
Both could be flavor-dependent.
The variable $\pperp$ is convenient because phenomenological experience 
shows that $\pperp$ in $D_1^{q/h}(z,\pperp^2)$ exhibits a Gaussian distribution 
with weakly $z$--dependent Gaussian width. The distribution of transverse 
momenta in $\left[D^a(z,z^2 k_T^2)\right]_{\mbox{\tiny Ref.\cite{Bacchetta:2006tn}}}$ 
would require a strongly $z$--dependent Gaussian width. It is a matter of 
taste which one prefers to use.

It is convenient to work with transverse moments of TMDs and FFs
which are defined, and in the Gaussian model given by
\begin{alignat}{6}
	f^{(n)}(x) &=& \int d^2\bfkperp
	\left(\frac{\kperp^2}{2M^2}\right)^{\!\!n}\; f(x, \kperp^2)
	\;&\stackrel{\rm Gauss}{=}&\;
	\frac{n!\,\la \kperp^2\ra^n}{ 2^n\,M_N^{2n}}\, f(x),\;\;\;\nonumber\\
	D^{(n)}(z) &=& \int d^2\bfpperp
	\left(\frac{\pperp^2}{2 z^2 m_h^2}\right)^{\!\!n}\; D(z, \pperp^2)
	\;&\stackrel{\rm Gauss}{=}&\;
	\frac{n!\,\la \pperp^2\ra^n}{ 2^n_{ }z^{2n}_{ }\,m_{h}^{2n}} D(z)\,. \;\;\;
	\label{eq:moments}
\end{alignat}
It is important to keep in mind that these objects are well-defined
in the Gaussian model. However, in QCD and even in simple models
\cite{Avakian:2010br,Schweitzer:2012hh} one faces issues with UV 
divergences and has to carefully define how to deal with them. 

In Eqs.~(\ref{eq:moments}) the Gaussian dependence is factorized 
from $x$ or $z$ dependence and parametrizations are made with 
respect to either $f(x)$ or $D(z)$. As we saw in Appendix 
\ref{App:basis} some TMD functions are parametrized with higher 
moments directly as operator product expansion of TMDs may start 
from higher twist matrix element instead of the usual twist-2 one. 
In those cases equivalent formulas to Eqs.~(\ref{eq:moments}) can 
be easily derived.

\newpage
\subsection{Gaussian Ans\"atze for the derived TMDs used in this work}
\label{App-B:Gauss-Ansatz-non-basis-TMDs}

The Gaussian Ans\"atze for the 8 basis functions were shown 
in Eqs.~(\ref{Eq:Gauss-f1}--\ref{Eq:Gauss-h1Tperp}) in 
Sec.~\ref{Sec-4.4:evaluation}, and the pertinent parameters
and parametrizations were reviewed in App.~\ref{App:basis}.
Besides the basis function we also need Ans\"atze for the 
following TMDs as listed below:
\begin{subequations}\begin{align}
	g_{1T}^{\perp q}(x,\kperp^2) 
	  &=	g_{1T}^{\perp (1) q}(x)\,
		\frac{2 M_N^2}{\pi \avkperp_{g_{1T}^\perp}^2}\,
		e^{-\kperp^2/{\avkperp_{g_{1T}^\perp}}}  \,, 
	  && 	\mbox{cf.\ Sec.~\ref{Sec-6.1:FLTcosphi-phiS},}
		\label{eq:g1t}\\
	h_{1L}^{\perp a}(x,\kperp^2) 
	  &= 	h_{1L}^{\perp (1) a}(x)\,
		\frac{2 M_N^2}{\pi{\avkperp_{h_{1L}^\perp}^2}}\,
		e^{-\kperp^2/{\avkperp_{h_{1L}^\perp}^2}}
	  && 	\mbox{cf.\ Sec.~\ref{Sec-6.2:FULsin2phi},}	
		\label{eq:h1l_final}\\
	g^{q}_{T}(x,\kperp^2) 
	  &=	g^{q}_{T}(x)\,\frac{1}{\pi \avkperp_{g_{T}}}\,
		e^{-\kperp^2/{\avkperp_{g_{T}}}}\;,
	  && 	\mbox{cf.\ Sec.~\ref{Sec-7.2:FLTcosphiS},}	
		\label{eq:gtnew} \\
	g^{\perp q}_{T}(x,\kperp^2) 
	  &= 	g^{\perp (2) q}_{T}(x)\,\frac{2 M^4}{\pi \avkperp^3_{g_{T}^{\perp} }}\,
		e^{-\kperp^2/{\avkperp_{g_{T}^{\perp} }}}\;,
	  && 	\mbox{cf.\ Sec.~\ref{Sec-7.3:FLTcos2phi-phiS},}	
		\label{eq:gtperpnew} \\
	g_{L}^{\perp q}(x,\kperp^2) 
	  &=	g_{L}^{\perp (1) q}(x)\,
		\frac{2 M_N^2}{\pi \avkperp_{g_{L}^\perp}^2}\,
		e^{-\kperp^2/{\avkperp_{g_{L}^\perp}}}  \,, 
	  && 	\mbox{cf.\ Sec.~\ref{Sec-7.3:FLLcosphi},}
		\label{eq:gLperp} \\
	h_L^q(x,\kperp^2) 
	  &= 	h_L^q(x) \,\frac{1}{\pi \avkperp_{h_{L}}}\,
		e^{-\kperp^2/{\avkperp_{h_{L}}}}\;,
	  && 	\mbox{cf.\ Sec.~\ref{Sec-7.4:FULsinphi},}
		\label{eq:hLnew}\\
	h_T^{\perp q}(x,\kperp^2) 
	  &=	h_T^{\perp (1) q}(x)\,\frac{2 M^2}{\pi \avkperp_{h_{T}^\perp}^2}\,
	  	e^{-\kperp^2/{\avkperp_{h_{T}^\perp}}}\;,
	  && 	\mbox{cf.\ Sec.~\ref{Sec-7.6:FUTsinphiS},}
		\label{eq:hTperpnew}\\
	h_T^{q}(x,\kperp^2) 
	  &=	h_T^{(1) q}(x) \,\frac{2 M^2}{\pi \avkperp_{h_{T}}^2}\,
		e^{-\kperp^2/{\avkperp_{h_{T}}}}\;,
	  && 	\mbox{cf.\ Sec.~\ref{Sec-7.6:FUTsinphiS},}
		\label{eq:hTnew}\\
	f^{\perp q}_{T}(x,\kperp^2) 
	  &= 	f^{\perp (2) q}_{T}(x)\,\frac{2 M^4}{\pi \avkperp^3_{f_{T}^{\perp} }}\,
		e^{-\kperp^2/{\avkperp_{f_{T}^{\perp} }}}\;,
	  && 	\mbox{cf.\ Sec.~\ref{Sec-7.8:FUTsin2phi-phiS},}
		\label{eq:ftperpnew}\\
	f^{\perp q}(x,\kperp^2) 
	  &= 	f^{\perp (1) q}(x) \,\frac{2 M^2}{\pi \avkperp_{f^\perp}^2}\,
		e^{-\kperp^2/{\avkperp_{f^\perp}}}\;,
	  && 	\mbox{cf.\ Sec.~\ref{Sec-7.7:FUUcosphi},}
		\label{eq:fperpnew}
\end{align}\end{subequations}

\newpage
\subsection{Comment on TMDs subject to the sum rules (\ref{Eq:sum-rules-T-odd})}
\label{App-B:comment-Todd-twist-3}

In this section we comment on the twist-3 TMDs 
$f_T^q(x,\kperp)$,  $h^q(x,\kperp)$, $e_L^q(x,\kperp)$,  
which are T-odd, appear in the decompositions of the correlator with no 
explicit $k_\perp^j$--prefactors, and would have colinear PDF counterparts. 
But T-odd PDFs are forbidden by time-reversal and parity invariance of 
strong interactions, which dictate the sum rules (\ref{Eq:sum-rules-T-odd}), 
see Sec.~\ref{Sec-3.8:limitations}. 
Such TMDs could be described by functions with a node in 
$k_\perp$\footnote{The possibility of TMDs with nodes is not unrealistic.
	For instance in the covariant parton model the helicity TMDs 
	exhibit nodes for the $u$-- and $d$--flavor \cite{Efremov:2010mt}.
	We will have to revise our description of $g_1^q(x,\kperp)$ 
	in Eq.~(\ref{Eq:Gauss-g1}) and App.~\ref{App:basis-g1} to something 
	of the type (\ref{Eq:multiple-Gauss}), if that prediction is 
	confirmed experimentally. }
such that they can integrate to zero in Eq.~(\ref{Eq:sum-rules-T-odd}).
A single Gaussian has no node and is not adequate for that.
However, one could work with a superposition of Gaussians
with different widths, 
\begin{alignat}{1}
	x \, f_T^q(x,\kperp) =  - \, f_{1T}^{\perp (1)q}(x)\;
	&\sum\limits_{i=1}^{n} a_i\;
	\frac{\exp(-\kperp^2/\la\kperp^2\ra_i^{ })}{\pi\la\kperp^2\ra_i^{ }}\,,
	\label{Eq:multiple-Gauss}\\
	&\sum\limits_{i=1}^n a_i = 0\,, \;\;
	\la\kperp^2\ra_i^{ }\neq\la\kperp^2\ra_j^{ }\;\;\forall\;i\neq j,
	\; 1\le i,\,j\le n,\;n\ge 2.\nonumber
\end{alignat}
Notice that in (\ref{Eq:multiple-Gauss}) we cannot write ``$f_T^q(x)$'' 
which would be zero according to (\ref{Eq:sum-rules-T-odd}) and we
explore here the WW-type approximation (\ref{Eq:WW-type-fT}). 
The minimal choice would be $n=2$ with $a_1=-a_2=1$ and
$\la\kperp^2\ra_1^{ } = \la\kperp^2\ra_{f_{1T}^\perp}$ to make use 
of the theoretical guidance provided by the WW-type approximation 
(\ref{Eq:WW-type-fT}). 
The second Gaussian width $\la\kperp^2\ra_2^{ }$ could be chosen
very large $\la\kperp^2\ra_2^{ } \gg \la\kperp^2\ra_{f_1^\perp}$ to
model the Gaussian description of $f_T^{q}(x,\kperp)$ similar to 
that of $f_{1T}^{\perp(1)q}(x,\kperp^2)$ at intermediate $\kperp$.
A very large parameter $\la\kperp^2\ra_2^{ }$ could be thought of as
a relict which enters in the sum rule (\ref{Eq:sum-rules-T-odd}) 
where the $\kperp$--integration formally extends up to infinity 
where the TMD description does not apply. The theoretical
understanding of higher--twist TMDs is too limited at
the present stage, but in principle this could be a 
pragmatic way of modelling the TMD $f_T^q(x,\kperp)$
and analogously $h^q(x,\kperp)$, $e_L^q(x,\kperp)$.

\newpage
\subsection{Convolution integrals in Gaussian Ansatz}
\label{App:convol-details}

Solving the convolution integrals relevant for SIDIS in the 
Gaussian Ansatz yields 
\begin{subequations}\label{Eq:Gaussian-integrals-RAW}\ba
  {\cal C}\bigl[\;\omega^{\{0\}}\, f \, D \bigr]
    &=&	\phantom{-}\,u \;{\cal G}(\Phperp)\\
  {\cal C}\bigl[ \omega^{\{1\}}_{\rm A} \, f \, D \bigr]
    &=&	\phantom{-}\,u \;{\cal G}(\Phperp)\;
	\biggl(\frac{z\Phperp}{m_h}\biggr)\;\frac{\la\pperp^2\ra}{z^2\lambda}\\
  {\cal C}\bigl[\omega^{\{1\}}_{\rm B} \, f \, D \bigr]
    &=&	- \,u \;{\cal G}(\Phperp) \; 
	\biggl(\frac{z\Phperp}{M_N}\biggr)\;\frac{\la\kperp^2\ra}{\lambda}\\
  {\cal C}\bigl[ \omega^{\{2\}}_{\rm A} \, f \, D \bigr]
    &=&	\phantom{-}\,u \; {\cal G}(\Phperp)\;
	\frac{\la\kperp^2\ra\la\pperp^2\ra}{\lambda M_N m_h}
	\left(-1+\frac{2\Phperp^2}{\lambda}\right)\\
  {\cal C}\bigl[ \omega^{\{2\}}_{\rm B} \, f \, D \bigr]
    &=&	\phantom{-}\,u \; {\cal G}(\Phperp)\;
	\frac{\la\kperp^2\ra\la\pperp^2\ra}{\lambda M_N m_h}
	\left(1-\frac{\Phperp^2}{\lambda}\right)\\
  {\cal C}\bigl[ \, \omega^{\{2\}}_{\rm AB} \: f \, D \bigr]
    &=&	\phantom{-}\,u \; {\cal G}(\Phperp)\;
	\biggl(\frac{z\Phperp}{M_N}\biggr)
	\biggl(\frac{z\Phperp}{M_h}\biggr)
	\;\frac{\la\kperp^2\ra}{\lambda}
	\;\frac{\la\pperp^2\ra}{z^2\lambda}\\
  {\cal C}\bigl[ \, \omega^{\{2\}}_{\rm C} \: f \, D \bigr]
    &=&	\phantom{-}\,\frac{u}{2} \; {\cal G}(\Phperp)\;
	\biggl(\frac{z\Phperp}{M_N}\biggr)
	\biggl(\frac{z\Phperp}{M_N}\biggr)
	\;\frac{\la\kperp^2\ra}{\lambda}
	\;\frac{\la\kperp^2\ra}{\lambda}\\
  {\cal C}\bigl[ \, \omega^{\{3\}}_{\rm { }} \: f \, D \bigr]
    &=&	\phantom{-}\,\frac{u}{2} \; {\cal G}(\Phperp)\;
	\biggl(\frac{z\Phperp}{M_N}\biggr)
	\biggl(\frac{z\Phperp}{M_N}\biggr)
	\biggl(\frac{z\Phperp}{m_h}\biggr)
	\;\frac{\la\kperp^2\ra}{\lambda}
	\;\frac{\la\kperp^2\ra}{\lambda}
	\;\frac{\la\pperp^2\ra}{z^2\lambda} 
\ea\end{subequations}
with the $\omega^{\{n\}}_i$ as defined in Eq.~(\ref{Eq:wi}), 
and we introduced the abbreviations
\be
	u=x\,\sum_a e_a^2 f^a(x)D^a(z) \,, \;\;\;
	{\cal G}(\Phperp)=\frac{\exp(-\Phperp^2/\lambda)}{\pi\lambda}\,,\;\;\;
	\lambda = z^2\la\kperp^2\ra+\la\pperp^2\ra\,,
\ee
with the normalization $\int d^2\Phperp\,{\cal G}(\Phperp) = 1$.
It is important to keep in mind that strictly speaking 
${\cal G}(\Phperp) = {\cal G}(\Phperp,x,z)$ {\it also} depends
on $x$ and $z$.
The ``non-compact'' notation in Eqs.~(\ref{Eq:Gaussian-integrals-RAW}) 
was chosen to display the pattern. The masses $M_N$ or $m_h$ in the 
denominators of the $\Phperp$ indicate the ``origins'' of the
contributions: due to intrinsic $\kperp$ from target, due to 
transverse momenta $\pperp$ acquired during fragmentation, or both.
The weight $\omega^{\{2\}}_{\rm B}$ is the only which enters
cross sections and does not have a homogeneous scaling in $\Phperp$.

For practical application it is convenient to absorb as many 
(Gaussian model) parameters as possible into expressions that can
be more easily fitted to data. One way to achieve this is to make 
use of the transverse moments (\ref{eq:moments}). 
We introduce the following abbreviations
\begin{alignat}{6}
u^{\{1\}}_{\rm A} 	&=& \;x\,\sum_a e_a^2 f^{   a}(x)D^{(1)a}(z) \,, &\hspace{1cm}&
u^{\{1\}}_{\rm B} 	&=& \;x\,\sum_a e_a^2 f^{(1)a}(x)D^{   a}(z) \,, \\
u^{\{2\}}_{\rm AB}	&=& \;x\,\sum_a e_a^2 f^{(1)a}(x)D^{(1)a}(z) \,, &\hspace{1cm}&
u^{\{2\}}_{\rm C}	&=& \;x\,\sum_a e_a^2 f^{(2)a}(x)D^{   a}(z) \,, \\
u^{\{3\}}_{\rm C}	&=& \;x\,\sum_a e_a^2 f^{(2)a}(x)D^{(1)a}(z) \,. &\hspace{1cm}& &&
\end{alignat}
In this notation the results in Eqs.~(\ref{Eq:Gaussian-integrals-RAW}) 
read 
\begin{subequations}\label{Eq:Gaussian-integrals-working}\ba
  {\cal C}\bigl[ \omega^{\{1\}}_{\rm A} \, f \, D \bigr]
    &=&	\phantom{-}\,u^{(1)}_{\rm A} \;{\cal G}(\Phperp)\;
	\biggl(\frac{z\Phperp}{m_h}\biggr)\;\frac{2m_h^2}{\lambda}\\
  {\cal C}\bigl[ \omega^{\{1\}}_{\rm B} \, f \, D \bigr]
    &=&	- \,u^{(1)}_{\rm B} \;{\cal G}(\Phperp) \; 
	\biggl(\frac{z\Phperp}{M_N}\biggr)\;\frac{2M_N^2}{\lambda}\\
  {\cal C}\bigl[ \omega^{\{2\}}_{\rm B} \, f \, D \bigr]
    &=&	\phantom{-}\, u^{(2)}_{\rm B}\; {\cal G}(\Phperp)\;
	\frac{4z^2 m_h\,M_N}{\lambda}
	\left(1-\frac{\Phperp^2}{\lambda}\right)\\
  {\cal C}\bigl[ \, \omega^{\{2\}}_{\rm AB} \: f \, D \bigr]
    &=&	\phantom{-}\,u^{(2)}_{\rm AB} \; {\cal G}(\Phperp)\;
	\biggl(\frac{z\Phperp}{M_N}\biggr)
	\biggl(\frac{z\Phperp}{M_h}\biggr)
	\;\frac{2M_N^2}{\lambda}
	\;\frac{2m_h^2}{\lambda}\\
  {\cal C}\bigl[ \, \omega^{\{2\}}_{\rm C} \: f \, D \bigr]
    &=&	\phantom{-}\,\frac{u^{(2)}_{\rm C}}{2} \; {\cal G}(\Phperp)\;
	\biggl(\frac{z\Phperp}{M_N}\biggr)
	\biggl(\frac{z\Phperp}{M_N}\biggr)
	\;\frac{2M_N^2}{\lambda}
	\;\frac{2M_N^2}{\lambda}\\
  {\cal C}\bigl[ \, \omega^{\{3\}}_{\rm { }} \: f \, D \bigr]
    &=&	\phantom{-}\,\frac{u^{(3)}_{\rm  }}{2} \; {\cal G}(\Phperp)\;
	\biggl(\frac{z\Phperp}{M_N}\biggr)
	\biggl(\frac{z\Phperp}{M_N}\biggr)
	\biggl(\frac{z\Phperp}{m_h}\biggr)
	\;\frac{2M_N^2}{\lambda}
	\;\frac{2M_N^2}{\lambda}
	\;\frac{2m_h^2}{\lambda} 
\ea\end{subequations}
In this notation the results in Eqs.~(\ref{Eq:Gaussian-integrals-RAW}) 
read 
\be
	{\cal C}\bigl[ \omega^{\{n\}}_{i} \, f \, D \bigr]
	= u^{(n)}_{i} \;{\cal G}(\Phperp)\;\times
	\left[
	\delta_{n2}\,\delta_{i\rm B}\;a^{(2)}_{B} + b^{(n)}_{i} 
	\left(\frac{z\Phperp}{\lambda}\right)^{\!\!n}\;\right]
\ee
with
\begin{alignat}{7}
	b^{(0)}_{\rm  } 	&= 1 \, , \\
	b^{(1)}_{\rm A} 	&= 2m_h	\, , &
	b^{(1)}_{\rm B} 	&= 2M_N	\, , \\
	a^{(2)}_{\rm B}	&= 4M_Nm_h\lambda^{-1}\,z^2  \, , \;\;\;\;\;\; & 
	b^{(2)}_{\rm AB} 	&= -\,b^{(2)}_{\rm B}
			 = 4M_Nm_h 	\, , \;\;\;\;\;\; & 
	b^{(2)}_{\rm C} 	&= M_N^2  \, ,\\
	b^{(3)}_{\rm  } 	&= 2M_N^2m_h \, .
	\label{Eq:Gaussian-integrals-working-III}
\end{alignat}
Finally, integrating out transverse hadron momenta yields
\be
	\int d^2\Phperp\;
	{\cal C}\bigl[\,\omega^{\{n\}}_{i} \, f \, D \bigr] 
	= u^{(n)}_{i} \;c^{(n)}_{i}\;\biggl(\frac{z}{\lambda^{1/2}}\biggr)^{\!\!n}
\ee
with
\begin{alignat}{7}
	c^{(0)}_{\rm  } 	&= 1 \, , \\
	c^{(1)}_{\rm A} 	&= \sqrt{\pi}\,m_h	\, , &
	c^{(1)}_{\rm B} 	&= \sqrt{\pi}\,M_N	\, , \\
	c^{(2)}_{\rm AB} 	&= 4M_Nm_h		\, , \;\;\;\;\;\; & 
	c^{(2)}_{\rm C} 	&= M_N^2 		\, , \;\;\;\;\;\;\;\;\;\; & 
	c^{(2)}_{\rm B}	&= 0 			\, ,\\
	c^{(3)}_{\rm  } 	&= {\textstyle\frac32}\sqrt{\pi} \,M_N^2m_h 
	\, . \;\;\;     &
			&
	\label{Eq:Gaussian-integrals-working-IV}
\end{alignat}

\newpage
\section{Mathematica package}
\label{app:package}

A package available to a wide physics community is needed for several 
reasons. First of all it is intended to facilitate visualization, 
reproduction, and verification of our results presented in this paper. 
Second, the package may prove useful for \ps{experimentalists in need of}
estimates of measured quantities for certain kinematics of a particular 
experiment. Third, the \ps{phenomenological workers in the} community 
will have an easy access to our results for comparison and/or use in 
phenomenological or theoretical papers. The authors believe that the 
open source access to the codes used in a project is the right way of 
handling scientific information in the 21$^{st}$ century. \ps{For example,}
collinear parton distribution functions are well tabulated in 
the \texttt{LHAPDF} project \cite{Buckley:2014ana}, but collinear 
fragmentation function still need a dedicated repository. There exist 
already several databases \ps{which are} being developed for extensions 
of collinear one-dimensional picture of the nucleon structure.  
Generalized Parton Distributions are present in the framework 
called \texttt{PARTONS} \cite{Berthou:2015oaw}, several extractions 
of Transverse Momentum Dependent distributions are tabulated in 
\texttt{TMDlib} project \cite{Hautmann:2014kza}. 

If you use this package, please, \ps{cite} this paper and the link to 
the repository. Please, acknowledge using any of the extracted TMDs 
\ps{presented} in this package. The corresponding \texttt{bibitem} 
codes are \ps{included} in the description of each TMD function. 

The source code of the project can be downloaded or cloned from the 
open source repository: \texttt{https://github.com/prokudin/WW-SIDIS}. 
Using \texttt{Mathematica}  run the  file named \texttt{example.nb}, 
you will see an example of usage of the code, code description and 
several visualizations of functions, asymmetries, and SIDIS cross-sections. 
Make sure that you do not move \texttt{example.nb} to another directory 
as it uses grids for TMDs, PDFs, and FFs.

SIDIS cross section examples are presented in Fig.~\ref{fig:package} .


\begin{figure}[ht]
\centering

\vspace{5mm}

\includegraphics[width=0.45\textwidth]{\FigPath/cross_section_phih.pdf} 
\includegraphics[width=0.45\textwidth]{\FigPath/cross_section_phiS.pdf} 
\caption{\label{fig:package} 
	Left panel:  
	$\phi_h$ dependence of the SIDIS cross section at $\phi_S = \pi/2$. 
	Right panel:  
	$\phi_S$ dependence of the SIDIS cross section at $\phi_h = 0$. 
	Also $x = 0.3$, $z=0.2$, $\Phperp = 0.5$ GeV, 
	$Q^2$ = 2.4 GeV$^2$, and the beam energy $E_{beam}$ = 5.7 GeV, 
	averaged over beam helicities,  transverse proton target polarization.}
\end{figure}



\newpage
\bibliography{\BibPath/biblio_ww}

\end{document}