\documentclass[a4paper,11pt]{article}
\bibliographystyle{JHEP}
\usepackage{jheppub} 
\begin{document}

\section*{REPLY TO THE EDITOR}

Dear Editor,

\ \\
We would like to thank the Referee for careful reading, 
constructive criticism, and helpful recommendations. We 
revised the manuscript closely following the Referee's
suggestions in all points.

\ \\
Sincerely,

\ \\
Authors \\


\section*{REPLY TO THE REFEREE}

We would like to thank the Referee for careful reading, 
constructive criticism, and the many helpful recommendations. 
We revised the manuscript closely following the Referee's
suggestions  as detailed out in the attached ``Summary of Changes.''
Below we respond to several specific points in the report.

\begin{itemize}

\item	We agree that the $Q^2$ evolution is an important 
	limiting factor, and added a brief comment in the
	Introduction as well as a more detailed paragraph
	in Sec.~3.8.

\item	The preliminary COMPASS data shown in Fig.~12 and
	several subsequent figures have been released and
	can be shown in our work under the condition that
	we show these Figures exactly as provided to us
	by the COMPASS Collaboration. This means that at
	this point we are not allowed to include the 
	results of our calculations in these plots, and
	must show them separately. These data will 
	presumably be unconditionally released at some later 
	time in the summer of this year. We added explanatory
	remarks in Fig.~12 and other figures where this applies.
	On the one hand, this makes the presentation of the
	results and the formatting of the figures cumbersome
	in many cases. On the other hand, we are grateful
	to the COMPASS Collaboration for giving us this
	possibility to show their preliminary data.
	We are also grateful to the HERMES Collaboration
	for making their preliminary data available and
	allowing us to include our curves in those plots.
	We added a remark in the Acknowledgments.

\item	We understand the concerns of the Referee regarding the 
	subpercent results shown e.g.\ in Fig.~15. We had similar
	concerns but decided to show these results, even though
	subpercent asymmetries are not realistic predictions for 
	COMPASS. But the smallness of the effect shown in Fig.~15 is 
	not only due to the (in our approximation) small involved 
	TMDs but also due to the kinematical suppression in the
	COMPASS kinematics. The figure still gives insights e.g.\ 
	on the relative flavor dependence for the production of 
	positive vs negative pions in the WW-type approximation.

\item	Finally, we followed the recommendations of the Referee
	and removed App.~C on the mathematica package, unified 
	the notation for the masses of produced hadrons $m_h$, 
	specified the Mellin moments before Eq.~(3.8), fixed 
	the literal repetition on pages 16/17, improved
	non-optimal wording and fixed typos. It is inexplicable
	how some of these things could have escaped our attention 
	during proof-reading, and we are grateful for these corrections.

\end{itemize}




\section*{SUMMARY OF CHANGES}

\begin{itemize}

\item
Sec.1, page 3, the following paragraph and the new references
\cite{Ali:1991em,Koike:1994st,Balitsky:1996uh,Belitsky:1997zw} are added:

``The WW-type approximation is not preserved under $Q^2$ evolution.
 Some intuition can be obtained from the collinear case
 \cite{Ali:1991em,Koike:1994st,Balitsky:1996uh,Belitsky:1997zw}. 
 However, much less is known about the $k_\perp$-evolution especially at 
 subleading twist. More theoretical work is required here.''

\item
Sec.1, page 3, we replaced: 


``study of all SIDIS structure functions up to twist-3 in a unique approach.''
$\to$\\
``study of all SIDIS structure functions up to twist-3 evaluated within one
common systematic theoretical guideline''

\item
Sec.1, page 3, we replaced the paragraph:

	``In App.~C we describe an open-source package implemented
	in \texttt{Mathematica}~\cite{Mathematica} (already available)
	and \texttt{Python} (to be released in the near future) that is
	made publicly available on \texttt{github.com}: {\href{
	https://github.com/prokudin/WW-SIDIS}{
	https://github.com/prokudin/WW-SIDIS}}''
	
by:

	``An open-source package is available which allows one to visualize 
	and reproduce the results presented in this work, and may easily 
	be adapted by interested colleagues for their purposes
	\cite{Mathematica}.''

and we removed App.~C.

\item
Sec.~2.1, Eq. 2.2: we absorbed the definition of the unpolarized lepton-quark 
subprocess cross section in the overall cross section formulas.




\item
Sec.~3.3, page 14: The sentence before Eq.~(3.8) is modified as:

	``For the $n = 3$ Mellin moments (i.e.\ the lowest non-trivial ones 
	for these tilde-functions) it was found $\dots$''

\item
Sec.~3.5, page 16: In the second paragraph we added a sentence refering 
to the latest developments concerning lattice:

``For the latest developments we refer the interested reader to Refs. [70�78]''
 

\item
Sec.~3.5, page 17: The first instance of the repetitive sentence 
starting with ``This assumption holds $\dots$'' is removed.


\item
Sec.~3.8, page 22: The paragraph is added:

``As it was mentioned in the Introduction
one important limitation concerns the fact that the WW-type approximations
are not preserved under $Q^2$ evolution. Still some intuition can be 
obtained from the collinear case: the evolution equations for $g_T^a(x)$ 
and $h_L^a(x)$ exhibit complicated mixing patterns typical for higher 
twist functions, which simplify to DGLAP-type evolutions in the limit 
of a large number of colors $N_c$ and in the limit of large-$x$ 
\cite{Ali:1991em,Koike:1994st,Balitsky:1996uh,Belitsky:1997zw}. 
These evolution equations differ from those of the leading-twist
functions $g_1^a(x)$ and $h_1^a(x)$. 
However, since $Q^2$ varies moderately in the considered experiments 
(e.g.\ for common values of $x$ the $Q^2$ at COMPASS is only about a factor 2-3
larger than at HERMES), this point is not a major uncertainty in our study.
More theoretical work will be required to understand $k_\perp$-evolution 
effects of subleading twist TMDs in future experiments (EIC) 
covering kinematic regions that vary by orders of magnitude in $Q^2$.''

\item
Sec. 5.2, page 29:
``We remark that HERMES and COMPASS data also show flat 
$P_{hT}$-distributions \cite{Airapetian:2018rlq,Adolph:2016vou}.''

\item
Sec. 5.3, page 30: the sign change of the Sivers asymmetry in hadron-hadron collisions 
is pointed out, including references to recent experimental results by STAR and COMPASS:

``The Sivers function is predicted to enter the description of hadron-hadron 
collisions (with transversely polarized protons) with an opposite sign 
compared to SIDIS \cite{Collins:2002kn,Brodsky:2002cx,Brodsky:2002rv}.
Recent results on single-spin asymmetries in weak-boson production 
from RHIC \cite{Adamczyk:2015gyk} and Drell--Yan from COMPASS 
\cite{Aghasyan:2017jop,Parsamyan:2018zju} are consistent 
with this prediction.''

\item
Sec. 5.4, page 30:
``Transversity can also be accessed as a PDF in Drell--Yan or dihadron 
production \cite{Bacchetta:2002ux,Bacchetta:2003vn,Bacchetta:2011ip,
Bacchetta:2012ty,Radici:2015mwa,Radici:2018iag}."

\item 
(Caption of) Fig.~7, page 31, removed the debated statement about the different sign 
conventions and plotted the COMPASS data using the conventions used here:

``Collins asymmetry for a proton target 
	vs $x$ based on the fit \cite{Anselmino:2013vqa}.
	(a) $A_{UT,  \langle y\rangle}^{\sin(\phi_h+\phi_S)}$ in comparison
	to HERMES \cite{Airapetian:2010ds} data.
	(b) $A_{UT}^{\sin(\phi_h+\phi_S)}$ in comparison
	to COMPASS \cite{Adolph:2014zba} data''


\item
Sec.~5.6, page 33 (our correction): we added $\langle\dots\rangle$
in the following in-line formula  
$A_{UT, \langle y \rangle}^{\sin(3 \phi_h - \phi_S)}=\langle
(1-y)F_{UT}^{\sin(3 \phi_h - \phi_S)}\rangle/\langle(1-y + y^2/2)F_{UU}\rangle$
to indicate that the kinematic variable $y$ is averaged over.

\item
Sec.~5.6, page 33, the text is removed:

 "A notable exception is COMPASS, where the largest $x$--bins
 (where $Q^2$ is largest) bear the best hints on this TMD,
 see Fig.~9."


\item
Captions of Figs.~9, 11, 12, 15, 16, 17, 19:
we added explanations why our curves cannot be added
on the plots provided by the COMPASS Collaboration.

``
[We remark that in this and several subsequent figures we have 
        the permission to show the preliminary data \cite{Parsamyan:2013fia} 
	only in the official figures provided by the COMPASS collaboration
 	in (a,b), and have to display our results separately in (d).
	Notice also the different scale on the y-axis in panel (d) 
	as compared to (a,b).]
"
\item
Sect.~7.2, caption of Fig.~14: added  ``where a different scale is chosen to better visualize the theory curves.''

\item
Sect.~7.4, third line (our correction): $F_{UL}^{\cos\phi_h}$ changed to $F_{LL}^{\cos\phi_h}$

\item
Fig.~18 (our correction):
the previous Fig.~18c was showing an estimate of the asymmetry
from an earlier (less consistent and superseded) way of using
the Gaussian model and applying the WW-type approximation, and 
we removed this plot. In the presently adopted scheme this
asymmetry vanishes as described in the caption of Fig.~18
and in Sec.~7.6.

\item
Fig.~20 (our correction): HEPDATA were used for the COMPASS results, which we realized have a bug. The plot got replaced using the original data tables from the COMPASS paper.

\item
Conclusions, page 47, the text is removed:

 ``The classic WW approximation for $g_2(x)$ works with a relative
 accuracy of $\pm\,40\,\%$ or better. This is remarkable."

 The following phrase is modified:

 ``on the positive site we also observe no alarming hints" $\to$\\
 ``on the positive side we also observe no hints" 

\item 
The misprints are fixed, the consistent notation $m_h$ is used,
and several figures are rearranged and placed more accurately 
within the sections where they are discussed. We did minor
editing in several places (purely stylistic improvements), 
and used a $\log x$ scale also for the HERMES data at several places for consistency with COMPASS plots shown in the same figure. 



\item   Acknowledgments, we added the remark:

We thank the COMPASS and HERMES Collaborations for the 
permissions to show their preliminary data on several figures.

\end{itemize}

% REMARK: 
%
% "When submitting the document source (.tex) file to external parties, 
% it is strongly recommended that the BIBTEX .bbl file be manually copied 
% into the document (within the traditional LATEX bibliography environment) 
% so as not to depend on external files to generate the bibliography and to
% prevent the possibility of changes occurring therein"
%
% This means: 
% STEP 1: compile with \bibliography{biblio_ww} as usual, 
% STEP 2: uncomment \bibliography{biblio_ww}
% STEP 3: paste the contents of the file reply-to-referee-v2C.bbl
%
%
% STEP 1: done
%
% STEP 2: uncommenting
% \bibliography{biblio_ww}
%
% STEP 3: pasting (yeah, it works ;-)
%

\providecommand{\href}[2]{#2}\begingroup\raggedright\begin{thebibliography}{1}

\bibitem{Ali:1991em}
A.~Ali, V.~M. Braun and G.~Hiller, \emph{{Asymptotic solutions of the evolution
  equation for the polarized nucleon structure function $g_2 (x, Q^2)$}},
  \href{http://dx.doi.org/10.1016/0370-2693(91)90753-D}{\emph{Phys. Lett.} {\bf
  B266} (1991) 117--125}.

\bibitem{Koike:1994st}
Y.~Koike and K.~Tanaka, \emph{{$Q^2 $evolution of nucleon's chiral odd
  twist-three structure function: $h_L (x, Q^2)$}},
  \href{http://dx.doi.org/10.1103/PhysRevD.51.6125}{\emph{Phys. Rev.} {\bf D51}
  (1995) 6125--6138}, [\href{https://arxiv.org/abs/hep-ph/9412310}{{\tt
  hep-ph/9412310}}].

\bibitem{Balitsky:1996uh}
I.~I. Balitsky, V.~M. Braun, Y.~Koike and K.~Tanaka, \emph{{$Q^2$ evolution of
  chiral odd twist-three distributions $h_L(x,Q^2)$ and $e(x,Q^2)$ in the large
  $N_c$ limit}},
  \href{http://dx.doi.org/10.1103/PhysRevLett.77.3078}{\emph{Phys. Rev. Lett.}
  {\bf 77} (1996) 3078--3081},
  [\href{https://arxiv.org/abs/hep-ph/9605439}{{\tt hep-ph/9605439}}].

\bibitem{Belitsky:1997zw}
A.~V. Belitsky and D.~{M\"u}ller, \emph{{Scale dependence of the chiral odd
  twist-three distributions $h_L(x)$ and $e(x)$}},
  \href{http://dx.doi.org/10.1016/S0550-3213(97)00432-X}{\emph{Nucl. Phys.}
  {\bf B503} (1997) 279--308},
  [\href{https://arxiv.org/abs/hep-ph/9702354}{{\tt hep-ph/9702354}}].

\bibitem{Bacchetta:2002ux}
A.~Bacchetta and M.~Radici, \emph{{Partial-wave analysis of two-hadron
  fragmentation functions}},
  \href{http://dx.doi.org/10.1103/PhysRevD.67.094002}{\emph{Phys. Rev.} {\bf
  D67} (2003) 094002}, [\href{https://arxiv.org/abs/hep-ph/0212300}{{\tt
  hep-ph/0212300}}].

\bibitem{Bacchetta:2003vn}
A.~Bacchetta and M.~Radici, \emph{{Two-hadron semi-inclusive production
  including subleading twist contributions}},
  \href{http://dx.doi.org/10.1103/PhysRevD.69.074026}{\emph{Phys. Rev.} {\bf
  D69} (2004) 074026}, [\href{https://arxiv.org/abs/hep-ph/0311173}{{\tt
  hep-ph/0311173}}].

\bibitem{Bacchetta:2011ip}
A.~Bacchetta, A.~Courtoy and M.~Radici, \emph{{First Glances at the
  Transversity Parton Distribution through Dihadron Fragmentation Functions}},
  \href{http://dx.doi.org/10.1103/PhysRevLett.107.012001}{\emph{Phys. Rev.
  Lett.} {\bf 107} (2011) 012001}, [\href{https://arxiv.org/abs/1104.3855}{{\tt
  1104.3855}}].

\bibitem{Bacchetta:2012ty}
A.~Bacchetta, A.~Courtoy and M.~Radici, \emph{{First extraction of valence
  transversities in a collinear framework}},
  \href{http://dx.doi.org/10.1007/JHEP03(2013)119}{\emph{JHEP} {\bf 03} (2013)
  119}, [\href{https://arxiv.org/abs/1212.3568}{{\tt 1212.3568}}].


\bibitem{Radici:2015mwa}
M.~Radici, A.~Courtoy, A.~Bacchetta and M.~Guagnelli, \emph{{Improved
  extraction of valence transversity distributions from inclusive dihadron
  production}}, \href{http://dx.doi.org/10.1007/JHEP05(2015)123}{\emph{JHEP}
  {\bf 05} (2015) 123}, [\href{https://arxiv.org/abs/1503.03495}{{\tt
  1503.03495}}].

\bibitem{Radici:2018iag}
M.~Radici and A.~Bacchetta, \emph{{First Extraction of Transversity from a
  Global Analysis of Electron-Proton and Proton-Proton Data}},
  \href{http://dx.doi.org/10.1103/PhysRevLett.120.192001}{\emph{Phys. Rev.
  Lett.} {\bf 120} (2018) 192001},
  [\href{https://arxiv.org/abs/1802.05212}{{\tt 1802.05212}}].
\bibitem{Airapetian:2018rlq}
{\scshape HERMES} collaboration, A.~Airapetian et~al., \emph{{Longitudinal
  double-spin asymmetries in semi-inclusive deep-inelastic scattering of
  electrons and positrons by protons and deuterons}},
  \href{https://arxiv.org/abs/1810.07054}{{\tt 1810.07054}}.


\bibitem{Adolph:2016vou}
{\scshape COMPASS} collaboration, C.~Adolph et~al., \emph{{Azimuthal
  asymmetries of charged hadrons produced in high-energy muon scattering off
  longitudinally polarised deuterons}},
  \href{http://dx.doi.org/10.1140/epjc/s10052-018-6379-7}{\emph{Eur. Phys. J.}
  {\bf C78} (2018) 952}, [\href{https://arxiv.org/abs/1609.06062}{{\tt
  1609.06062}}].

\bibitem{Parsamyan:2013fia}
{\scshape COMPASS} collaboration, B.~Parsamyan, \emph{{Transverse spin
  asymmetries at COMPASS: beyond Collins and Sivers effects}}, {\emph{PoS} {\bf
  DIS2013} (2013) 231}, [\href{https://arxiv.org/abs/1307.0183}{{\tt
  1307.0183}}].


\bibitem{Mathematica}
A.~Prokudin and K.~Tezgin, ``{Open-source packages with implementations of
  SIDIS structure functions in the WW-type approximation are publicly available
  on \texttt{github.com}: \\ in \texttt{Mathematica, {V}ersion 11.3} on {\href{
  https://github.com/prokudin/WW-SIDIS}{
  https://github.com/prokudin/WW-SIDIS}},\\ in \texttt{Python} on {\href{
  https://jeffersonlab.github.io/jam3d/_build/html/index.html}{
  https://jeffersonlab.github.io/jam3d/\_build/html/index.html}}}.''

\bibitem{Collins:2002kn}
J.~C. Collins, \emph{{Leading-twist single-transverse-spin asymmetries:
  Drell--Yan and deep-inelastic scattering}}, {\emph{Phys. Lett.} {\bf B536}
  (2002) 43--48}, [\href{https://arxiv.org/abs/hep-ph/0204004}{{\tt
  hep-ph/0204004}}].

\bibitem{Brodsky:2002cx}
S.~J. Brodsky, D.~S. Hwang and I.~Schmidt, \emph{{Final-state interactions and
  single-spin asymmetries in semi-inclusive deep inelastic scattering}},
  {\emph{Phys. Lett.} {\bf B530} (2002) 99--107},
  [\href{https://arxiv.org/abs/hep-ph/0201296}{{\tt hep-ph/0201296}}].

\bibitem{Brodsky:2002rv}
S.~J. Brodsky, D.~S. Hwang and I.~Schmidt, \emph{{Initial-state interactions
  and single-spin asymmetries in Drell-Yan processes}},
  \href{http://dx.doi.org/10.1016/S0550-3213(02)00617-X}{\emph{Nucl. Phys.}
  {\bf B642} (2002) 344--356},
  [\href{https://arxiv.org/abs/hep-ph/0206259}{{\tt hep-ph/0206259}}].

\bibitem{Adamczyk:2015gyk}
{\scshape STAR} collaboration, L.~Adamczyk et~al., \emph{{Measurement of the
  transverse single-spin asymmetry in $p^\uparrow+p \to W^{\pm}/Z^0$ at RHIC}},
  \href{http://dx.doi.org/10.1103/PhysRevLett.116.132301}{\emph{Phys. Rev.
  Lett.} {\bf 116} (2016) 132301},
  [\href{https://arxiv.org/abs/1511.06003}{{\tt 1511.06003}}].

\bibitem{Aghasyan:2017jop}
{\scshape COMPASS} collaboration, M.~Aghasyan et~al., \emph{{First Measurement
  of Transverse-Spin-Dependent Azimuthal Asymmetries in the Drell--Yan
  Process}},
  \href{http://dx.doi.org/10.1103/PhysRevLett.119.112002}{\emph{Phys. Rev.
  Lett.} {\bf 119} (2017) 112002},
  [\href{https://arxiv.org/abs/1704.00488}{{\tt 1704.00488}}].

\bibitem{Parsamyan:2018zju}
{\scshape COMPASS} collaboration, B.~Parsamyan, \emph{{First measurement of
  transverse-spin-dependent azimuthal asymmetries in the Drell-Yan process}},
  \href{http://dx.doi.org/10.22323/1.297.0243}{\emph{PoS} {\bf DIS2017} (2018)
  243}, [\href{https://arxiv.org/abs/1801.01487}{{\tt 1801.01487}}].

\bibitem{Anselmino:2013vqa}
M.~Anselmino, M.~Boglione, U.~D'Alesio, S.~Melis, F.~Murgia and A.~Prokudin,
  \emph{{Simultaneous extraction of transversity and Collins functions from new
  SIDIS and $e^{+}e^{-}$ data}},
  \href{http://dx.doi.org/10.1103/PhysRevD.87.094019}{\emph{Phys. Rev.} {\bf
  D87} (2013) 094019}, [\href{https://arxiv.org/abs/1303.3822}{{\tt
  1303.3822}}].

\bibitem{Airapetian:2010ds}
{\scshape HERMES} collaboration, A.~Airapetian et~al., \emph{{Effects of
  transversity in deep-inelastic scattering by polarized protons}},
  \href{http://dx.doi.org/10.1016/j.physletb.2010.08.012}{\emph{Phys. Lett.}
  {\bf B693} (2010) 11--16}, [\href{https://arxiv.org/abs/1006.4221}{{\tt
  1006.4221}}].

\bibitem{Adolph:2014zba}
{\scshape COMPASS} collaboration, C.~Adolph et~al., \emph{{Collins and Sivers
  asymmetries in muonproduction of pions and kaons off transversely polarised
  protons}},
  \href{http://dx.doi.org/10.1016/j.physletb.2015.03.056}{\emph{Phys. Lett.}
  {\bf B744} (2015) 250--259}, [\href{https://arxiv.org/abs/1408.4405}{{\tt
  1408.4405}}].



\end{thebibliography}\endgroup


\end{document}

