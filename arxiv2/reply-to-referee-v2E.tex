\documentclass[a4paper,11pt]{article}
\bibliographystyle{JHEP}
\usepackage{jheppub} 
\begin{document}

\section*{REPLY TO THE EDITOR}

Dear Editor,

\ \\
We would like to thank the Referee for careful reading, 
constructive criticism, and helpful recommendations. We 
revised the manuscript closely following the Referee's
suggestions in all points.

\ \\
Sincerely,

\ \\
Prokudin's Eleven \\


\section*{REPLY TO THE REFEREE}

We would like to thank the Referee for careful reading, 
constructive criticism, and the many helpful recommendations. 
We revised the manuscript closely following the Referee's
suggestions  as detailed out in the attached ``Summary of Changes.''
Below we respond to several specific points in the report.

\begin{itemize}

\item	We agree that the $Q^2$ evolution is an important 
	limiting factor, and added a brief comment in the
	Introduction as well as a more detailed paragraph
	in Sec.~3.8.

\item	The preliminary COMPASS data shown in Fig.~12 and
	several subsequent figures have been released and
	can be shown in our work under the condition that
	we show these Figures exactly as provided to us
	by the COMPASS Collaboration. This means that at
	this point we are not allowed to include the 
	results of our calculations in these plots, and
	must show them separately. These data will 
	presumably be unconditionally released at some later 
	time in the summer of this year. We added explanatory
	remarks in Fig.~12 and other figures where this applies.
	On the one hand, this makes the presentation of the
	results and the formatting of the figures cumbersome
	in many cases. On the other hand, we are grateful
	to the COMPASS Collaboration for giving us this
	possibility to show their preliminary data.
	We are also grateful to the HERMES Collaboration
	for making their preliminary data available and
	allowing us to include our curves in those plots.
	We added a remark in the acknowledgments.

\item	We understand the concerns of the Referee regarding the 
	subpercent results shown e.g.\ in Fig.~15. We had similar
	concerns but decided to show these results, even though
	subpercent asymmetries are not realistic predictions for 
	COMPASS. But the smallness of the effect shown in Fig.~15 is 
	not only due to the (in our approximation) small involved 
	TMDs but also due to the kinematical suppression in the
	COMPASS kinematics. The figure still gives insights e.g.\ 
	on the relative flavor dependence for the production of 
	positive vs negative pions in the WW-type approximation.

\item	Finally, we followed the recommendations of the Referee
	and removed App.~C on the mathematica package, unified 
	the notation for the masses of produced hadrons $m_h$, 
	specified the Mellin moments before Eq.~(3.8), fixed 
	the literal repetition on pages 16/17, improved
	non-optimal wording and fixed typos. It is inexplicable
	how some of these things could have escaped our attention 
	during proof-reading, and we are grateful for these corrections.

\end{itemize}




\section*{SUMMARY OF CHANGES}

\begin{itemize}

\item
Sec.1, page 3, the following paragraph and the new references
\cite{Ali:1991em,Koike:1994st,Balitsky:1996uh,Belitsky:1997zw} are added:

"The WW-type approximation is not preserved under $Q^2$ evolution.
 Some intuition can be obtained from the collinear case
 \cite{Ali:1991em,Koike:1994st,Balitsky:1996uh,Belitsky:1997zw}. 
 However, much less is known about the $k_\perp$-evolution especially at 
 subleading twist. More theoretical work is required here."

\item
Sec.1, page 3, we replaced: 

"study of all SIDIS structure functions up to twist-3 in a unique approach."
$\to$\\
"study of all SIDIS structure functions up to twist-3 evaluated within one
common systematic theoretical guideline."

\item
Sec.1, page 3, we replaced the paragraph:

	In App.~C we describe an open-source package implemented
	in \texttt{Mathematica}~\cite{Mathematica} (already available)
	and \texttt{Python} (to be released in the near future) that is
	made publicly available on \texttt{github.com}: {\href{
	https://github.com/prokudin/WW-SIDIS}{
	https://github.com/prokudin/WW-SIDIS}}

by:

	An open-source package is available which allows one to visualize 
	and reproduce the results presented in this work, and may easily 
	be adapted by interested colleagues for their purposes
	\cite{Mathematica}.

and we removed App.~C.


\item
Sec.~3.3, page 14: The sentence before Eq.~(3.8) is modified as:

	For the $n = 3$ Mellin moments (i.e.\ the lowest non-trivial ones 
	for these tilde-functions) it was found $\dots$

\item
Sec.~3.5, page 17: The first instance of the repetitive sentence 
starting with ``This assumption holds $\dots$'' is removed.


\item
Sec.~3.8, page 20: The paragraph is added:

As it was mentioned in the Introduction one important limitation concerns 
the fact that the WW-type approximations are not preserved under $Q^2$ 
evolution. Still some intuition can be 
obtained from the collinear case: the evolution equations for $g_T^a(x)$ 
and $h_L^a(x)$ exhibit complicated mixing patterns typical for higher 
twist functions which simplify to DGLAP-type evolutions in the limit 
of a large number of colors $N_c$ and in the limit of large-$x$ 
\cite{Ali:1991em,Koike:1994st,Balitsky:1996uh,Belitsky:1997zw}. 
These evolution equations differ from those of the leading-twist
functions $g_1^a(x)$ and and $h_1^a(x)$. This point is of not much
practical relevance here, because the $Q^2$ varies only moderately
between JLab, HERMES, and COMPASS (except for the largest $x$-bins).
However, if for some reason the $\bar{q}gq$-terms were found to be
very small at one renormalization scale, it is not guaranteed the
WW-type approximation will work equally well also at other scales. 
 More theoretical work is required, especially also in order to 
understand $k_\perp$-evolution effects at subleading twist.


\item 
Caption of Fig.7, page 31, is modified as follows:

"(where $(-1)A_{UT}^{\sin(\phi_h+\phi_S)}$ is shown since COMPASS
 uses an opposite sign in its Collins asymmetry convention)."

\item
Sec.~5.6, page 32 (our correction): we added $\langle\dots\rangle$
in the following in-line formula  
$A_{UT, \langle y \rangle}^{\sin(3 \phi_h - \phi_S)}=\langle
(1-y)F_{UT}^{\sin(3 \phi_h - \phi_S)}\rangle/\langle(1-y + y^2/2)F_{UU}\rangle$
to indicate that the kinematic variable $y$ is averaged over.

\item
Captions of Figs.~9, 11, 12, 14, 15, 16, 19:
we added explanations why our curves cannot be added
on the plots provided by the COMPASS Collaboration.

\item
Fig.~18 (our correction):
the previous Fig.~18c was showing an estimate of the asymmetry
from an earlier (less consistent and superseded) way of using
the Gaussian model and applying the WW-type approximation, and 
we removed this plot. In the presently adopted scheme this
asymmetry vanishes as described in the caption of Fig.~18
and in Sec.~7.6.

\item
Sec.~5.6, page 33, the text is removed:

 "A notable exception is COMPASS, where the largest $x$--bins
 (where $Q^2$ is largest) bear the best hints on this TMD,
 see Fig.~9."

\item
Conclusions, page 47, the text is removed:

 "The classic WW approximation for $g_2(x)$ works with a relative
 accuracy of $\pm\,40\,\%$ or better. This is remarkable."

 The following phrase is modified:

 "on the positive site we also observe no alarming hints" $\to$\\
 "on the positive side we also observe no hints" 

\item 
The misprints are fixed, the consistent notation $m_h$ is used,
and several figures are rearranged and placed more accurately 
within the sections where they are discussed. We did minor
editing in several places (purely stylistic improvements).


\item   Acknowledgments, we added the remark:

We thank the COMPASS and HERMES Collaborations for the 
permissions to show their preliminary data on several figures.

\item	{\bf Harut:} Do we show something preliminary from JLab?\\
	If so, we should acknowledge JLab too!  ;-)

\item	{\bf Alexei:} please eloborate on Ref.~\cite{Mathematica}
	as needed.

\end{itemize}

% REMARK: 
%
% "When submitting the document source (.tex) file to external parties, 
% it is strongly recommended that the BIBTEX .bbl file be manually copied 
% into the document (within the traditional LATEX bibliography environment) 
% so as not to depend on external files to generate the bibliography and to
% prevent the possibility of changes occurring therein"
%
% This means: 
% STEP 1: compile with \bibliography{biblio_ww} as usual, 
% STEP 2: uncomment \bibliography{biblio_ww}
% STEP 3: paste the contents of the file reply-to-referee-v2C.bbl
%
%
% STEP 1: done
%
% STEP 2: uncommenting
% \bibliography{biblio_ww}
%
% STEP 3: pasting (yeah, it works ;-)
%

\providecommand{\href}[2]{#2}\begingroup\raggedright\begin{thebibliography}{1}

\bibitem{Ali:1991em}
A.~Ali, V.~M. Braun and G.~Hiller, \emph{{Asymptotic solutions of the evolution
  equation for the polarized nucleon structure function $g_2 (x, Q^2)$}},
  \href{http://dx.doi.org/10.1016/0370-2693(91)90753-D}{\emph{Phys. Lett.} {\bf
  B266} (1991) 117--125}.

\bibitem{Koike:1994st}
Y.~Koike and K.~Tanaka, \emph{{$Q^2 $evolution of nucleon's chiral odd
  twist-three structure function: $h_L (x, Q^2)$}},
  \href{http://dx.doi.org/10.1103/PhysRevD.51.6125}{\emph{Phys. Rev.} {\bf D51}
  (1995) 6125--6138}, [\href{https://arxiv.org/abs/hep-ph/9412310}{{\tt
  hep-ph/9412310}}].

\bibitem{Balitsky:1996uh}
I.~I. Balitsky, V.~M. Braun, Y.~Koike and K.~Tanaka, \emph{{$Q^2$ evolution of
  chiral odd twist-three distributions $h_L(x,Q^2)$ and $e(x,Q^2)$ in the large
  $N_c$ limit}},
  \href{http://dx.doi.org/10.1103/PhysRevLett.77.3078}{\emph{Phys. Rev. Lett.}
  {\bf 77} (1996) 3078--3081},
  [\href{https://arxiv.org/abs/hep-ph/9605439}{{\tt hep-ph/9605439}}].

\bibitem{Belitsky:1997zw}
A.~V. Belitsky and D.~{M\"u}ller, \emph{{Scale dependence of the chiral odd
  twist-three distributions $h_L(x)$ and $e(x)$}},
  \href{http://dx.doi.org/10.1016/S0550-3213(97)00432-X}{\emph{Nucl. Phys.}
  {\bf B503} (1997) 279--308},
  [\href{https://arxiv.org/abs/hep-ph/9702354}{{\tt hep-ph/9702354}}].

\bibitem{Mathematica}
A.~Prokudin and K.~Tezgin, ``{Open-source packages with implementations of
  SIDIS structure functions in the WW-type approximation are publicly available
  on \texttt{github.com}: \\ in \texttt{Mathematica, {V}ersion 11.3} on {\href{
  https://github.com/prokudin/WW-SIDIS}{
  https://github.com/prokudin/WW-SIDIS}},\\ in \texttt{Python} on {\href{
  https://jeffersonlab.github.io/jam3d/_build/html/index.html}{
  https://jeffersonlab.github.io/jam3d/\_build/html/index.html}}}.''

\end{thebibliography}\endgroup


\end{document}

